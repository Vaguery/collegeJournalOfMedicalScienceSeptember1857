% The next lines tell TeXShop to typeset with xelatex, and to open and save the source with Unicode encoding.

%!TEX TS-program = xelatex
%!TEX encoding = UTF-8 Unicode

\documentclass[9pt,a5paper]{memoir}
\usepackage{fontspec,xltxtra,xunicode}
\usepackage{hyperref}
\usepackage{lettrine}

\defaultfontfeatures{Mapping=tex-text}
\setromanfont[Mapping=tex-text, Ligatures={Common}]{P22 Foxtrot Pro}
\setsansfont[Mapping=tex-text, Ligatures={Common}]{Lucida Grande}


\frenchspacing
\plainfootnotes

%%
% inline semantic markup
%%
\newcommand{\booktitle}[1]{\textit{#1}}
\newcommand{\foreign}[1]{\textit{#1}}
\newcommand{\latinate}[1]{\textit{#1}}

%%
% sidenote macros
%%
\newcommand{\typo}[2]{\textbf{#2}\marginpar{\tiny{{#1}/? ---\textit{Ed.}}}}
\newcommand{\oldpage}[1]{\marginpar{\centering{}\tiny{pg. {#1}}}}

%%
% typographic symbol macros
%%
\newcommand{\prescription}{\textsf{℞}}
\newcommand{\ounce}{\textsf{℥}}
\newcommand{\dram}{\textsf{ʒ}}
\newcommand{\scruple}{\textsf{℈}}

\begin{document}

\chapter*{Transcriber's Note}This document is a draft, being re-typeset page by page, based on scanned and OCRed images of the original 1857 book's pages. See \href{http://github.com/Vaguery/collegeJournalOfMedicalScienceSeptember1857}{the github project} for more information, or to fork and work on the project.
\chapter*{Transcription begins:}

THE
COLLEGE JOURNAL
OF MEDICAL SCIENCE.

Vol. II.
SEPTEMBER, 1857.
No. 9.

ORIGINAL CONTRIBUTIONS.

\section*{Nature and Causes of Asiatic Cholera.}

by Prof.\ Michael von Visanik.

Translated from the German by T.\ C.\ Miller, M.\ D.

\textsc{Ever} since the first appearance of the epidemic cholera physicians
have endeavored to investigate and discover the inner nature and true
cause of the disease, that upon the knowledge thus obtained they might
base a rational mode of treatment.

But in accordance with the prevailing usages of the profession, the
nature and causes of this fell disease have remained unknown and the
attempted explanations have been based upon hypotheses and empirical
observation and not upon any rational and accurate course of
observation and reasoning. Some have explained the etiology as
a ``\textit{virus}'' others as ``\textit{humores alienati},'' and have contented themselves
with these unintelligible abstractions as sufficient to account for
the origin and duration of the contagion and disease. But we have
recently become dissatisfied with these phrases in place of ideas, and
since the researches of the pathologist and the chemist have given us
more definite information, the reformatory truths obtained from them
have enabled us to penetrate far more profoundly into the true nature
of this epidemic.

% \textsc{vol. ii. no. 9.---25.}
\endinput

It would prove fatiguing to the reader for me to enumerate all the
various hypotheses which have from time to time obtained in regard to
the etiology of cholera, and hence I shall only glance at a few of the
more prominent of them as prefatory to the opinions and conclusions to
which I have arrived.


It has been observed that fat, and particularly the fats that are to be
found in diet where the food has become sour and rancid, will if eaten
often produce symptoms very closely analagous to cholera, and from
this observation the conclusion has been drawn that the cholera miasm
is one produced by a specific decomposition of animal tissues forming
a combination of gases similar to those evolved in the decomposition
of sausages, and which is known as the \emph{sausage poison}. That this
view cannot be correct has become evident to nearly all.


Others have supposed that cholera is caused by what they have been
pleased to style the \emph{cholera-mite} a supposed microscopic animalcule
diffused in vast quantities through the air, the food and the drink, and
that these animalcules are the \emph{potentia nocens} of the disease. The
advocates of this hypothesis attack all other opinions and defend their
own with great violence, and are very strenuous in the advocacy of
what they assert to be the cause of cholera. Among those of this
class are many able microscopists, and yet they have neglected to bring
forward the only strong and indisputable evidence necessary to establish
the accuracy of their deductions. They are unable to bring forward
any person who has been so fortunate as ever to have seen this
wonderful cholera-mite.


Not a few have directed their attention mainly to the stomach and
the intestines and think they find in the vomiting and purging the true
explanation of the cause of epidemic cholera; which cause to them is
the irritative and congested condition of the alimentary track. On
this hypothesis have they based their high estimate of the value of
opium and have viewed it as a specific against the disease.


This view of the matter is so superficial and so illy sustained by the
symptoms of the disease and the results of treatment that most have
abandoned it. All who have had any experience in cholera and its
treatment will have observed that the danger of the attack is by no
means proportionate to the activity of the vomiting and purging, but
that it frequently appears extremely severe and fatal where but little
vomiting or purging have occurred. But this class of persons have
their attention so closely drawn to their fancied seat of the difficulty
that they never perceive these facts and never have the faintest glimpse
of the true cause of the disease.
\endinput

\oldpage{387}Some also, have considered cholera to be caused by some derangement
of the chylopoietic vicera, but the anatomical and autopsical examinations
have given no countenance to this conclusion.

The peculiar symptoms manifested in the asphyctic condition of
cholera has induced some to suppose that the whole difficulty arose
from a weakness or loss of functional power of the nerves, but more
especially of the spinal cord; but the revelations of the dead-house
have not confirmed these conclusions and did much to disprove the
accuracy of the opinion entertained.

More recently in their search for the seat and origin of cholera
physicians have been guided by what they have learned in regard to
the blood, its changes and decompositions, and they have observed that
in cholera there is a partial decomposition of the blood, with contemporaneous 
alteration in the walls of the capillaries by means of which
the \emph{serum sanguinis} in large quantities passes through their walls, or
is poured directly into the stomach and intestines, leading to the profuse
vomiting and purging by which this serum is removed entirely beyond
the organism. This decomposition of the blood and the outflow of
its watery portions is often produced with wonderful rapidity---while
the more solid portions, as the red-corpuscles, is retained in the vessels,
and thus the fluid is rendered thick, dark and very liable to stagnation,
to clog up and produce congestion of the vessels. The thick blood
also stagnates in the vessels of the skin, and causes the blue appearance
nearly always observed in that tissue. In the larger veins, and in the
brain, in the liver, and the spleen, and the lungs this stasis also occurs,
and hence the difficulty of breathing, the deafness, the aphonia, the
thirst, the scanty urine, the coldness and numbness, that accompanies
this disease.

Although this explanation is in accordance with the observations of
the profession, yet it may not satisfy all, for many deny that any explanation
can be given which shall prove satisfactory, and they desire
to know how it is, if the blood is \typo{really}{realy} separated, that a part of the
serum is not thrown into the cellular tissue, and not the whole of it
poured into the alimentary canal, or through the skin in profuse perspiration.
Why are not the pleural and abdominal cavities filled with this
fluid? why is the patient not attacked with hydro-thorax and anasarca?
are questions urged against this opinion.

There can be no doubt but the nerves which supply the vital force to
the walls of the blood-vessels are impaired during an attack of cholera
but those who entertained the opinion that the nervous power is diminished,
are divided as to which is the \emph{prime} cause of the difficulty, one\endinput
\oldpage{388}class supposing the atony of the vessels and the consequent out-flux
of the fluid impairs the nerves, while the other supposes that a poison
has been introduced into the system which acting directly on the brain
thus lessens the nerve power, and the loss of that power leads to the
atony of the blood-vessels and consequent exhalation of the serum.

There seems to be \emph{three} principle classes of opinions as to the primal
nature of cholera. 1st. A primary poisoning of the blood. 2nd. A
primary affection of the nerves. 3d. A primary local affection of the
alimentary canal.

From what has been said, we may perhaps draw the conclusion, that
cholera, like other epidemics, as scarlatina, measles, intermittent fever,
typhus fever, influenza, etc., owes its origin to a cause having a uniform
origin, or at least a uniform character, while, as in the other instances,
as to its peculiar nature, we may be entirely ignorant. Many physicians
have striven hard to learn the exact nature and character of this morbific
agent, and yet they have acknowledged a want of success.

With others, I too, have tried to solve this mystery and having had
an opportunity of observing personally more than two thousand cases
of cholera in different epidemics, I am led to present the following observations.

Cholera does not appear every where and at all times to possess precisely
the same characteristics. At one time it will appear mild and
very easily cured. But this slight form only appears in those persons
whose systems appear to have but a slight \emph{disposition} for the disease,
and hence it cannot exert as powerful an influence upon such as it does
upon those who are more predisposed to its attacks, and the disease will
take \emph{the form of cholera periculosa exquisita}.

Most of those who are disposed to inflammatory disease seem also
disposed to receive cholera, and hence the two diseases are often met
with in company. In ``\emph{cholera febrilis}'' there are several congestions
of the head, the lungs, or heart, in conjunction with the more ordinary
symptoms of cholera. In persons who have a predominating disposition
to vomit, the cholera will commence with vomiting, while with
those who are disposed to looseness of the bowels, it will commence
with a diarrhœa, while with those who are predisposed to the cholera,
and at the same time their nervous and arterial systems are equally
susceptible, the disease will take the form of \emph{cholera fulminitisima
asphyctica}.

Climate exercises upon any prevailing disease a powerful influence.
This is manifested in epidemic cholera. In some countries and climates,
it appears as \emph{cholera febrilis}, with intense congestions; in others\endinput
as \emph{cholera spasmodica}; while in other climates, vomiting and purging
are the most prominent features of the disease.

It is' also important, in the investigation of the cause of cholera, to
learn to distinguish between the genuine cholera and other forms of
disease, as well as to decide what results are produced from cholera
and what from other causes, and it is only those who have had considerable
experience in this disease who can always make this distinction.

The pathologico-anatomical, and the pathologico-chemical results of
the disease \typo{contribute}{coutribute} largely to the knowledge requisite to determine
the cause and nature of it, and they therefore must never be neglected
or overlooked. The post-mortem examinations and the examinations
of the secretions and excretions must be made, particularly the secretions
and excretions of the liver, the spleen, the kidneys, the stomach,
and the intestines, but more particularly the excretions of the kidneys,
and on the results of these examinations may we base our own view of
the nature and cause, as well as of the treatment of the diseases.

During the last epidemic the post-mortem examinations have furnished
in general and in particular, the same results as those obtained
during former epidemics. The skin of those who died of the disease has
been cyanotic, or blue colored, particularly the skin over the extremities.
The vessels in the sinuses and meninges of the brain have contained
much thick dark blood. The inner meninges have been congested, with
ecchymoses. The brain itself has been firm, and on intersection has
disclosed similar ecchymoses in its substance. The pleura and pericardium,
and all the serous membranes have a slippery feeling, showing a
separation of the delicate lining from the subjacent parts, and covered
with a glutenous albuminoid fluid. The lungs are dry and of a clear
red, and bloodless. The heart, particularly the left ventricle has been
drawn up, and in it and in the large vessels we have found a thick
black, tarry blood possessing little or no power of coagulation. The
liver is pale, and the gall-bladder filled with much dark bile. The
spleen is usually enlarged, dark red-brown, and its enveloping membrane
thrown into wrinkles. The stomach and intestines are filled up
with a rice-watery, or a bloody colored fluid, with the mucus membrane
of the stomach swollen and injected. The epithelium of the intestines
in nearly the whole extent is dead, and rubbed off; the mucus membrane
red, and the follicles swollen. The kidneys in most instances
presented distinctly the changes which are found in the disease known
as \emph{morbus Brightii}. This changed condition was particularly found in
those who had died of that form of cholera known as the \emph{cholera-typhus}.
The bladder was found contracted and empty.


In addition to the changes here specified others were noted but they
were supposed to be caused by the presence of some other modifying
disease, and hence not attributable to the cholera and not to be
accounted as a pathological result of the epidemic.

Pathologico-chemically, it was found that the blood was relatively
and absolutely poorer, or more deficient in water, having an appearance
resembling mud.

It was also quite deficient in alkalinity, particularly in the \typo{triple}{tripple}
phosphates, and the carbonate of soda. There was also often a deficiency
of the carbonate of ammonia which it is well known has equal
power to influence the coagulability of the blood and the integrity of
the red corpuscles.

In all instances it was found that the cholera blood chemically was
closely allied to putrescent blood, and readily made to undergo the
putrefactive ferment, far more easily than healthy blood.

The evacuations were all found to be rich in water, and in the alkalinity
of which the blood was deficient, particularly the tripple phosphates
and the carbonate of soda, while they contained but a trace of
albumen. Occasionally in the bladder would there be found a little of
the blue coloring matter mixed with chlorides and the earthy phosphates,
while under the microscope could be discerned in the sediment the tuff
cylinders and the epithelium which had been discharged from the lining
of Bellini's small urin-ducts.

The secretions from other parts of the body have not been as carefully
examined as they should be, but thus far have furnished only
negative results.

If now we consider the changes produced in cholera as here described
are not always uniform, or of an equally marked character, but that they
depend upon the force of different influences—that epidemic cholera not
unfrequently occurs with entire absence of vomiting or purging, but
with an extraordinary amount of \typo{perspiratory}{prespiratoy} exudation, or with spams
that speedily cause death—that in spasmodic cholera the anti-spasmodics
are generally found useful—that not a few cholera patients die from
want of what is called reaction, even where there was no appearance of
decomposition of the blood or deprivation of serum in the blood vessels,
we must come to the conclusion that the first impression of the cause of
cholera is sometimes made upon the blood and at other times upon the
nervous system, while in more rare instances it may impress both the
blood and the nerves at the same time.

The question as to why the serum or watery portion of the blood
\oldpage{391}should escape into the stomach and bowels to produce the rice-water
discharges may be answered by referring to the inevitable result of severe
congestion of the lymphatics, as is also shown in the pouring out the
serum upon the surface of the skin in the excessive perspiration which
is sometimes present.

The serum of the blood dissolves the epithelial cells of the alimentary
canal and these dissolved and partially dissolved cells are what gives to
the fluid its peculiar or ricy appearance.

\vspace{\baselineskip}

Is the cholera miasm, or \foreign{sui generis}, independent of the miasms which
produce other epidemic diseases?

Many physicians and natural philosophers have held that the cholera
miasm is but the product of the receding of some other form of disease
or rather a modification of a miasm which had produced some other
form of disease, and they have endeavored to sustain this position by
referring to the fact that an epidemic of cholera is usually preceded by
an epidemic of a different character. Others have considered that it
possesses an individual and independent character, unaltered by changes
and unaffected by climates, everywhere acting upon the alimentary canal
and on which, therefore, it must make its first impression.

Those who entertain this latter view consider the cholera miasm a
peculiar miasm, and call the cholera epidemic \emph{the epidemic of epidemics}
or the producer of epidemics, and the cholera miasm the miasm of
miasms, or the producer of miasms.

As has before been remarked, all miasms which produce epidemic
diseases have somewhat in common, but each also has something peculiar
or specific, and hence while the cholera has many characteristics
manifested in other epidemics, that it has an individuality of character
and an individuality of cause cannot well be denied.

The common characteristics which we observe in epidemics arise
from the fact that all miasms are of telluric and atmospheric origin,
and that all miasms in course of time have their power and influence
modified and changed. Yet they all nevertheless manifest essential
peculiarities of character and produce by a specific process each its own
individual disease. For instance, one miasm will produce scarlet fever,
another measles, and another cholera. If there is none of the specific
miasm there will be no measles, or no cholera, as the case may be.
Neither can one miasm produce another disease, for measles never produced
cholera, or cholera measles, or anything else but cholera.

This is the necessary result of the peculiar and specific character of
each individual miasm which possesses its own specific power and disposition.\endinput
\oldpage{392}``\emph{Quod libet miasma proprium generationes suae typum
in agendo sequitur.}''

In this as in every branch of the natural sciences, the conclusions
adopted may prove so clearly the hypothesis to be correct that it ceases
to be simply a hypothesis but may claim to be classed as an established
scientific truth. As in the natural sciences, so in medicine, the inquirer
after truth must at times adopt a hypothesis for the explanation of the
phenomena which he observes; and in this instance the explanation
which the hypothesis of a cholera miasm gives to the phenomena of
the disease comes near proving that to be the true origin of the
epidemic.

So also the later advances made in the science of chemistry have
nearly proved the cholera miasm to be a reality and not merely a
hypothesis. Dr. Horn of Munich, obtained from the atmosphere
\emph{Ozone}, or a negative electric body, and another body, \emph{Todsomone},
which has been found to combine in the body with carbon and by the
combination to produce effects upon the structures very similar to the
effects produced under similar circumstances by the cholera miasm. I
would not assert that these discoveries prove beyond cavil that the
cholera miasm is Todsomone, but this much is certain that we may feel
sure that observation will establish many practical truths by accepting
this hypothesis, and will also stamp upon it the seal of truth.

\vspace{\baselineskip}

Does the cholera miasm, as many suppose, make its direct impression
upon the stomach and intestines?

The circumstance that the first symptoms of cholera are vomiting and
purging, and other indications of derangement of the alimentary canal
goes to favor the idea that the mucus membrane of the prima vie is the
point at which the reception of the miasm first occurs and from which
it progresses farther into the organism. In opposition to this idea is
the fact that \emph{spasmodic} cholera, as was observed in thousands of cases
in the epidemic of 1831, frequently destroys the patient before vomiting
or purging presents itself; and also the processes of vomiting and
purging removes from the system a large amount of fluid which chemical
researches have proved to be changed blood serum, thus proving
conclusively that the vomiting and the purging are \emph{secondary}, and
sequela to the primary changes which had occurred in the fluids. So
also is shown that the cholera miasm must have impressed several parts
of the system and not alone the alimentary canal. The nervous system,
the blood, the lungs, and the ganglionic system all bear evidence of the\endinput
\oldpage{393}
presence of the cholera miasm, and that it must come in contact with
all these structures.

Neither may we loose sight of the fact that cholera has often been
produced by what is styled the \foreign{contagium psychicum}. It is a well-established
fact that many die through a fear of the disease, and particularly
through the influence of the sight of a cholera case upon an
impressible person. Many doubtless are thus led to suffer from the
epidemic who otherwise would have entirely escaped it.

With regard to which is first acted upon, the blood or the nervous
system, I think the true answer is that in this regard there is a great
diversity in the different cases, but that in many cases both the blood
and the nerves are simultaneously impressed.

\vspace{\baselineskip}

Has the cholera miasm and the Asiatic cholera undergone any alterations
in its original nature and character in its transit? Does it always
present forerunners of epidemics? Has it always also been followed by
other forms of epidemic disease as it has passed away?

The cholera miasm has certainly \emph{not} undergone any change but remains
ever the same in nature and quality as when it started from the
Punjaub as is shown by the unaltered and specific character of the epidemic
in all climes and seasons, without any regard to the state of the
weather, uninfluenced by heat or cold, or dryness or moisture. But no
one will deny that the cholera miasm and consequently the disease
which it produces does loose from time to time apart of its potency and
assume a more mild and manageable form, for the history of its various
epidemics has fully established these facts. But the succeeding epidemic
is found to equal in intensity any former, and the one of the year
1855 in the month of May, was more intense and destructive than any
which had preceded it.

In most instances preceding an epidemic of cholera, other epidemics
have been observed as preceding this, as intermittents, diarrhœa, dysentery.
So also after the epidemic of cholera has passed by, have
epidemic forms of disease appeared, as typhus, influenza, etc. These
observations have led to the opinion that the preceding diseases might
be considered as the forebodings of cholera, and the succeeding as the
sequelae of the disease.

I have already pointed out the common sources from which all
miasms arise and hence the connections of these various forms of epidemic
diseases can be explained without our concluding there is anything
more in common with these miasms than simply a relationship
of origin.

[to be continued.]\endinput
\section*{Observations on the Uses of Sanguinaria
Canadensis.}

by \textsc{Abr'm.\ Livezey, a.\ m., m.\ d.}

\oldpage{394}
In several medical journals I have taken the liberty to call the attention
of the profession to some of the uses of our indigenous medicinal
plants, and in the present communication I beg leave to offer some
remarks upon the medicinal value of the Sanguinaria---a plant incident
to all localities and the root of which is easily gathered.

Without prejudice to the use of any other article I feel warranted in
saying, from no little experience, that this plant with the aid of podophyllin
will exert a more happy influence in all hepatic derangements---both
as a cholagogue purgative and as an alterative---than any combination
of calomel.

Possessing undoubted nauseant, sedative and alterative properties,
blood-root will in cases of slight inflammation of the biliary organs, or
congestive states of the same, or where a species of spasmodic action
pervades those structures, give prompt relief; and where torpidity
exists and the physician thinks that the stimulant action of some mercurial
is indicated he need only combine a minute portion of the
podophyllin to obtain all the advantages that are supposed to be
derived from calomel.

Sanguinaria gives a decided aid to the action of podophyllin or any
other cathartic to which it is added.   It is, in the form of tincture, an
alterative expectorant in chronic bronchitis.   It is valuable in chronic
hepatitis combined with ext.\ taraxicum and ext.\ podophyllum, jalap or
rhei, if obstinate constipation exists.    Tinct.\ Sanguinaria can with
advantage be substituted for wine of antimony in the \emph{brown mixture}
and wherever the wine of antimony is used.   As a substitute for the
compound cathartic pill the following combination---already published
will generally prove more satisfactory:

\begin{center}
\begin{tabbing}
  \textsf{℞}. \= Podophyllin, \= gr. \= i., \\
    \> Leptandrin, \> ''\> iv., \\
    \> Sanguinaria, \> ''\> ii., \\
    \> Ext. Taraxicum, q.\ s.\ Misce.\ ft.\ pil.\ No.\ iv. \\
\end{tabbing}
\end{center}
Two or three for a cathartic; ½ to a whole one night and morning as a
hepatic alterative.

A graduate student of mine, Dr.\ Rice, late resident physician in the
W.\ C.\ Infirmary of Philadelphia, had a case of obstinate constipation
which had persisted four weeks---so said the patient, an Irish woman,\endinput
\oldpage{395}
when she presented herself at the clinic---and in twelve hours time
she had a free alvine evacuation from the use of Sanguinaria, well triturated
with white sugar and given in small doses every two hours.
Dr.\ R.\ is fully persuaded that blood-root is an admirable adjuvant in
all prescriptions for the restoration of healthful function in the liver,
and especially when constipation is coincident.

\textsc{Note}.---Perhaps no indigenous plant has attracted more attention
from those physicians who are accustomed to notice the living specimens
of materia medica as they spring up in the woods and fields
than the one under consideration. The early appearance of its beautiful
and pure blossom, the dark blood-color of its fleshy root, its marked
taste and its prompt action on the system all lead to its obtaining the
attention which has been bestowed upon it.

A trial of its therapeutic virtues has led those who have made use of
it to speak of it in the highest terms of praise and to earnestly recommend
it to the favorable notice of the profession, and yet, strangely, it
has never obtained that prominent position in the list of medicines all
its advocates think it deserves.

Nearly every writer on Botany and Materia Medica in our country
has delighted to give a full description of this plant and to speak highly
in praise of its beauty and usefulness. Among the earlier writers who
have made mention of it Dr.\ Shoepf says that fifteen or twenty grains
of the pulverized root will produce powerful emesis, but that it must not be
given in the form of a powder as thus it is apt to produce great irritation
of the fauces. He prefers a decoction or the pill form. Merat says it
is useful in gonorrhœa. Shoepf also mentioned the value of a weak
decoction of the root in gonorrhœa and refers to the fact that Golden
had found it useful in jaundice. In doses sufficient to produce emesis
it was found to dislodge worms from the stomach. Thatcher, in his
Dispensatory, speaks of the use made of it by Dr.\ Dexter in doses of
one grain of the powder or ten drops of the saturated tincture, as a
stimulant and diaphoretic. Dr.\ Downy was of the opinion that the
dose as recommended by Drs.\ Shoepf and Colden was larger than could
be administered with safety. In speaking of the value of the root in
jaundice Dr.\ Thatcher says it was believed to be the chief ingredient
of the quack medicine known as \emph{Rawson's Bitters}.

The younger Barton thinks that the only form in which the blood-root
should be used is that of a spirituous tincture. In this form he
used it in connection with the tincture of bitter-plants as a tonic with
great satisfaction. He also found it useful as a wash for old indolent
ulcers and sores with hardened edges and an ichorous discharge. He\endinput
\oldpage{396}
had also used the powdered root as an application to fungoid growths
and nasal polypi. Bigelow, and Dr.\ Smith also used it for the same
purpose. So also Dr.\ Shanks and Dr.\ Israel Sterling, according to
Thatcher, used it in place of digitalis in coughs and pneumonic complaints.
Dr.\ Darwin has used it in peripneumonia trachealis in the
form of a decoction and from the benefit thence derived Dr.\ Barton
thought it must be a useful medicine, particularly in cynanche maligna,
in cynanche trachealis and other similar affections.

Drs.\ Barton and Downy said that the \emph{leaves} of the puccoon as well as
the seeds are possessed of a \emph{narcotic} power similar to that of the seeds
of the stramonium and that they had produced dangerous symptoms.

In 1831 Daniel B.\ Smith published in the \booktitle{Journal of the Philadelphia
College of Pharmacy} a dissertation on this plant, in which,
he gives its natural and botanical history and speaks of the experiments
made by Dr.\ Dana on the root in 1824, when the \emph{Sanguinarina} was
probably first obtained.

Dr.\ Tully has carefully examined the medicinal powers of blood-root
and thinks it is therapeutically allied to squills, seneca, digitalis, guaiacum
and ammoniacum.

More recently Dr.\ Williams, formerly of Massachusetts but now of
Illinois, has written several valuable essays on the Sanguinaria, but
unfortunately I have lost the reference to them and I only remember
that he considered it one of the most valuable if not the most valuable
of all the North American plants.

Dr.\ Thom of Ohio, in a communication to the \booktitle{Western Journal},
says that for two years he had been closely engaged in observing the
effects of this remedy in various diseases and he concludes that it is a
\emph{sedative} of no ordinary powers. For reducing the force and frequency
of the pulse without prostrating the system he considered it one of the
most efficient remedies. He also styled it an \emph{alterative} with a marked
influence on the liver and the glandular system generally. He employed
it in hemorrhage from the lungs, particularly in those cases where
the hemorrhage appeared to be caused by vicarious menstruation, and
considered it of more value than any other agent he had used.

Dr.\ M'Bride in the \booktitle{South.\ Jour.\ of Med.} said he considered this
plant eminently serviceable in those disorders of the liver where the
secretion of the bile is either suppressed, deficient or vitiated. In imperfect
convalescence after bilious fever he says, ``the puccoon is the
best remedy.'' As an emmenagogue he thought highly of it. He
recommended it as a substitute for mercury.

Dr.\ J.\ L.\ Mothershead used it in dyspepsia in the form of pills,\endinput

\oldpage{397}giving from one to three grains at a dose three times a day. In
troublesome cough he found it valuable. He also used it satisfactorily
in tinea capitis, tetter and other forms of skin disease in the form of
powder or strong tincture on the affected part. He said: ``Of all
the articles in the Materia Medica, next to mercury and its preparations,
none in my opinion can compare with it in its powers to excite the
action of the liver, and it has the advantage of the former in its capability
of being used at all times and continued without producing any
of its unpleasant results.''

Dr.\ Bard in his Inaugural Dissertation confirmed the statement of
Dr.\ Downey in regard to the narcotic effect of the seeds and speaks
of using the root in croup, pneumonia, whooping-cough, phthisis and
jaundice. Dr.\ J.\ Allen of New York, says it powerfully promotes diaphoresis
in inflammatory rheumatism. Dr.\ Downy says that the leaves
are used in veterinary practice in Maryland for the purpose of facilitating
the shedding of the hair of animals. Dr.\ Griffeths has also given it to
horses for the cure of bots, one or two roots serving to produce a cure.

Dr.\ Branch, of South Carolina, thinks a decoction of the root of
more value than any other single remedy in croup. He denies that it
is possessed of any poisonous properties.

I have not been able to obtain the Inaugural Dissertation of Dr.\ Henry
West, of Belmont Co., Ohio, upon the use and value of this
agent, but evidently he must have placed a high value upon it to make
it the subject of his remarks.

Recently Dr.\ J.\ W.\ Fell has been permitted to make a trial of his
mode of treating cancer on the patients of the Middlesex Hospital of London
and as he had not previously made known the agents he had used
the \booktitle{London Lancet} condemned the secrecy which had governed him, and
finally Dr.\ Fell was led to publish a work on Cancer and its Treatment
in which he said he had used the ``bruised bloody pulp of the white-flowering
puccoon.''

The formula used by Dr.\ Fell differs from the chloride of zinc
paste of Dr.\ Papengurth and Prof. Hancke of Breslau and Dr.\ Canquoine
of Paris, from the addition of the blood-root to the ingredients used by
these surgeons in the treatment of cancer.   The formula is as follows:

\begin{center}
\begin{tabbing}
  \prescription. \= Sanguinaria Canad., \ounce ss, vel ounce j., \\
    \> Zinci Chlorid., \ounce ss, vel \ounce ij., \\
    \> Aqua, f \ounce ij., \\
    \> Tritic. Hybern. Sem. pulv., q. s.
\end{tabbing}
\end{center}
M.\ f.\ paste as thick as treacle and apply to the cancer,

For years this has been a popular remedy for the purpose of destroying\endinput

\oldpage{398}granulations and other morbid growths, and it is more than probable
the blood-root which has been added to various ointments and
applications which have been used upon cancerous affections has done
much toward effecting the cure.

A reference to the use of blood-root in the cure of cancer is now
causing considerable discussion in various sections of the country, particularly
in New England, and where much is being said in regard to
who was the physician who first used it for the cure of cancer. My
own opinion is that it was in use by the people and the \emph{country} physicians
long before we have any record of its being thus applied.

From the very imperfect abstract here given of a few of the articles
that have been published in our periodicals on the use of this root, we are
warranted in drawing the conclusion that it is a very valuable medicine
and should be introduced into more general use. But doubtless one
reason for its neglect is the fact that the root rapidly looses its value by
age and if kept more than one year may become nearly worthless.

The tincture and other preparations should be made from the root as
soon as possible after it is gathered and not from the old and nearly
worthless specimens usually sold by druggists.

In regard to the preparations sold under the names of \emph{Sanguinarin}
and \emph{Sanguinarina}, although I have had frequent letters of inquiry
addressed to me, I cannot give any satisfactory answer. I have no
means of knowing what these preparations are or how manufactured,
and of those who have used them I have never been able to obtain any
evidence of their character or value as therapeutic agents, but a friend
of mine who has manufactured these articles and sold them in considerable
quantities has told me, that as the result of his own observations
and the observations of those of the profession who, had bought and
used them, he was fully convinced they were of even less value than
the pulverized root. I consider it a duty I owe to the readers of the
\textsc{Journal} to present these facts.

If those who manufacture these agents would let us know enough
about them to warrant us in making a trial of them, and if those who
have used them would carefully observe their action and notify us of
the result, soon the readers of the \textsc{Journal} would be in the possession
of the required information. In the present state of the case the only
answer I can give is that I have never used them and know nothing
positive about them. \hfill{}C.\quad\endinput

\section*{Extractum Nicotianiæ Rademacheri.}

by \textsc{Theodore C.\ Miller,\ m.\ d.}

\oldpage{399}\textsc{I herewith} present a notice of an agent which to me is possessed of
extreme value. It is the \emph{extract of the Nicotiania rustica}, as prepared
by the late Dr.\ T.\ G.\ Rademacher.

It is not prepared from the dry but from the fresh and green tobacco
plant. In preparing the extract it is necessary that \emph{immediately} and
without delay, after the leaves have been pulled they must be pressed
so as to force out the juice and that juice evaporated to the consistency
of an extract. When prepared in this manner the extract has none of
the taste of the dried tobacco leaves; but if the leaves are pulled only
a few hours before the juice is expressed, then the extract will have a
taste more or less like that of smoking tobacco, in which case it is not
fit for therapeutical purposes. I have ahvays found it best to have the
leaves pressed at once on being pulled; and I have always prepared it
according to the directions of Rademacher, from the Nicotiania rustica,
and not from the Nicotiania tabacum. Rademacher's extract is one of
the best remedies in genuine cough of the lungs, and for that I can wdth
a clear conscience recommend it to the readers of the \textsc{College Journal}.
It is a remedy for which probably I could not find a substitute.

Rademacher gave it in doses of from one half to two grains, and repeated
it several times a day. It may be made into a pill with the
powdered marsh mallow root. It is a quick and safe remedy in a particular
diseased condition of the lungs for which I am not able to give
a name, but the want of the \emph{name} is no loss to the practical physician,
who must be governed by the nature of the disease and not by its
nomenclature.

That we can, with the extract of the fresh leaves of tobacco, cure an
inveterate genuine lung cough, and thus prevent pulmonary tuberculosis,
in my mind does not admit of a doubt, provided the cough is kept under
the control of the remedy. But there are forms of lung cough which
this extract will not control, and in those cases I would recommend a
trial of the \emph{Stibium Sulphuretum Auranticum}, as mentioned in the
\textsc{College Journal} for June, page 351. Rademacher truly says: ``In
general we must be guided in our minds in the practice of our art, by
the following fact: The diseases are not governed or changed in character
by the ideas and opinions of the physician, but the opinion of
the physician must be governed by the nature of the disease.''

That the extract of tobacco has a powerful controlling influence over\endinput

\oldpage{400}the genuine lung cough, serves as a diagnostic as to the real nature of
the disease. If the cough originates from the lungs it will be benefitted
by the extract, while if it owes its origin to a disease of some
other part of the system, the extract may fail of benefitting the
patient. But there may be coughs which in reality are caused by some
diseases of the lungs, and yet the extract may not prove beneficial.
For instance, a cough may be caused by a node, or from a closed or
an open abscess in the lungs, or from the pressure of a fractured rib
upon the pulmonary tissue and yet the extract would not produce a
cure. The extract has a favorable influence upon idiopathic but not
on secondary coughs. With opium we can often relieve secondary or
sympathetic coughs. We do not with that agent obtain a cure, but
we do obtain relief from the cough, and moderate it or pacify it. The
extract of tobacco is not as active as opium to \emph{allay} a cough, but far
more powerful to cure it when of the genuine lung origin.

\textsc{Idiopathic bleeding of the lungs.} When I speak of bleeding of the
lungs I mean to be understood that form of the disease which is commonly
called \emph{spitting of blood}, where a greater or less quantity of clear
blood, or blood mixed with phlegm, or phlegm streaked with blood,
will be raised from the lungs. The extract is valuable in these cases,
but may not be depended upon in \emph{Pneumorrhagia, or Apoplexia pulmonalis},
in which latter form of disease we must resort to the use of
allum and ice internally and cold wet cloths to the surface of the
chest, and to other appropriate remedial measures.

I would here remark that this preparation will not produce the vomiting
and purging which follows the administration of the dry tobacco,
and I have never used the dry tobacco as an emetic or an injection, as I
find the Lobelia inflata an equally efficient remedy.

I was called a few days since to see a patient where many other remedies
had been tried by three eminent physicians who had attended on
the case, without avail. The patient had been sick quite a length of
time but owing to my recent illness and the distance from me I could
not treat it. The case presented the characteristics of consumption, a
harrassing cough, with bloody sputa, etc. As I was unable to visit the
patient I was consulted by letter, and had ordered inhalations, and directed
the Wild Cherry, Lycopus Virginicus, and Lobelia combined
with Ipecacuanha, without benefit. I used the Lobelia, from having
found it of great value in cramps and affections of the chest, and
particularly in phthisis pulmonalis. For these purposes, and to relieve
the dry harrassing cough and tickling of the throat, it is in use
by many German physicians.\endinput

<span class=font1>1857.]&nbsp;<i>Extraclum Niootianiœ Rademacheri.&nbsp;</i>* 401</span>

<span class=font1>** As I was at the time out of the extract of tobacco I made a trial of
the Lobelia, but I obtained some from my brother and about two
weeks since I commenced its use. In six days the cough and the expec-
toration entirely ceased. I have since visited the patient and although
the symptoms are so much relieved, auscultation does not promise much
for the final recovery &lt;?f the patient. Too many persons had prescribed,
and the lungs are too much diseased to allow much hopes of a permanent
cure; but this case illustrates the power of the agent.</span>

<span class=font1>I am the more urgent to induce the profession to make a trial of
this extract, as I think it is nearly or quite unknown to the physicians
in this country.</span>

<span class=font0><b>aqua  nicotians tabacum  sperituos^] radamacheei.</b></span>

<span class=font1>This preparation is recommended highly in affections of the brain
accompanying fever, in <i>rheumatismus acutus fixus at vagus, </i>in other
affections of the brain and spinal marrow, in cholera morbus, and in
cholera Asiatica.</span>

<span class=font1>To prepare it: Take of choice fresh green leaves of Nicotianse ta-
bacum eight pounds, and cut them finely. Add of the best alcohol,
by weight one and a half pounds, of distilled water as much as is
necessary to distill over eight pounds (by weight) of the water.</span>

<span class=font1>The leaves are to be cut and the distillation effected immediately
after they are pulled, with great care that there shall be no over-heating
of the liquid, as, if the liquor be over heated it will have a very dis-
agreeable odor of tobacco, which it does not have when the water is
properly prepared.</span>

<span class=font1>Eademacher uses this water in every stage of the Asiatic cholera.!
In the earlier stages he gave the following:</span>

<span class=font1>Aqua Purse, f 3 vij.,</span>

<span class=font1>Soda Acet., 3 jss.,</span>

<span class=font1>Aqua Nicotian., f 3j.,</span>

<span class=font1>Gumi Arab., 5ss.
M.   Dose, one table-spoonful every hour.</span>

<span class=font1>The great majority of cases treated with this mixture recovered
immediately from the attack. In those cases where the attack was fol-
lowed with a typhoid condition, he gave:</span>

<span class=font1>Tinct. Ferri Acetici, f 3 j., ''.[</span>

<span class=font1>Aqua Nicotian., f 3j.,</span>

<span class=font1>Aqua Puree, f 3vj.,</span>

<span class=font1>Gumi Arabici, 3.
<b>M.   </b>Dose, one tea-spoonful every hour.</span>

<span class=font1><b>vol</b>. ii. <b>no. </b>9.-26.&nbsp;v&nbsp;- &quot;<sup>J</sup></span>\endinput

\oldpage{402}

With this treatment the patients all recovered after a longer or shorter
period.

\textsc{Note.---}The formula for preparing the acetic tincture of iron is to be
found on page 351 of the \textsc{College Journal}.

\chapter[Pleuritis, Latant][Pleuritis, Latant]{Pleuritis, Latant.}

by \textsc{C.\ E.\ Witham,\ m.\ d.}

\textsc{Pleuritis} is a disease which often presents obscure, important and interesting
complications, taxing the utmost skill of the experienced physician
in tracing the precise bearing and extent of the morbid action
established.

The heart, lungs, bronchia and liver are often implicated in this disease.
Asthenic pneumonia, and chronic and latent pleuritis have many common
symptoms. It is stated that pleuritis is more prone to produce
tubercular disease than pneumonia is, and it is thought by some authors
that the absorption of pus into the blood may explain this rather singular
fact. In the treatment of disease our object should be to remove
morbid action by the most simple and effectual treatment the case will
admit of. If the following report should be the means of stimulating
the young practitioner to a more thorough study of thoracic diseases I
shall be amply rewarded.

On the 11th of June, 1856, F.\ W., a lad 14 years of age was presented
for my advice. He was of a sanguine temperament, and a twin brother.
I had never seen him before, but from his father gained the following
history of his case. Five months previous to calling upon me he suddenly
lost the power of speech; did not know that he had previously
suffered from cold or exposure. The loss of speech was the first symptom
of disease he could recollect and this was not preceded by any
very marked indications of hoarseness. A low, hoarse and painful
whisper was the result of all his efforts at conversation. This condition
continued for one month when to his surprise and great joy he found
himself complete master of his vocal organs and congratulated himself
on so strange and unexpected a recovery. He said that while making
some slight exertion he felt something give away in his chest and immediately
he could talk as well as ever. At the end of one week he
was again deprived of speech in the same unexpected and sudden manner.
His physician after inspecting his throat, but making no other
examination, prescribed a gargle of ``pepper tea'' saying it would soon\endinput

\oldpage{403}effect a cure; but after a trial of several weeks this prescription was
discarded, as no change had resulted. Lancinating pain would occasionally
be felt in the chest, slight cough, expectoration streaked slightly
with blood. He continued to perform light work and had not been
confined to his bed. I found him presenting the following symptoms
five months after the first appearance of the disease.

The mucous membrane of the pharynx presented a pale and debilitated
appearance; the chest inclined forward, the body assuming a
stooping position; great tenderness of the spine from the first cervical
vertebra to the last dorsal; pressure over the lungs, liver, stomach
and spleen gave pain. In short no part of the chest nor abdomen
could be percussed without revealing deep-seated tenderness. The
skin was dry, pulse quick; there was much dyspnœa with abdominal
respiration. Percussion of the lungs gave rather a dull sound. Bowels
torpid. I diagnosed the disease to be Latent Pleuritis complicated
with chronic inflammation of the larynx which gave rise to the Aphonia.
As the patient was of a strumous diathesis and the disease of long standing
I doubted the efficacy of treatment but advised it and took charge
of the case on the 12th of June.

I first ordered morning bathing to be practiced daily, the water used
to be impregnated with chloride of sodium and bicarbonate of potassa.
Internal treatment:

\begin{center}
\begin{tabbing}
  \prescription. \= Podophyllin, \dram{}ss., \\
    \> Capsicum, gr. X., \\
    \> Ext. Taraxicum, q s.
\end{tabbing}
\end{center}
M.\ f.\ Pill, No. X.\quad{}Take one of these pills morning, noon and night
until the bowels are freely moved, then take but two a day.
\begin{center}
\begin{tabbing}
  \prescription. \= Comp.\ Syr.\ Stillingia, f\ounce{} iij., \\
    \> Capsicum, gr. X., \\
    \> Iodide of Potassium, \ounce{} j.
\end{tabbing}
\end{center}
M.\quad{}Take one teaspoonful four times a day. To test the progress of
the case I saw the patient daily. I discovered no change until the third
day; the bowels were then active, less tenderness about the cervical
vertebra, could whisper with less pain and more distinctly. On the
fourth day still more improved. I ordered the same treatment continued
and on the next day the patient recovered full power of speech and could
talk freely and without pain. He continued to improve and on the tenth
day of treatment I could discover no abnormal symptoms. Percussion
over the abdominal and thoracic viscera was no longer painful; no tenderness
of the spine could be detected. I now discontinued the former
treatment excepting one pill to be taken each day, and prescribed the
following:\endinput

\oldpage{404}

\begin{center}
\begin{tabbing}
  \prescription. \= Prussiate of Iron,  \\
    \> Hydrastin, āā \dram{} ss.
\end{tabbing}
\end{center}
Mix.\quad{}Make 15 powders, two to be taken a day. The patient felt
well and returned home, and now nearly one year has remained well as
usual.


\section*{What Influence has the Moon Upon Disease?}

by \textsc{Comely Jessup,\ m.\ d.}

\textsc{I wish} to ask your opinion and procure, if possible, the result of the
observations of your readers, relating to the influence (if such influence
exist), exerted by the moon upon disease. I have been of the number
who look upon the lunar influences except such as may be attributed to
the known laws of gravitation as entirely fabulous, but several instances
occurring within the sphere of my observation, which have indicated
the existence of some hidden agency, a few of which have been distinctly
marked, have awakened a desire to see the matter thoroughly
investigated and the truth or falsity of lunar influence fairly demonstrated.
The following are a few of the more marked instances of apparent lunar
periodicity which have fallen under my observation.

\textsc{Case 1.} Mr.\ C., aged perhaps 45, has been subject to epilepsy for
the last three years. About the time of the change and full of the
moon he will have from three or four to eight or ten convulsions. At
other times he is free from them, except occasionally about the time of
the first and last quarter.

\textsc{Case 2.} T.\ I., aged 72, was attacked some four years since with malignant
erysipelas, accompanied at first with paralytic symptoms,
which, together with a severe attack of ``Doctors,''---though he survived
them all---left him in a condition from which he has never recovered
and never will. The most prominent features in his case now
are pain in the back and head, which is remittent in its character,
being most severe in the early part of the day; nervousness, constant
trembling of the hands, or rather the peculiar shaking characteristic of
paralysis, to the extent that he can with difficulty feed himself; and
occasional attacks of general weakness and disposition to syncope.
These \typo{symptoms}{symptems} are all much aggravated at the time of the moon's
changes.

\textsc{Case 3}. A.\ E.\ S., aged 5; troubled with ascaris vermicularis at
the time of the new and full moon, which were during the intervening\endinput

\oldpage{405}space of time quiescent. This case would not have excited suspicion,
inasmuch as there seems to be frequently a periodicity in their
actions, but taken in connection with other cases it is a straw which
indicates the quarter from which the wind blows.

Now the question is, does the moon during its various phases exert
various influences which though unperceived by the robust constitution
of perfect health, make themselves felt to the sensitive system
of the invalid, or are these merely striking coincidences? These are
questions of interest to the Physiologist and medical Philosopher,
merely as significant facts, but doubly so to the practitioner to whom
a knowledge of every influence brought to bear upon those under his
charge is essential.

With a hope that others may be induced to make known the result
of their observations, I report these cases.

\chapter[Human Blood a Styptic][Human Blood a Styptic]{Human Blood a Styptic(?)}

by \textsc{O.\ Van Buskirk,\ m.\ d.}

\textsc{I wish} to communicate a few thoughts upon a case which came under
my observation a short time ago, in which I employed human blood as
a styptic with the most gratifying result. To you this may be no new
thing, but to me it is, and it may be to many other junior members of
the profession. From this consideration I thought I would write you
a brief account of the case and the manner in which I employed it.

The case was a lady from whom I extracted a tooth (the first molar),
and it was rather difficult to draw, but it came out whole and without
doing any perceptible damage to the jaw. The hemorrhage was not
very profuse at the time; not more than usual. When she left my office
she seemed as well as usual and continued so for two days, at which
time a profuse hemorrhage took place from the cavity in her jaw. By
means of a decoction of black-oak bark she checked it for about twenty-four
hours when it began again worse than before. I was then sent for
and found her quite weak and sick at her stomach. I applied geranin,
tannic acid, etc., all to no effect. I then took about two ounces of blood,
placed it over the fire and as soon as it came to the boiling point the
solid constituents of the blood coagulated, and left the aqueous portion
clear and limpid. I then poured off the water and left the other over
a slow fire until it assumed a thick, jelly-like form. I took a small lump
of this and filled the cavity and placed over it a small wad of cotton
wadding and directed her to close her jaws so as to keep the remedy in\endinput

<span class=font3><b>40G</b></span>

<span class=font3><i>Notes and Observations, by </i><b>D</b><span class=font2><b>r. </b></span><b>M</b><span class=font2><b>iller</b></span>. [September,</span>

<span class=font3>its place, and to my great delight it stopped the hemorrhage almost
instantly. The remainder of the blood which was in the vessel I took
through the same process and reduced it to a fine powder for future use.</span>

<span class=font3><b>R</b><span class=font2><b>emarks.</b></span><span class=font0>—</span>The use of blood as a remedial agent is not new to the
profession as a reference to the <b>J</b><span class=font2><b>ournal</b></span>, page 299 will show, where
Dr. Miller refers to the use made of it by Mauthner in ansemia, as
referred to by Dr. Davis, of Illinois, in the Transactions of the Illinois
State Medical Society for 1852, who says it may be given when inspis-
sated in doses of from ten to sixty grains at a dose, or dissolved in
water. The therapeutic use of blood is also referred to in the <i>Am.
Med. Journal </i>for 1853 and perhaps in other periodicals. But none of
these refer to the <i>styptic </i>properties it is supposed to possess by Dr.
Yan Buskirk, and which one single experiment neither proves or dis-
proves. The Pencil Savans are accustomed to deny nothing until it is
thoroughly disproved, and to admit nothing until it is thoroughly estab-
lished, but rather to <i>receive </i>the opinions of others and await farther
and full proof before their final disposition, and in this regard we may
do well to follow their example. <b>C.</b></span>

<span class=font3>NOTES AND OBSERVATIONS.</span>

<span class=font2><b>by t. c. miller, m. d.</b></span>

<span class=font3><b>A</b><span class=font2><b>mmonia </b></span><b>Y</b><span class=font2><b>alerianicum [ </b></span><i>Valerianate of Ammonia']. </i>This is formed
by the saturation of the Valerianic acid with the carbonate of ammonia.
It is usually a fluid, although an imperfect crystallization has been ob-
tained of the salt. In warm weather even the crystals are apt to de-
liquesce into a syrupy fluid, having a strong valerianic odor, and a
slight odor of ammonia.</span>

<span class=font3><b>O</b><span class=font2><b>ttinger </b></span>of Munich, Germany, has highly recommended this prepar-
ation in Asiatic Cholera in the following form.</span>

<span class=font1><i>V&lt;.   </i><span class=font3>Ammonia Valerianici, </span><i>B </i><span class=font3>j.,
Aqua Destillat., f'Siij.,
Syrup. Sacch., f&quot;3 ss.</span></span>

<span class=font3>M.   Dose, one tablespoonful once in from 15 to 30 minutes.</span>

<span class=font3><b>O</b><span class=font2><b>ttinger </b></span>used this mixture to the exclusion of every other internal
remedy, and after the severity of the attack had passed and reaction
was established, he gave but from four to six doses daily. He also
ordered ice to be rubbed over the abdomen externally, occasionally
changing the cold w<sup>r</sup>ater for hot or placing the patient in a hot bath in</span>

<span class=font2><b>1857.]</b></span>

<span class=font2><i>Notes and Observations, by </i><b>D</b><span class=font1><b>r. </b></span><b>M</b><span class=font1><b>iller.</b></span></span>

<span class=font2><b>407</b></span>

<span class=font2>which had been dissolved from one ounce to one ounce and a half of
caustic potash. Many other physicians have tried these remedial
measures for the treatment of Asiatic cholera and have spoken highly
of it.</span>

<span class=font2><b>M</b><span class=font1><b>ichael </b></span><b>Y</b><span class=font1><b>on </b></span><b>Y</b><span class=font1><b>isanik </b></span>speaks in regard to this mode of treatment
in the following manner : &quot; A very favorable result has been procured
by the Yalerianate of Ammonia,by adding a scruple to three ounces of
distilled w<sup>r</sup>ater. We tried it in <i>sixteen </i>selected dangerous cases, of
wdiich one third showed already symptoms of asphyxia. We gave in
the beginning one tablespoonful every quarter of an hour, but aftei
the system became affected with the remedy w<sup>T</sup>e gave the medicine
each half hour or every hour. After twelve or fifteen hours, and the
use of a few doses of the medicine, the pulse which before could not
be felt would appear again, the discharges and the cramps would cease,
and the skin would acquire its natural color, elasticity and feeling,
become moderately moist, and the patient present a tendency to sleep.</span>

<span class=font2>As soon as the use of the remedy had produced a turgescence of the
face, and symptoms of congestion of the brain were presented, we
ceased to longer use the valerianate, and by lifting the head up and
applying cold water these symptoms were checked. In this w<sup>T</sup>ay, of
the sixteen patients we saved <i>ten, </i>and <i>five </i>died. One case could not
be made to take the medicine. Others have complained that their
patients could not be made to retain the medicine, but our experience
convinces us that the remedy is one deserving every attention, and
should be recommended for further trials.</span>

<span class=font2><b>B</b><span class=font1><b>leeding in </b></span><b>P</b><span class=font1><b>regnancy.</b></span><span class=font0>—</span>The celebrated author and practitioner,
Dr. K. G. Neumann, expresses himself in regard to bleeding in preg-
nancy in the following manner:</span>

<span class=font2>&quot; While pregnant women do not continue to menstruate, some old
women of either sex in and out of the profession have imagined that
impurities must accumulate in the system unless <i>bleeding </i>is resorted to
to furnish the desired outlet. These imaginings are certainly foolish,
and while we can excuse women for entertaining them, we certainly
cannot excuse physicians for entertaining and perpetuating this folly.</span>

<span class=font2>Such physicians should, as often as they bleed pregnant women,
have several pounds of their own blood drawn off, so that they should
soon die for the benefit of humanity. The blood which is supplied by
the vital forces is required for the formation and perfection of the fœtus,
and it is a <i>crime </i>to waste it, and thus rob the unborn innocent of its
most precious patrimony.&quot;</span>

\oldpage{408}

\textsc{Typhoid Fever}.---A late German writer, Dr.\ F.\ C.\ Miller, says in
regard to the treatment of Typhoid Fever:

``The general recommendation of starvation and depletion by the
French School will never lead us Germans to adopt their extreme notions
however high they are lauded. We shall never as practitioners of
medicine, become Red Republicans on the sufferings of humanity.

It is true with us there are some who pass encomiums on Calomel as an
abortive remedy in this form of fever, yet others who view the disease
and that agent from a \textsc{Rational} point of view know mercury at the
best is a deceitful agent, and sometimes proves very disadvantageous 
and greatly destructive. \typo{should be leader}{* * * *}

The typhoid process sometimes proceeds rapidly and with great
severity, but mainly through dietetical and therapeutical errors. The
obstruction of the bowels which is often present in the commencement
of the disease, may by the injudicious use of laxatives or cathartics
be changed to a severe or unmanageable diarrhœa. Many physicians
have, by ordering purgatives, made it necessary to also order a coffin.

The treatment for Typhoid Fever adopted by the physicians of
Vienna seldom embraces either Emetics, Cathartics, or Venesection.
They give an infusion of a few grains of Ipecacuanha in the congestive
or nervous stage, together with Chlorine water, and a little dilute
Aromatic Sulphuric Acid in the evening, and where there are severe
exascerbations, with any appearance of intermittance they give about
\emph{four} grains of Sulph. Quinia daily. For the diarrhœa they prescribe
Alum. When the skin remains \typo{persistently}{presistently} hot, arid there is great
prostration, particularly if there is profuse diarrhœa they give camphor
and alum by the mouth and in injections.

If the disquiet and restlessness continues, but is more marked during
the night, and especially if there be a bloody diarrhœa, they give
Musk and Camphor in injections. They also apply as indicated, cold
water to the head and cold washings to the whole body, made of equal
parts of water and vinegar, so long as the skin remains hot and dry.
For the purpose of hastening convalescence they lay great stress upon
tepid bathing. This course I consider progressive and Rational treatment.''

\textsc{Phlegmasia Dolens}.---This disease is known by the names of White
Leg, Swelled Leg, Milk Leg, White Swelling of Lying-in Women,
Phlegmasia dolens alba. \foreign{Obstructus venerarum puerperalis}.

We find this a disease of lying-in women, and it commences sooner
or later with febrile excitement, and a painful, bright white, strained,\endinput

\oldpage{409}œdematous, acute, almost sudden swelling of the leg, and the one half
of the external genitals and the glands in the groins. It does not
affect both legs at the same time.

Many consider the swelling to be the result of a metastasis of the
milk, and hence the popular name; others think it a disease of the
lymphatics originally, and others think the veins to be the original
seat of the disease.

But the majority now hold it has its origin through and in consequence
of the inflammation of the crural vein, and resulting in the
obliteration of the same. I agree with \textsc{Lebert} that it is caused by a
checking of the venous circulation from an obstruction in the veins.
We find the disease mostly in those women who have suffered great
loss of blood by venesection or hemorrhage and in those who take cold
during the confinement.

The prognosis has always been very favorable under the treatment
which I adopt.

I always enjoin quiet, and wrap up the affected leg in roasted meal
and afterwards in oil-cloth. I give only the mildest salts, as Bochelle
salts for the purpose of evacuating the bowels, or use injections for
the same purpose, and give effervescing powders. As soon as the
febrile excitement has passed or is diminished I allow an easily digested
and nourishing diet, with Tonics, and the sub-carbonate of Iron. As
soon as the swelling becomes œdematous I consider the disease on the
decline.

\textsc{Dr.~Schrimer}, from his experience, as well as many others, was led
to the opinion that a careful tonic and sustaining treatment is the best
in Phlegmasia dolens alba and soon results in a cure. He states that
in an obstinate and severe case which followed a severe, tedious labor,
with extensive loss of blood, that, under what is styled the \emph{antiphlogistic},
or depleting treatment, the disease grew continuously worse, but
so soon as mild fomentation was applied, and a solution of Iodide of
Potassium and iron, given internally, the cure was speedily effected.

[\textsc{Note}. A somewhat different opinion, and the reasons therefore as
regards the nature and origin of this disease, may be found in Prof.~King's
\booktitle{American Obstetrics}.\hfill{}C.]

\textsc{Summer Complaint}. In 27 cases I have speedily arrested the disease
in from 4 to 12 hours, using in some cases the \emph{Compound powder
of Rhubarb}, and in other cases I have used only the \emph{Nitrate of Bismuth}
combined with a little Rhubarb.\endinput

<span class=font1>410</span>

<span class=font1><i>Abortion—The Medical Observer. </i>[September,</span>

<span class=font1>ABORTION—THE  MEDICAL  OBSERVER AND
ITS PUBLISHER.</span>

<span class=font0><b>by prof. cle a vel and.</b></span>

<span class=font1><b>I</b><span class=font0><b>n </b></span>the last <b>C</b><span class=font0><b>ollege </b></span><b>J</b><span class=font0><b>ournal </b></span>in answering a querie in regard to the
most effectual method of producing abortion, I made some remarks in
regard to a statement published in a medical journal of this city, which
has led to such a curious specimen of <i>Epistelation </i>that I am induced
to refer to the matter, and favor the readers of the <b>J</b><span class=font0><b>ournal </b></span>with the
letter alluded to.</span>

<span class=font1>. It is well known to the profession in this city that sometime since
Louis <b>B</b><span class=font0><b>auer</b></span>, M. D., of Brooklyn, New York, gave one or more lectures
to the profession, in the Miami Medical College of this city ; and that
while here, ho was on intimate relations with the editors of the <i>Cin-
cinnati Medical Observer.</i></span>

<span class=font1>After his return to the East the <i>Observer, </i>in March last, published a
communication from him, in which he made some remarks upon the
frequency wdth which he had been &quot;requested to assist ladies in procur-
ing abortion.&quot; Other circumstances he also stated, which led him to
suppose the production of criminal abortion is of not unfrequent occur-
rence, and referred to a lady wdiose &quot;death seemed to be connected with
criminal abortion.&quot;   He continued, page 106 :</span>

<span class=font1>&quot; Since then we have read and heard a good deal of similar instances
and trials in which medical men were implicated, <i>and in a large city
of the West the medical men with whom we happened to come in
contact, indulged in conversation that led us to the belief that the
procuring of abortion was one of their daily and most lucrative
engagements, in which even men occupying honorable distinction in
the ranks participated&quot;</i></span>

<span class=font1>In reference to this charge against members of the profession thus
published and circulated in the <i>Medical Observer, </i>in my article, I
said:</span>

<span class=font1>&quot; Some months since a physician from the East visited this city and
gave one or more lectures, by invitation, before the friends of one of
the Medical Colleges in this city. He was on intimate friendly terms
with the faculty of that College, and after his return to Brooklyn he
published a letter in which he charged upon those with whom he had
associated that their conversation led him to suppose that the unlawful
murder of unborn infants was a common occurrence with them and a
lucrative business.   So far as my personal knowledge of the profession</span>\endinput

<span class=font1><b>1857.1</b></span>

<span class=font1><i>Abortion—The Medical Observer.</i></span>

<span class=font1><b>411</b></span>

<span class=font1>of this city lias extended I consider the charge a base slander, but as I
am not on intimate terms with the members of that Faculty I cannot
answer for them and I leave them to endure the odium the charge has
cast upon them or prove its falsity as they see fit.&quot;</span>

<span class=font1>Being unwilling to allow such a charge as that contained in the
<i>Observer </i>to rest upon the profession, I expressed my full conviction
that it was a &quot;base slander,&quot; leaving those more particularly interested
to answer for themselves, for while I would ever do all I can to sus-
tain the reputation of the profession, I would not presume to be bet-
ter acquainted with the character and practice of Dr. Bauer's friends
than he is.</span>

<span class=font1>But hardly had the ink become dry on the pages of the <b>J</b><span class=font0><b>ournal</b></span>, ere
I received the following <i>public document, </i>made intentionally public by
being sent in the form of an <i>open circular </i>through the City Post-Office.</span>

<span class=font1><i>&quot; Office of the Cin. Med. Observer.</i></span>

<span class=font1><b>A</b><span class=font0><b>ugust </b></span><b>15, 1857.</b></span>

<span class=font1>Dr. <b>C</b><span class=font0><b>leaveland.—</b></span><i>Sir.</i>—As you are fully aware—the allusion to
Dr. Bauer, in your article on <i>&quot;Abortion&quot; </i>in the Aug. No. of the
<b>C</b><span class=font0><b>ollege </b></span><b>J</b><span class=font0><b>ournal </b></span>is grossly and inexcusably false; and the personal
reference to his friends here are as intensely malicious—as unwarrant-
ed ;—the <i>&quot;Observer&quot; </i>therefore declines further exchange.</span>

<span class=font1><b>E</b><span class=font0><b>dward </b></span>B. <b>S</b><span class=font0><b>tevens.&quot;</b></span></span>

<span class=font1>During the present, &quot;heated term&quot; many have found it difficult to
keep up an equipoise of temperature, but I am of the opinion that the
publisher of the <i>&quot;Observer&quot; </i>must now have as much internal as
external caloric, and according to the Thomsonian notion he is on
the high road to health and happiness. In regard to the &quot; No. 6 &quot; the
&quot;Composition,&quot; and the &quot;Capsicum &quot; which he has introduced into
his prescription, since he has manifested a great familiarity with them
I shall leave the readers of the <b>J</b><span class=font0><b>ournal </b></span>to decide if the terms <i>'&quot;false&quot;
</i>and &quot; <i>malicious&quot; </i>do not rather belong to those who publish the &quot; <i>Ob-
server </i>,&quot; and write letters, and not to myself.</span>

<span class=font1>I have stated no falsehood, but cannot tell whether Dr. Bauer has
or not. I have made no &quot;malicious&quot; accusation against the Editors of
the <i>Observer, </i>or the former Faculty of the Miami Medical College. ♦
Neither have I allowed the use of the <b>J</b><span class=font0><b>ournal </b></span>of which I am pub-
lisher or any other to do the same. Both the <b>C</b><span class=font0><b>ollege </b></span><b>J</b><span class=font0><b>ournal </b></span>and
its publisher have ever done what they could to sustain the char-
acter and the reputation of the profession, and both can possibly survive
the terrible catastrophy of the threatened loss of &quot; exchange&quot; with the</span>

<span class=font1>412</span>

<span class=font1><i>Tincture of Gelseminum in Dysentery. </i>[September,</span>

<span class=font1><i>Cincinnati Medical Observer. </i>Whether refusing to <i>exchange </i>will
satisfy the profession that the charges made in the <i>Observer </i>are without
foundation---are <i>malicious </i>and <i>false, </i>I cannot tell. The charge of false-
hood and malice must rest against the <i>Observer </i>and its correspondent
if anywhere, for my remarks as I have shown are almost a literal quota-
tion from its pages.</span>

<span class=font1>TINCTURE OF GELSEMINUM IN DYSENTERY.</span>

<span class=font0><b>by h. m. kaigler, m. d.</b></span>

<span class=font1><b>B</b><span class=font0><b>eing </b></span>favorably impressed with the medical properties of the Yellow
Jessamine, from what I read concerning its uses in various diseases by
Dr. Mayes, of South Carolina, I give you these few lines to know of
you in what diseases you have used it most. Dr. Mayes made frequent
mention of you in writing the article. I used it in one case and it an-
swered my expectations beyond my most sanguine hopes. It w<sup>r</sup>as a case
of dysentery in which the pulse ranged from one hundred and forty to
one hundred and sixty; in twelve hours time the pulse fell to one hun-
dred and two beats. I gave it because I was fearful that if I gave the
veratrum it would in all probability give rise to cartharsis in my almost
exhausted patient. It was the first time 1 had ever given the Tine, of
Jessamine; its effects were those as described by Dr. Mayes, viz: dimness
of vision, double-sightedness, inability to open the eyedids, etc.</span>

<span class=font1>I at first gave twenty drops every three hours, afterwards increased it
to forty drops because of her dangerous situation. I prepared my tinc-
ture according to the formula of Dr. Mayes ; he put four ounces of the
root chopped fine in a pint of dilute alcohol and let it stand fourteen days;
he says from twenty to fifty drops is a dose. From his mentioning your
name as using the article largely in diseases, I address you to know
what is your manner of administering the medicine and in what dis-
eases do you think it will answer ? Do you think it a remedy that will
cure Gonorrhea ? Dr. Douglas of Chester, S. C, says it will cure the above
named disease ; he says he saw it used thirty years ago in that complaint.
I should like it very much if you would give me your view<sup>7</sup>s concerning
the article. We want something that will control the vascular system
without running any risk to our patient, as ia frequently the case in
using the Tincture of Yeratrum. The people in this part of the country
are so prejudiced against it that it is impossible to use it here, so we will
have to hunt up another remedy to use in its place.</span>\endinput

<span class=font2>1857.&nbsp;<i>Tincture of Gdseminum in Dysentery.</i></span>

<span class=font2><b>R</b><span class=font1><b>emarks.</b></span><span class=font0>—</span>I am pleased to know that my efforts to extend a knowledge
of this agent are already successful to a considerable extent, and I
gladly respond to the request for farther information.</span>

<span class=font2>In regard to the value of the Tincture in Gonorrhoea I am not at
present prepared to advance an opinion, as neither my own experience
or that of my friends, has presented a sufficient amount of results on
which to base any absolute opinion.</span>

<span class=font2>In regard to its use in other diseases, perhaps it will be found to be
possessed of other properties in addition to its power as a sedative to
the heart, which will prove it to be of great value in many instances.
The splanchnic system of nerves doubtless govern the secreting organs,
as well as the processes of chemical change and nutrition, and when
these functions, as well as that of circulation are performed too ac-
tively, great harm may result and the agent which is capable of
moderating, checking or controlling these changes, may be found to
possess more valuable remedial properties than have been heretofore
suspected.</span>

<span class=font2>In the article quoted by Dr. Mayes, I said : &quot; I am satisfied that as
a sedative to the nerves branching from the spinal cord and going to
the organs of locomotion, or the nerves of voluntary motion ; and in a
lesser degree to the vagus and sympathetic nerves that are distributed
to the heart and lungs, inducing a less powerful and less frequent pulse,
and a more sluggish and feeble respiration, the Gelseminum will prove
highly satisfactory to any who may give it a trial.&quot;</span>

<span class=font2>I also accord wdth the remarks made by Dr. Mayes, as quoted in the
<b>J</b><span class=font1><b>ournal</b></span>, p. 187, except that I think as the agent impresses, as has
been stated, the <i>Exito-Secretory </i>nerves, it is capable of diminishing their
undue activity, as in Gonorrhoea and Dysentery and other forms of un-
due activity and excitability of those nerves, and hence it will prove
not only a valuable adjuvant to other treatment, but also a direct rem-
edial agent of no inconsiderable value in a very large number of dan-
gerous and painful diseases, including inflammations of the brain, the
lungs, the pleura, the viscera and in rheumatism, and various disorders
of the fluids of the body.</span>

<span class=font2>But before we can determine the actual value of this potent agent
we need the results of many carefully made trials of it, cautiously noted^
and frequently repeated, and we hope to be favored with these from
all who have made such observations and have noted the results ob-
tained. C.</span>\endinput

<span class=font2><i>Queries Answered. </i>[September,</span>

<span class=font2>QUERIES ANSWERED.
<span class=font1><b>&quot;catarrh.&quot;</b></span></span>

<span class=font2>I <span class=font1><b>will </b></span>make the following query to which I would be pleased to get
an answer in the next No. of the <b>C</b><span class=font1><b>ollege </b></span><b>J</b><span class=font1><b>ournal</b></span>. I am annoyed con-
siderably with noises in my ears, it being however almost entirely confined
to the left one. Sometimes the noise is of a ringing'character; at others
there is a roaring and rushing. It is almost constant, yet frequently
more marked in the evening and aggravated by every slight cold I take.
I have no pain in the ears but sometimes a slight itching ; I have been
very subject to irritation of the larynx from the slightest check of perspi-
ration. My tonsils also were formerly irritated and somewhat en-
larged, but by means of astringents and stimulants locally applied they
were reduced and have not given me any trouble for some months. My
general health is good, but I have been somewhat troubled with papular
eruptions on the face. I have at times felt more or less dizziness when
raising my head suddenly after having bowed down ; and not long since,
after having taken a slight cold, the noise was greatly augmented in my
ear and I became quite dizzy and for a short time (after a somewhat full
meal), was unable to walk straight. My hearing has-been slightly affected.
Now what is most probably the difficulty and what course of treatment
would you recommend % I might have mentioned that I never had acute
inflammation of the ear either external or internal, nor have I been sub-
ject to headache.</span>

<span class=font2><b>A</b><span class=font1><b>nswer.</b></span><span class=font0>—</span>Your annoyances probably arise from chronic inflammation
of the mucous membrane of the nares and pharynx ; the &quot;ringing&quot; in the
ears is produced by a partial obstruction of the Eustachian tube, preven-
) ting a free passage of air from the throat to the ear. The laryngeal diffi-
culty depends upon the same disease which is continued into the vocal
organ ; and the dizziness is consequent upon a congested state of the ves-
sels of the head, the blood being determined thither in undue quantities
by the irritation existing'in and about the pharynx. The disease, in the
acute form called &quot; coryza&quot; and &quot; influenza&quot; is a common one, especially
in countries subject to great atmospheric vicissitudes.</span>

<span class=font2>The lower animals are subject to a similar disease. It is well known
that within the heads cf mammalia there are extensive pneumatic cavi-
ties communicating with the mouth and nose. These cavities in man
are called, &quot;antrums,&quot; as that of Ilighmore ; and &quot;sinuses,&quot; as the fron-
tal, ethmoidal and sphenoidal sinuses. The elephant and the owl derive
considerable reputation for intellectual profundity from the prominence
given by these airy cells.   Pneumatic cavities in the bones of birds</span>

<FONT?>1857.J&nbsp;.^^J <i>J;  Queries Answered.&nbsp;</i><FONT?><b>'       ' 4£{J.</b>

<FONT?>rilled with rarified air serve them a good purpose in night, and in
mammals in supporting their ponderous heads.

<FONT?>The horse is subject to attacks of acute inflammation of the membrane
lining these cavities, and among farmers the disease is called " horse
distemper." The inner structure of the horns of kine is liable to take
on the same disease, and then it has the appellation of " horn-ail."

<FONT?>Considering that a large portion of the human face is taken up with
antrums and sinuses, all lined with a membrane extending from the nasal
cavities, it is not surprising that " coryza" and "catarrh" so often pre-
vail. Existing in the chronic form, the symptoms of catarrh become
somewhat varied and complicated. From sympathy of continuity the
disease extends itself through the nasal duct and the lachrymal canals,
and affects the conjunctiva, which, together with a congestion of the
vessels about the origin and along the course of the optic nerves, inter-
feres with vision. The patient is unable to read or use the eyes upon
minute objects, for much time, without dimness or a blending of objects
being the result. Hearing is impaired in a manner before hinted at,
and the patient is often treated by pretending "aurists" with appli-
cations to the external ear, leaving the real cause entirely overlooked,,
while any laryngeal trouble is nursed as " bronchitis." Frequently the
congested state of the lining membrane of the frontal sinuses will pro-
duce headache, which is mostly confined to the region over the eyes and
about the temples---nervous headache, the patient calls it---and it is apt to
recur periodically, once or twice a day. A feeling of heat and pressure
at a point half way between the crow<sup>T</sup>n and forehead, directly ov'er the
sphenoidal sinuses, is not uncommon. There is a dry, unpleasant sensa-
tion in the anterior nares, and by dilating these openings the septum nasi
will be observed redder than natural. The patient feels a disposition to
" hem" in]order to relieve the fauces and a quantity of mucus mixed with
globules of the same in a more condensed form will be brought into the
mouth by the effort. Every morning the throat has to be cleared of
" phlegm" by coughing and other efforts. In several cases mucus finds
its way down the oesophagus, exciting nausea and favoring accumula-
tions of gases in the stomach ; and from the proximity of the heart to the
stomach its functions are interrupted, causing the patient at night some-
times to spring from bed as though suffocation were about to result from
a heart disease.

<FONT?>The treatment of chronic catarrh, as given by authors, is meagre and
unsatisfactory in its results. By many physicians the disease is pro-
nounced incurable, yet they recommend their patients to snuff up the
nose cold water and use astringent gargles.   Fumigation by directing\endinput

<span class=font2><b>416</b></span>

<span class=font2><i>Application of Croton Oil to the Eye. </i>[September,</span>

<span class=font2>into the nostrils the fumes of burning sugar, ginger and cinnabar,
has comprised a part of the treatment for catarrh, but it is attended
with too little success to favor repetition.</span>

<span class=font2>Dr. Ira Warren, of Boston, for many years has douched the nostrils
and pharynx with a solution of nitrate of silver, employing a syringe
with a long curved nozzle.</span>

<span class=font2>Instead of any of the above treatment, <b>1 </b>would recommend the pa-
tient to take in some convenient vehicle the muriate or chloride of gold
in one-twentieth grain doses three times a day, and make frequent use
of an errhine composed of pulverized kalmia angustifolia and sassafras,
equal parts. Tincture of Bryonia inhaled, often proves serviceable in
this disease, and when there is much pain in the head chloroform may
be added to the Bryonia. Deafness arising from obstructions of the
Eustachian tubes may be relieved by douching the passages with a dilute
tincture of Arnica flowers. H.</span>

<span class=font2>ACCIDENTAL APPLICATION OF CROTON OIL</span>

<span class=font2>TO THE EYE.</span>

<span class=font1><b>by wellington rose, m. d.</b></span>

<span class=font2>A <span class=font1><b>very </b></span>respectable lady, of a plethoric habit, and sanguine temper-
ament, aged about 60 years, had for two or three years been afflicted
with an infirmity of her eyes. The sight was becoming dim and black
specks were apparently flying before them.</span>

<span class=font2>She poured some Croton Oil from one phial into another, an some
of it adhered to one of her fingers, with which she indiscreetly rubbed
the eye which was the most affected. That eye and eyelid immediately
began to smart and burn very severely. Sw<sup>r</sup>eet cream was first applied
to it but gave no relief. Olive Oil was next used, and the pain soon began
to subside, and ere long all disappeared. The black specks also imme-
diately disappeared, and her eyesight has been more clear and strong
since the accident than it had been for a long time previous. This I nar-
rate as it may serve as a useful hint to physicians in regard to the treat-
ment of some diseases of the eye.</span>

<span class=font2><b>R</b><span class=font1><b>emarks</b></span><span class=font0>—</span>I have delayed publishing the above for the purpose of first
obtaining the report of a case treated in this city. The patient's father
died of consumption, and his mother now suffers from cough and other
pulmonary difficulties. He and others of the family had sore eyes in <b>1846 ^
</b>from which he recovered after a few weeks. In <b>1852 </b>he again had inflam-</span>\endinput

<span class=font0>1857.]</span>

<span class=font0><i>Application of Croton Oil to the Eye.</i></span>

<span class=font0>417</span>

<span class=font0>mation of the eyes, which kept hirn from his business about two months.
He did not entirely recover from the disease this time, and in 1855
there was another acute attack w<sup>T</sup>hich lasted some four weeks, leaving
his eyes still affected. About a year ago his eyes became again sud-
denly inflamed, and the swelling and pain made him completely blind.
He was purged, bled repeatedly, cupped, blistered, and had a variety of
washes applied to his eyes, and treated on this plan of no plan until the
beginning of winter, when the lids were thickly studded with hard, irri-
table granulations, and similar hard and large granulations had sprung
up over the sclerotic conjunctiva ,and the pannus threatened to cover over
the entire cornea. There was much pain and intolerance of light, pro-
fuse lachrymation, and a free discharge of muco-purulent matter.</span>

<span class=font0>He was leeched, the lids were scarified, purgatives were administered,
and the nitrate of silver applied regularly to the eyes for some weeks.
After this the nitrate of silver was alternated with the sulphate of cop-
per, and warm cataplasms were applied. After a time he improved
but did not get well, and soon there was a relapse.</span>

<span class=font0>In April the attendant surgeon determined on inoculating the eyes
with the <i>virus </i>of <i>Gonorrhœa, </i>a Germanic transcendental mode of
treatment, apparently an offspring of the Hahnemannic school. The pa-
tient was not informed in regard to the nature of the virus of inoculation
but was told &quot;that it was a new preparation called <i>glandola.&quot;</i></span>

<span class=font0>The introduction of the Gonorrhoea virus produced very violent in-
flammation. On the third day &quot;the lids were enormously swollen and
purple, and the whole side of the face and neck erysipelatous—the dis-
charge was excessive, and he w<sup>r</sup>as racked with intense neuralgic pain in
the eyebrow. There was, at this time, no possibility of seeing the globe of
the eye, in consequence of the extreme tumefaction and acute pain
where the lids were touched.&quot;</span>

<span class=font0>The after treatment consisted in washing the eye with lead water, and
a tw<sup>?</sup>enty grain solution of the nitrate of silver, an occasional purgative,
and morphia. After many weeks the eyes improved, the pupils were
free, the cornea had a soreness, and the sight was improving daily.</span>

<span class=font0>We are half promised that the Surgeon who treated the above case,
will publish a paper <i>on the indications for inoculation, </i>and for one,
I should be pleased to have him show wherein his &quot;new preparation
called glandola&quot; is superior to the Croton oil mentioned by Dr. Rose.</span>

<span class=font0>The European papers recently make frequent mention of this novel
mode of treating eye diseases, but it will take more than a European
reputation to commend the method to the favor of American Surgeons.</span>

<span class=font0><b>vol</b>. ii. <b>no</b>. 9.—27.</span>

<img src="p - 0033-1.png" alt="" style=" width:19.30pt; height:12.67pt;">

\oldpage{418}
\section*{Caution---An Attempt at Fraud.}

\lettrine[lines=1]{}{On} making inquiry this morning at one of our first class book-seller's
for the English edition of ``Gregory's Chemistry,'' I was offered a
book which, on examination, I found to be curiously mutilated, viz:
the title page had a strip of white paper very neatly pasted over a
portion of it, which piece of paper upon close examination I found answered
the purpose of concealing the fact that the book was published
under the editorial care of one \emph{J.~Milton Sanders, M.~D.} To this attempt
at concealment on the part of the book-seller or publisher (I care not
which), I feel it my duty to call the attention of the profession. I also
wish to notice the fact that those having the work to sell feel ashamed
of the so-called ``reprint'' of Prof.~Gregory's excellent work, yet are
attempting by this contemptible trick to avoid the influence of the
many severe, though just notices the work as edited by Sanders has received.
I would call the attention of the reader to the notices on pages
\typo{82--84}{82-3 4} and 232--3 of the present volume of the \textsc{College Journal}.\hfill{}J.~F.~J.\quad

\plainbreak{1}

\textsc{Note}---Although I was prepared for almost any attempt at deceiving
the profession by the American Editor of Gregory's Chemistry, I was
not inclined to allow a statement like the above to rest solely on the
authority of the writer of the above communication, every way worthy
of entire belief as I know his statement to be; hence I went to the
book store where this work was offered for sale, and on examination
I found it was the old edition which W.~H.~Derby~\& Co.\ of this city,
had copyrighted.

The honorable proprietor of the book store viewed this attempt at
deception as any honorable, high-minded man must, and offered his
assistance in determining the extent to which the mutilation had been
carried and the fraud perpetrated.

If the repeated instances of deception of the American editor and
the publishers of Gregory's Chemistry should lead the readers of the
\textsc{College Journal} to renewed interest in the book, I would refer them
to the \emph{American Journal of Pharmacy} for Jan.\ last, page 89 \foreign{et Seq.}\ 
and to the \emph{American Journal of Arts and Sciences} for March last, for
a more extended reference to this subject. We should feel it our duty
to say much more in regard to these frequent attempts at deception
on the part of the individual referred to above, were we not aware that
our readers are fully acquainted with the character of the party engaged
in the fraud.\hfill{}C.\quad{}\endinput

\oldpage{419}
\chapter*{Editorial Department}

\fancybreak{*}

\section*{State Medical Associations.}

\lettrine[lines=1]{}{That} a State Medical Association, based upon correct principles and
sustained and conducted with the proper spirit, would be of great benefit
to the profession and to the community we have no doubt. Such an
institution would do much in developing the resources of the healing
art, correcting errors in medical practice and establishing a high standard
of qualifications as a criterion of professional respectability. We
are cons trained however, to say that such results from Associations that
have heretofore been organized we have not seen. In the old school
organization no rule has ever been adopted that has had the practical
effect of excluding ignorant and unworthy members who have tact
enough to maintain the claim of regularity (\emph{i.~e.}), subserviency to the
party dictum in regard to medical faith; though their rules are potent
in ostracising any one, however learned and accomplished, who has boldness
to think for himself and to express his thoughts.

As some of our friends in different States are making the attempt to
form State Eclectic Medical Associations we deem it an opportune
season to speak a word of fraternal admonition on this subject. We
have no right---we have no desire---to dictate, yet since this is a matter
in which all are directly or indirectly interested and since we have been
repeatedly addressed on the subject, it will not, we hope, be deemed
arrogant in us to suggest, that an Eclectic Association formed at the
present time should take a position far above that occupied by those
to which we have referred. It should take a position that shall save
it from the odium of fostering and indorsing ignorance and charlatanry,
under the cloak of professional dignity, and which shall secure to true
merit a full opportunity of being appreciated and sustained. In short
an Eclectic Association should, if practicable, be composed of members
only who have thoroughly qualified themselves for their profession, and
whose object is, not to gain notoriety nor manufacture reputation for
themselves, but to mutually improve each other, and promote the common
interests of all: of men who are willing and able to contribute by
their labors to the advancement of science, instead of making the
association bend under the weight of their self-vaunting ambition or
blush at the demonstrations of their ignorance.

But here arises a question which must be met at the outset and
upon the judicious settlement of which greatly depends the reputation,
harmony and usefulness of the association. Who shall constitute the
body?  Shall any one \emph{claiming} to be a physician be admitted as a\endinput

\oldpage{420}member, irrespective of professional attainments? The vast number
of ignoramuses who are to be found throughout the land professing to
be ``doctors,'' might furnish such an association with a formidable roll
so far as numbers are concerned, but we can readily conceive of such
an organization, with which no man who had any self-respect or any professional
reputation to lose would willingly be identified. But by what
means shall the ignorant and unprincipled be excluded? We fear it
will be difficult to do it successfully by any rule. Shall a diploma be
a sure passport to membership? Many men hold diplomas who have
very little claim besides to professional character, and who by obtaining
such documents knowing their own incompetency, have proven themselves
to be unworthy of confidence.

It seems however that there is no practicable criterion of membership
that is so nearly equitable as one having reference to graduation.
A diploma from a legally authorized Medical College with the usual
number of chairs filled by reputable professors, and requiring the usual
amount of study and collegiate instruction as a prerequisite to graduation
\emph{should} be prima facie evidence of respectable professional attainments;
and although there are diplomas which have been issued by
institutions claiming to be respectable, in the hands of men every way
unworthy, the admission of such would certainly be a less evil than to
admit in addition to these the legions of pretenders who have neither
diplomas nor medical education. We know that such a rule would exclude
some who are far above mediocrity as successful and scientific
practitioners, but such could soon obtain diplomas, and if they are men
of professional spirit they would be willing to suffer temporary inconvenience
for the sake of excluding the unworthy.

If we had the drafting of a constitution we should probably incorporate
at least four rules bearing on membership, embodying the following
principles:

1.\quad{}A voting member must be a graduate of a legally constituted
Medical College that requires the usual \typo{curriculum}{curiculum} of study before graduation.

2.\quad{}Whenever it should appear that a member had obtained his
diploma without complying with legal requirements he should be expelled.

3.\quad{}No one who professed to possess knowledge which he would not
impart to the profession, or who practiced with or encouraged the sale of
secret nostrums, should retain his membership.

4.\quad{}Reputable non-graduates should be eligible by vote of the association
to the position of honorary membership, with the privilege of
participating in the deliberations.\endinput

<span class=font1><b>1857.]</b></span>

<span class=font1><i>The Eclectic College of Medicine.</i></span>

<span class=font1><b>421</b></span>

<span class=font1>As we have already said, we have no desire to dictate to our medi-
cal brethren, but having been asked for our views on this subject we
have, as in duty bound, endeavored to give a succinct statement of them.
We will close this article by saying that we sincerely hope that Eclectic
physicians everywhere, whether members of associations or not, will
take and maintain high ground in opposition to every form of impos-
ture in medicine, and demonstrate to the world that the Ethics as well
as the medication of our branch of the profession is an improvement
upon that of the old school party.</span>

<span class=font1><b>the eclectic college of medicine,
It </b>is with pleasure that we announce to the friends and patrons of this
College that the Trustees have added to its former advantages the halls,
fixtures and furniture of the American Medical College, which recently
occupied part of the same edifice. They have also secured the services
of a full Faculty who will reside in the city. Whether Prof. Buchanan,
Emer. Prof, of Cerebral Physiology and Institutes of Medicine, who
resides in Louisville, Ky., can spend any time with us during the session
or not, we cannot say, though we hope he will be able to do so. The
Trustees and Faculty have, however, deemed it due to him who has so
long and efficiently labored for our cause and whose sympathies are still
with us, to retain his name in an honorable position in the Announce-
ment of the College.</span>

<span class=font1><b>dr. weedon's apparatus for fractured clavicle.</b></span>

<span class=font1><b>D</b><span class=font0><b>r. </b></span><b>F</b><span class=font0><b>rank </b></span>H. <b>H</b><span class=font0><b>amilton</b></span>, and others, are inclined to doubt if Dr.
Wheedon, of Albany, is the inventor of the apparatus described by him,
and which was mentioned in the <b>J</b><span class=font0><b>ournal </b></span>for August, p. <b>369.</b></span>

<span class=font1>One writes that the same apparatus has been in use some years, and
is sold in New York under the name of &quot; Bush's clavicular apparatus,&quot;
but that Dr. Wheedon has made an improvement by constructing the
rods of two parts—extensible—so as to fit persons of different heights.</span>

<span class=font1>Dr. Hamilton says : &quot; The use of a <b>T </b>splint, as a dressing for a broken
clavicle, is certainly as old as the days of Heister, who, in his great w<sup>r</sup>ork
entitled <i>Institutiones Chirurgicœ, </i>published at Amsterdam in <b>1839, </b>has
given a description and an engraving of this apparatus as it was then
used by himself.&quot;</span>

<span class=font1>In the <i>Transactions of the American Med. Association </i>for <b>1855,
</b>p. <b>407, </b>Dr. Hamilton reported a case treated by himself with an appa-</span>

<FONT?><b>422</b>

<FONT?><i>Yienna and its Hospital.</i>

<FONT?>[September,

<FONT?>ratus of this character, but which was not entirely satisfactory in its
action.

<FONT?>Dr. Hamilton thinks that the method of treating fractures of the clavi-
cle recommended by Hippocrates, and adopted by Celšusand Dupuytren,
and approved by Drs. Eastman, Eve, Buck, and Post, of placing the
patient upon his back, w<sup>r</sup>ould result more satisfactorily than by the use
of any of the various appliances figured in the books.

<FONT?><b>VIENNA AND ITS HOSPITAL.</b>

<FONT?><b>O</b><FONT?><b>ur </b>readers will observe that our intelligent friends, Dr.'s T. <b>C. </b>and
L. E. Miller, frequently refer to their experience and their professional
friends in Vienna, and it may be of interest to know that Yienna, the
capital of Austria, contains about 500,000 inhabitants.

<FONT?>The general Hospital was fouuded about one hundred years ago, by
the Emperor Joseph the II. It is the largest Hospital in the world,
the grounds embracing seven or eight squares, and is laid out in walks
shaded by rows of fine large trees.

<FONT?>Each square of ground is surrounded by a continuous line of build-
ings three stories high, and in all they contain 3,000 beds, which, how-
ever, are not always filled with patients, the usual number not averaging
more than from 2,300 to 2,500.

<FONT?>The clinics are held in this general Hospital, and usually there are
about 1,500 physicians and students in attendance on the clinics.

<FONT?>In the lying-in, or obstetrical department from 8,000 to 10,000 births
occur yearly.

<FONT?>In this city, with its University and Hospital, are presented admiral
and full opportunities for study and observation and we hope to pre-
sent to the readers of the <b>C</b><FONT?><b>ollege </b><b>J</b><FONT?><b>ournal </b>information in regard to
professional matters, as they occur in this great city, probably in each
number we issue.

<FONT?><b>BOOK NOTICES.</b>

<FONT?><b><i>The Effects of Climate on Tuberculous Diseases.   </i>By Edwix Lee, M. R. C. S., London.</b>

<FONT?><b>Being the Dissertation to which the Fiske Fund Prize was awarded June 6, 1855.
<i>The Influence of Pregnancy on the Development of Tubercles.   </i>By Edward Warren, M. D »</b>

<FONT?><b>of Edenton, N. C.   Being the Dissertation to which the Fiske Fund Prize was awarde d</b>

<FONT?><b>June 4, 1856.   Philapelphia, Blanchard and Lea. 1857.</b>

<FONT?><b>D</b><FONT?><b>r. </b><b>C</b><FONT?><b>aleb </b><b>F</b><FONT?><b>iske</b>, of Rhode Island, bequeathed at his death to the
State Medical Society a fund of two thousand dollars, directing that
the annual income of that sum should be expended in premiums for
Medical Essays, on such subjects as should be designated by the\endinput

<FONT?>1857.]

<FONT?><i>Book Notices.</i>

<FONT?>423

<FONT?>Society. This offer of <b>a </b>prize has led to an animal competition among
the best writers of the country, which competition has brought forth
some very valuable Essays upon Medical topics of great practical im-
portance.

<FONT?>The two Essays now issued in this volume were first printed in the
<i>American Journal of Medical Science, </i>and from its pages have they
been reprinted for the purpose of presenting them to the profession in
<b>a </b>more permanent form, and giving them a more extensive circulation.

<FONT?>After presenting the opinions of the most eminent pathologists in
regard to <i>the nature of pulmonary tuberculization, </i>Dr. Lee comes to
the conclusion that " tuberculization is a disease depending upon an
alteration of the blood from its normal condition." * * * " Princi-
pally caused by suppression or diminished action of the functions of the
skin and a deficiency of the red corpuscles, and that consequently it
should not be considered as merely a local disease but requires to be
treated with reference chiefly to the disordered condition of the blood
and to the causes which have been instrumental in producing it, before
it has arrived at so advanced a stage as to preclude all rational hopes
of recovery." * * * "It is therefore against the diathesis, or the
cachectic state of the system, and not against its local manifestations
that our remedies should be directed."

<FONT?>In regard to the <i>effects of climate, </i>in the treatment of tubercular
diseases, the author presents many valuable facts which he has embodied
in twenty separate cases.   He says :

<FONT?>" 6. The chief indications in the treatment of pulmonary tuberculi-
zation by means of climate, are first to remedy as far as possible the
morbid condition of the blood which constitutes the cachectic state, and
by this means to prevent or arrest the formation of the morbid pro-
duct ; and secondly, to allay the general and local excitation caused by
the organic lesion. These indications are not unfrequently opposed to
each other and in many cases the practitioner is obliged to restrict him-
self to endeavoring to fulfill the second, and to palliate the symptoms
by pharmaceutical remedies."

<FONT?>Although much has been said of late in favor of a high northern
latitude, and Dr. <b>K</b><FONT?><b>ane </b>has expressed the opinion that no one living
among the Esquimeaux will be likely to die of pulmonary tubercular
disease, Dr. <b>L</b><FONT?><b>ee </b>does not seem to have had his attention drawn to this
matter.

<FONT?>In regard to Jhe <i>influence </i>of <i>pregnancy, </i>which is the subject of Dr.
<b>W</b><FONT?><b>arren's </b>Essay, much has been said and yet but little of a reliable charac-
ter has been recorded, except in scattered fragments and isolated remarks.

<FONT?>Dr. <b>W</b><FONT?><b>arren </b>commences his essay with a quotation of the opposing\endinput

<FONT?><i>^|&nbsp;Booh Notices. </i>[September,

<FONT?>maxim of the Homœopathists, <i>"similia similibus curantur" </i>and the
Allopathic one of <i>"contraria contrariis curantur" </i>of Hippocrates
and his followers, and adopts the Allopathic doctrine as having its
foundation in reason, embodying the plain, practical, logical view <i>of
</i>the subject, and being sustained by the experience of a vast majority of
the most scientific men in every country.

<FONT?>Dr. <b>W</b><FONT?><b>arren </b>says: " The causes of phthisis may be properly divided
into two classes :   1. General causes.   2. Special causes."

<FONT?>Among the most prominent of the general causes he names <i>heredi-
tary predisposition, </i>and considers the fact that it is an hereditary
affection <i>as prima facie </i>evidence of its <i>nervous </i>origin. He next con-
siders the influence of improper aliments, the influence of impressions
made on the skin, and lastly, those impressions on the nerves connected
with the <i>emotions. </i>Of these latter he enumerates <sup>tl</sup> the gratification
of lust, indulgence in onanism, depression of spirits, violent grief, and
indeed all passions whereby immediate depression or subsequent reac-
tion is induced ;" quotes from Lombard, Moreton, Laennec, Hippocrates,
Dupay, Amestoy, Wood and Williams, in support of this proposition.

<FONT?>Among the <i>special causes </i>of the disease he names various callings,
improper clothing, suppression of habitual discharges, and various dis-
eases which tend to direct an unusual amount of blood upon the pul-
monary tissues.

<FONT?>In his second chapter Dr. <b>W</b><FONT?><b>arren </b>endeavors to prove that there is
an <i>antagonism </i>between the development of tubercle and the state of
pregnancy, and to do this he <i>assumes </i>that in pregnancy there is a dis-
position to the establishment of <i>inflammatory action, </i>which is so immi-
nent as to demand " the production of certain methods of relief to the
economy, whereby its normal condition may be secured and retained,"
and enumerates loss of blood, nausea, vomiting, disgust for food, etc., as
the means required for "the perfection of nature's most important work."

<FONT?>The final conclusion of the author is so clearly presented in the clos-
ing paragraph of his Essay that we quote it entire:

<FONT?><sup>u</sup> I have thus attempted, by arguments, facts and authorities, to
prove that pregnancy prevents the progress of phthisis, even when that
disease is perfectly developed. Whether this effort has been successful
or not, must be left to the judgement of my readers ; and to them I
- confide my cause, with the full assurance not only that their decision
will be equitable in regard to all that has been urged in support of my
position, but that they will agree with me in the conclusion that if
pregnancy can arrest the progress of consumption when fully established,
then for a still stronger reason it must "retard the development of
tubercles in those predisposed to phthisis." <sub>s</sub>\endinput

<span class=font1><b>1857.]</b></span>

<span class=font1><i>Book Notices.</i></span>

<span class=font1><b>425</b></span>

<span class=font0><b><i>Domestic Medicine. </i>A Treatise on the Practice of Medicine, adapted to the Reformed
System, comprising a Materia Medica, with numerous Illustrations, By J. Kost, M. D.,
etc., etc., etc.   Cincinnati, R. Wilson, 1857.</b></span>

<span class=font1><b>F</b><span class=font0><b>rom </b></span>the date upon the title page one would naturallv suppose this
book another recent addition to the literature of the Reformers in Med-
icine, but a cursory examination of the volume leads to the conviction
that it is a simple unchanged reprint of the work copyrighted by the
author some six years since.</span>

<span class=font1>Prof. Kost has written tw<span class=font0><b><sup>7</sup>0 </b></span>or three other works, which have had
a fair circulation, and has been engaged as a teacher in several
medical colleges, and hence the author and his productions are very
well known to the profession. In this work was embodied much of what
was known at the time it was written of the principles and practice of
that branch of medical reforms to which the author has been more in-
timately connected, and hence it is of interest as indicating the very
great progress <b>R</b><span class=font0><b>ational </b></span><b>M</b><span class=font0><b>edicine </b></span>has made during the past few years.</span>

<span class=font0><b><i>An Epitome of the American Eclectic Practice of Medicine; </i>embracing Pathology, Symptom-
atology, Diagnosis and Treatment. Containing also a complete list of the remedies used
by Allopathists, Homœopathists, and an Eclectic Pharmacy and Glossary. Designed
for Physicians, the Student of Medicine, and as a Domestic Practice for Families. By
William Paine, M. D., Professor of the Principles and Practice of Mediciue and Pa-
thology in the Eclectic Medical College of Pennsylvania. Philadelphia, H. Cow-
ferthwait and Co. 1857.</b></span>

<span class=font1><b>F</b><span class=font0><b>rom </b></span>the many subjects embraced in this title-page we might natu-
rally expect a library of many volumes ; but instead the author has been
able to present what he had to&quot; offer on these important matters in a
single twelve-mo. volume of about four hundred pages. Whether our
Allopathic, Homoeopathic, Hydropathic, and Chrono-thermal neighbors
will think full justice has been done them in this little manual or not,
we cannot determine. We are sure, however, that physicians, students
and families will agree with us in regretting that the majority of the
subjects have been so briefly considered.</span>

<span class=font1>We have repeatedly expressed our objections to the use of this class
of works for students and physicians, as tending to produce a careless,
indifferent or superficial consideration of matters of the deepest scientific
and humanitary interest, and a consequent empirical and unsuccessful
practice. But children usually creep before they walk, and walk
before they are able to run, and hence we may reasonably anticipate
these primal efforts to supply the profession with the demanded pro-
gressive literature will partake somewhat of a juvenility of character.
Even failures may, however, be considered successful in one point of
view, as the masts of sunken vessels serve as indicators of the pres-
ence and location of the rocks upon which they were stranded.</span>

<span class=font2><i>Book Notices.&nbsp;</i><span class=font0>1 </span>[September,</span>

<span class=font2>&quot;We anticipate that Prof. Paine will be called upon to issue a second
edition of his work, and in anticipation of that event, we would sug-
gest that in regard to his nomenclature for remedies he may be able to
make some corrections of errors and introduce some improvements.</span>

<span class=font2>In accordance with the usage of modern Chemists, all names of
Alkalies and Alkaloids should terminate with an <i>a, </i>as in Soda, Potassa,
Quinia, Morphia. The names of all Acids terminate in <i>ic, </i>as Acetic,
Tannic, Gallic, Sulphuric, Nitric. The names of all Resins and Resin-
oids terminate in <i>in, </i>as Resin, Rosin, Podophyllin, Jallapin. The
names of such agents as may belong to one of these four classes, but
whose character is as yet undetermined by scientific writers, are made
to terminate in <i>ine, </i>as Cornine, Santonine, Salicine, Piperine.</span>

<span class=font2>In looking over the list of medicines named in this work we are sorry
to perceive that these simple and truly scientific rules have been disre-
garded. Hence we have &quot; Cinchona&quot; as the name of an alkaloid
principle, instead of Cinchona; &quot; Emetine&quot; denned as the alkaloid
principle of Ipecacuanha; &quot; MyriczV as an alkaloid; &quot; Quhmie,&quot;
and other similar errors, w<sup>T</sup>hich can readily be corrected in future editions.</span>

<span class=font2>In conclusion we would cordially welcome, and heartily exert the in-
fluence of our <b>J</b><span class=font1><b>ournal </b></span>to assist all publications which are calculated to
enlighten and elevate the profession and advance the cause of humanity.</span>

<span class=font1><b><i>Remarks upon Aclhemy and the Alchemist, </i>indicating a method of discovering the tru«
nature of Hermetic Philosophy ; and showing that the search after the Philosopher's
Stone had not for its object the Transmutation of Metals. Being also an attempt to
rescue from undeserved opprobium the reputation of a class of Extraordinary Thinkers
in past ages. &quot;Man may not live by bread alone.&quot; Boston ; Crosby, Nichols, and
Company. 1857.</b></span>

<span class=font2><b>S</b><span class=font1><b>ome </b></span>two years since, the writer of the present volume published a
small pamphlet on the subject of Alchemy, in which he presented the
idea that the <i>Philosopher's Stone </i>of the old Hermetic Philosophers
was not a supposed reality, but a symbol of <i>wisdom </i>or <i>truth.</i></span>

<span class=font2>That pamphlet being designed solely for the friends of the author, was
never offered for sale. It attracted the attention of a writer in the
<i>Westminster Review, </i>who did not aim at the same conclusion with
the pamphleteer, and hence the author was called upon to defend his
position. He therefore pursued his investigation by reading still
other works on Alchemy, and this farther investigation only confirmed
him in the opinion that although the Alchemists were the forerunners
of modern Chemists, yet their investigations were not mainly directed
to material matter, but that &quot;the <i>subject </i>of Alchemy was <i>Man; </i>while
the <i>object </i>was the perfection of Man, which was supposed to center in</span>

<span class=font1><b>1857.]</b></span>

<span class=font1><i>Book Notices.</i></span>

<span class=font1>427</span>

<span class=font1>a certain unity with the Divine Nature;&quot; * * * &quot; that the salva-
tion of man---his transformation from evil to good, or his passage from
a state of nature to a state of grace---was symbolized under the figure
of the transmutation of metals, * * * that the Alchemists were
<i>Reformers </i>in their time, * * that they were religious men when
the spirit of religion was buried in forms and ceremonies, and when
the priesthood had armed itself with the civil power to put down all
opposition, and suppress all freedom, intellectual, civil, moral and re-
ligious.&quot; Hence these <i>freemen </i>were accustomed to w<sup>7</sup>rite and speak
to each other in symbolic language, using the terms salt, sulphur, gold,
mercury, etc., while refering to principles and truth.</span>

<span class=font1>According to the writer of this volume, the whole subject of Alchemy
is Man, but each writer for the most part designated man by a word of
his ow<sup>T</sup>n choosing. Hence one called man Antimony, another Lead,
another Zinc, etc., and under these and similar names we can if we
choose to investigate the matter, learn what those writers thought of
God, Nature, and Man, or Man, Nature and God---one and three, three
and one. That the Alchemists were Protestants, when Protestantism
could not find open free speech for fear of the <i>auto da fe, </i>the dungeon
or the stake, seems apparent from their entire writings, and that their
opinions were expressed to each other in quaint language, distinctly
understood by each other, but unintelligible to the uninitiated : that
Alchemy, or Hermetic Philosophy, was a religious philosophy or faith,
and the writers on this subject teachers of the new---or protestant form
of religion, rather than of new ideas in chemistry.</span>

<span class=font0><b><i>The Physician's Visiting List, Diary, and Book of Engagements for </i>1858. Philadelphia,
Lindsat and Blakiston.</b></span>

<span class=font1><b>T</b><span class=font0><b>he </b></span>united voice of all who have enjoyed the facilities and advant-
ages of this Pocket Account Book and Diary has been only in praise
of this convenient Annual, which has become a <i>necessity </i>to those who
have used it.</span>

<span class=font1>It contains an Almanac, a Table of Poisons and their Antidotes
(rather too brief), a Table for calculating the period of Utero Gestation,
a Visiting list for the year, for 25 or 50 patients daily, and a vast
amount of other convenient memoranda, by means of which physicians
avoid errors, or forgetfulness---and by the aid of which many times
the cost of the work may be saved daily.</span>

<span class=font1>This Visiting List is bound in a convenient form, and should be found
in the pocket of each practicing phys ician. Copies can be sent by mail
for $0,75, and $1,00.</span>\endinput

<FONT?><i>Book Notices.</i>

<FONT?>[September,

<FONT?><b><i>The Five Gateways to Knowledge; </i>by George Wilson, M. D , F. R. S. E., etc etc.,
Philadelphia, Parrt and McMillan. 1857.</b>

<FONT?><b>T</b><FONT?><b>his </b>little volume receives a hearty welcome ; it is one of those rare
works we occasionally meet with in which the useful is combined
with the ornamental, and we find something in each page which will
amuse and instruct.

<FONT?>The style is simple and clear, and at the same time beautiful and
comprehensive. The dull monotony that too often characterizes such
works is not observable here, but each page is enriched with original
thought, clothed in language which the child may understand, and the
sage may read with pleasure.

<FONT?>The Author imagines himself standing sentinel to the city of the Soul
described by <b>J</b><FONT?><b>ohn </b><b>B</b><FONT?><b>cnyan</b>, and as he watched the five gates known
as the <sup>u</sup> Eye-gate, Ear-gate, Nose-gate, Mouth-gate, and Feel-gate," he
records all who bear tidings to the dwellers within.

<FONT?>Seldom have we seen so many of the truths of philosophy enriched
by the beauties of poetry. In his description of the Ear the following
occurs which we give as characteristic of the work.

<FONT?><sup>u</sup> But music is never more delightful than when listened to in utter
darkness without obtrusion of music paper, or instrument, or perform-
ers, and whilst we forget that we have ears, and are content to be liv-
ing souls floating in a sea of melodious sound."

<FONT?>We cannot recommend the work more highly than to advise our readers
to obtain and read it for themselves. <b>J.</b>

<FONT?><b><i>The American Family Physician, </i>or Domestic Guide to Health. For the use of Physi-
cians, Families, Plantations, Ships, Travelers, etc. By John King, M. D., Professor of
Obstetrics and Diseases of Women and Children, in the Eclectic College of Medicine;
formerly Professor of Materia Medica, Therapeutics, and Medical Jurisprudence in the
Memphis Institute ; author of the American Eclectic Dispensatory, the American
Eclectic Obstetrics, etc., etc. Cincinnati, Lonqley Brothers, Publishers, 1857.</b>

<FONT?><b>T</b><FONT?><b>he </b>above title-page, and several sheets of the above work have
been presented us by the Publishers, with the intimation that it would
be ready for sale in about a month.

<FONT?>When the work is published we shall express our opinion upon its
contents and character; but for the present would simply refer to it
knowing that most of our readers are so well acquainted with the former
works of the Author as not to require any commendation of the one
forthcoming.

<FONT?><b><i>The American Journal of Dental Science </i>for J uly.</b>

<FONT?><b>T</b><FONT?><b>his </b>able advocate of the Dental profession is filled with valuable
matter, which it would be well for physicians, as well as dentists, to
become acquainted with.\endinput

<span class=font1><b>1857.]</b></span>

<span class=font1><i>■ Boole Notices.</i></span>

<span class=font0><b><i>Demand of the Age on Colleges.</i>—Speech Delivered by the Hon. Horacb Mann, President of
Antioch College, before the Christian Convention, at its Quadrennial Session, held at
Cincinnati, Ohio, October 5, 1854.   New York.   Fowler and Wells, 1857.</b></span>

<span class=font1><b>T</b><span class=font0><b>his </b></span>is the product of a Master Mind, governed by a Master Will,
which leads to a bold, fearless enunciation of whatever is considered as
<i>truth, </i>particularly those truths which have a bearing upon the spiritual
or physical welfare of the human race.</span>

<span class=font1>Those who give utterances to <i>positive </i>and more paiticularly <i>original
</i>ideas, are apt to find far more opponents than followers at first, and
hence, probably, but few will be ready to take the same views as our
author on the first perusal of this little work, and yet most will read it
the second and the third time, and whether they adopt the opinions
advanced or otherwise, they must be benefitted thereby.</span>

<span class=font1>We most earnestly wish that such Speeches as this and the Address
to the Students of Antioch College, which is added as an Appendix,
together with the <i>Address to the Students of the Eclectic College of
Medicine, </i>by Prof. I. <b>J. A</b><span class=font0><b>llen</b></span>, M. D., L.L. D., a part of which was
published in the <b>C</b><span class=font0><b>ollege </b></span><b>J</b><span class=font0><b>ournal </b></span>in March last, could be read by every
Medical Student in the land.</span>

<span class=font1>That our readers may judge of the character oC this Speech, we
make the following brief extracts:</span>

<span class=font1>&quot; Let me say, then, in a single sentence, that our hope and aim is to
<i>meet not merely the advanced, but the advancing Demands of the Age.
</i>What, then, does the age demand that our College should be ? or rather,
in the first place, w<sup>T</sup>hat does the age demand that it should not be ?</span>

<span class=font1>&quot; It should not be an Egyptian pyramid, for the preservation of old
mummies, literal or psychological. Whatever has vitality in it, what-
ever has truth in it, these let us religiously preserve, for Truth is endued
with immortal youth and beauty, and can give forever and to all with-
out self-exhaustion or impoverishment. But as for the mummies, let
the Arab peasants continue to burn them, as travelers tell us they are
now accustomed to do, for cooking their dinners. Would to Heaven
that all the tyrants of the preseŋt day, political and mental, could be
put to as good a use.</span>

<span class=font1>&quot;Dugald Stewart likens some of the literary institutions of his time
to old hulks sunk in the stream, which by their stationary position
show to the passers-by how far the living have advanced beyond the
dead on the Biver of Progress. We do not desire to enter into any
competition with those old hulks for the honor or repose of their con-
servatism. Among the moral surveyors who are measuring the onward
march of mankind, we would aspire to be found among the foremost
chaiubearers, pressing right forward in defiance of any obstacle, and
up any a eclivity; and let those who come after keep the tally.</span>

<span class=font1><b>430 <sub>;</sub>&nbsp;.&nbsp;</b><i>Book Notices. </i>[September,</span>

<span class=font1>&quot; We loathe to be classed among the fossil remains of by-gone ages,
as belonging to that order of men who, if they had been born during
an eclipse of the sun would have protested against the return of its
light, or, if they had been born in the ark during the deluge of Noah
would have remonstrated against the subsidence of the waters.&quot;</span>

<span class=font0><b><i>The American Eclectic Practice of Medicine. </i>Bv I. G. Jones, M. D., late Professor of the
Theory and Practice of Medicine in &quot; The Eclectic Medical Institute of Cincinnati,&quot;
etc., etc. Extended and Revised at request of the Author by Wm. Sherwood, M. D.,
Professor of Medical Practice and Pathology in &quot; The Eclectic College of Medicine;&quot;
formerly Professor of General, Special and Pathological Anatomy in &quot;The Eclectic
Medical Institute of Cincinnati/' etc. In two volumes. Cincinnati: Moore, Wil-
stach, Keys and Co., 25 West Fourth Street. 1857.</b></span>

<span class=font1><b>W</b><span class=font0><b>e </b></span>are not informed how much extension and revision these volumes
will undergo by Prof. Sherwood, but his connection with the work when
first issued and his well known talent as a writer and a speaker must
entirely satisfy our readers that the forthcoming books will be every
way worthy those who are engaged in their production.</span>

<span class=font0><b><i>The American Journal of Insanity </i>for July.</b></span>

<span class=font1><b>I</b><span class=font0><b>f </b></span>physicians will admit as true the statement of <b>G</b><span class=font0><b>rotius </b></span>that &quot; The
care of the human mind is the most noble branch of Medicine,&quot; they
will appreciate the value of this admirable Quarterly which is entirely
devoted to that speciality. This number is the initial one to the four-
teenth volume, and we think fully equal to any of its predecessors.</span>

<span class=font0><b><i>The Half-Yearly Abstract of the Medical Sciences, </i>etc. etc., edited by W. H. Ranking, M. D.
and C. B. Radoliffe, M. <i>D., </i>No. 25, January to July, 1857. Reprinted by Lindset and
Blakiston, Philadelphia.</b></span>

<span class=font1><b>W</b><span class=font0><b>e </b></span>have already expressed the opinion that this is the <i>best </i>Medical
Periodical published for the active practitioner, and that no one who
has become acquainted with its value can afford to do without it.</span>

<span class=font0><b><i>Alphabetical Index to Braithwaite's Retrospect; </i>Embracing Part I.—XXXIV. 1840—1857
and comprising the whole seventeen years of Republication. New York, Stringer and
Townsend. 1857.</b></span>

<span class=font1><b>T</b><span class=font0><b>his </b></span>is an admirable and most valuable index to the well-known
Semi-Yearly publication of Dr. <b>B</b><span class=font0><b>raithwaite</b></span>, and the many readers of
that excellent work will welcome it with great pleasure.</span>

<span class=font0><b><i>The British and Foreign Medico-Chirurgical Review; </i>or Quarterly Journal of Practical
Medicine and Surgery.   July, 1857.</b></span>

<span class=font1><b>T</b><span class=font0><b>he </b></span>reprint of this work, by the Messrs. <b>W</b><span class=font0><b>ood</b></span>, always comes
promptly. The present number fully sustains the high reputation it
has acquired during a period of near forty years continuous publication.</span>

<FONT?>1857.]

<FONT?><i>Abstracts.</i>

<FONT?><b>431</b>

<FONT?><b>ABSTRACTS.</b>

<FONT?><b>E</b><FONT?><b>pilepsy.</b><FONT?><b>---</b>Dr. Eobert Hunt maintains that an abnormal excess of
alkali in the blood directly predisposes the nervous system to disease
and not only this but that it also causes various chemical changes in'
the blood, which result in the generation or retention of noxious matters
in the system, which excite a tendency to spasms and convulsions. In
epilepsy, where there is an uniform deficiency of organic principles, as
urea, and an excess of mineral matters, especially chloride of sodium,
in the urine, they have some connection w<sup>7</sup>ith the state of the blood and
the disease in question, and in such cases he has effected cures by keep-
ing the bowels regular, administering twenty drops of diluted nitro-
muriatic acid two or three times a day, before meals, with the use of a
dilute nitro-muriatic acid bath every night, remaining in it 15 or 20
minutes each time; subacid fruits are to be used freely.

<FONT?><b>H</b><FONT?><b>emoptysis.</b><FONT?><b>---</b>The only remedies to be trusted in severe cases are oil
of turpentine, gallic acid, chloride of sodium, or nitre with digitalis;
alum is not sufficient to meet the danger. Blood-letting temporarily
arrests the bleeding, but is dangerous, owing to the debility and increased
susceptibility to the intercurrent affections it gives rise to. In using the
gallic acid or turpentine in severe cases we should first procure a transient
check of the hemorrhage by ligatures to the limbs and ice to the chest,
allowing the internal means to consolidate this temporary cure.

<FONT?><b>H</b><FONT?><b>emorrhagic </b><b>D</b><FONT?><b>iathesis.</b><FONT?><b>---</b>Give full diet: mutton chop, eggs, milk,
etc.; as a medicine, gallic acid five grains, dilute sulphuric acid twenty
minims, decoction of cinchona an ounce and a half, mix for a dose; to
be repeated three or four times a day. If the hemorrhage be from the
mouth, apply turpentine with a sponge three times a day. If the surface
of the skin be extensively ecchymosed, give dilute sulphuric acid 20
minims, sulphate of iron four grains, sulphate of magnesia half a drachm,
infusion of columbo an ounce and a half; mix for a dose, and repeat
three times a day. Should there be restlessness, give at night muriate
of morphia one-third of a grain, lactucarium four grains.

<FONT?><b>B</b><FONT?><b>right's </b><b>D</b><FONT?><b>isease.</b><FONT?><b>---</b>Liquor Ammoniae Acetatis, by acting on thelskin
and relieving the congested condition cf the kidney, has been found the
best of all remedies for this disease ; diuretics at the same time to be
strictly avoided, but solvents may be used.

<FONT?><b>P</b><FONT?><b>iles.</b><FONT?><b>---</b>When associated with protrusion of the rectum, may be
treated with nitric acid ; it is much safer and quite as effectual <b>a
</b>remedy as the ligature or knife, and there is much less danger from
phlebitis.\endinput

<span class=font1>432</span>

<span class=font1><i>Miscellany.</i></span>

<span class=font1>[September,</span>

<span class=font0><b>M S CELL ANY.</b></span>

<span class=font0><b>THE BOSTON JOURNAL AND PATENTS.</b></span>

<span class=font1><b>I</b><span class=font0><b>t </b></span>is somewhat amusing to see the various <i>inconsistencies </i>would-be
conservatives often indulge in. Even the fossiliferous <i>Boston Medical
and Surgical Journal, </i>the type of conservatism has recently com-
mended a patented article, and according to the Ethics of the Am.
Med. Association, has now placed itself quite beyond the pale of the
<i>regular </i>profession. In an Editorial published in the Journal for June
the 25th, it adopts the usual style of the paid Editorials of the secular
papers while puffing nostrums, and says:</span>

<span class=font1>&quot; Although we are always reluctant to recommend any medicine or
instrument, with the virtues of which we are not personally acquainted,
we are induced by the favorable opinion expressed to us by two of
the most eminent physicians of this city concerning <i>Parker's Patent&quot;
</i>etc., etc., <sup>u</sup> to invite the attention of the profession to a contrivance
which seems admirably adapted,&quot; etc., etc. In February last while
giving a commendatory notice of another patented article, the Boston
Journal was a little more cautious, and left out the word &quot;patent,&quot;
possibly with the idea of thus escaping the fiat of excommunication
which was pronounced against the State Medical Society, of Ohio.</span>

<span class=font0><b>&quot; Who shall guard the Shepherds ?&quot;</b></span>

<span class=font0><b>NEURALGIA.</b></span>

<span class=font1>A <span class=font0><b>correspondent </b></span>who has read Dr. <b>B</b><span class=font0><b>arrows</b></span>' article published on
page 256 of the <b>J</b><span class=font0><b>ournal</b></span>, has referred to a work by Dr. <b>M</b><span class=font0><b>egelin </b></span>entitled
<i>&quot;Researches in regard to Facial Neuralgia&quot; </i>and published at Stras-
bourg in 1816. We will consider it a favor to be furnished with still
farther information in regard to this book, and the remedies used by Dr.
<b>M</b><span class=font0><b>egelin.</b></span></span>

<span class=font0><b>MEETING OF THE INDIANA ECLETIC STATE MEDICAL ASSOCIATION.</b></span>

<span class=font1><b>T</b><span class=font0><b>he </b></span>members of the Indiana Eclectic Medical Association, and all
Eclectic physicians throughout the State, are requested to meet at the
city of Indianapolis on the 6th day of Oct., 1857, at 2 o'clock <b>F. M.,
</b>for the purpose of perfecting an organization of the faculty throughout
the State. A general attendance is earnestly solicited as business of im-
portance to the profession will come before the meeting.   By order,</span>

<span class=font1>O. H. <b>K</b><span class=font0><b>endrick, </b></span><i>Secretary.</i></span>

<span class=font0><b>MARRIED.</b></span>

<span class=font0><b>At North Brookfield, Mass , 4th inst., Henby T. Bates, M. D., of Lowell Mass., to Miss
Lottie H. Bush, of North Brookfield.</b></span>\endinput


\end{document}