% The next lines tell TeXShop to typeset with xelatex, and to open and save the source with Unicode encoding.

%!TEX TS-program = xelatex
%!TEX encoding = UTF-8 Unicode

\documentclass[9pt,a5paper]{memoir}
\usepackage{fontspec,xltxtra,xunicode}
\usepackage{hyperref}

\defaultfontfeatures{Mapping=tex-text}
\setromanfont[Mapping=tex-text, Ligatures={Common}]{P22 Foxtrot Pro}
\setsansfont[Mapping=tex-text, Ligatures={Common}]{Lucida Grande}


\frenchspacing
\plainfootnotes

\newcommand{\typo}[2]{\textbf{#2}\marginpar{\tiny{``{#1}''? ---\textit{Ed.}}}}
\newcommand{\oldpage}[1]{\marginpar{\centering{}\tiny{pg. {#1}}}}

\begin{document}

\chapter*{Transcriber's Note}This document is a draft, being re-typeset page by page, based on scanned and OCRed images of the original 1857 book's pages. See \href{http://github.com/Vaguery/collegeJournalOfMedicalScienceSeptember1857}{the github project} for more information, or to fork and work on the project.
\chapter*{Trancription begins:}

THE
COLLEGE JOURNAL
OF MEDICAL SCIENCE.

Vol. II.
SEPTEMBER, 1857.
No. 9.

ORIGINAL CONTRIBUTIONS.

\section*{Nature and Causes of Asiatic Cholera.}

by Prof.\ Michael von Visanik.

Translated from the German by T.\ C.\ Miller, M.\ D.

\textsc{Ever} since the first appearance of the epidemic cholera physicians
have endeavored to investigate and discover the inner nature and true
cause of the disease, that upon the knowledge thus obtained they might
base a rational mode of treatment.

But in accordance with the prevailing usages of the profession, the
nature and causes of this fell disease have remained unknown and the
attempted explanations have been based upon hypotheses and empirical
observation and not upon any rational and accurate course of
observation and reasoning. Some have explained the etiology as
a ``\textit{virus}'' others as ``\textit{humores alienati},'' and have contented themselves
with these unintelligible abstractions as sufficient to account for
the origin and duration of the contagion and disease. But we have
recently become dissatisfied with these phrases in place of ideas, and
since the researches of the pathologist and the chemist have given us
more definite information, the reformatory truths obtained from them
have enabled us to penetrate far more profoundly into the true nature
of this epidemic.

% \textsc{vol. ii. no. 9.---25.}
\endinput

It would prove fatiguing to the reader for me to enumerate all the
various hypotheses which have from time to time obtained in regard to
the etiology of cholera, and hence I shall only glance at a few of the
more prominent of them as prefatory to the opinions and conclusions to
which I have arrived.


It has been observed that fat, and particularly the fats that are to be
found in diet where the food has become sour and rancid, will if eaten
often produce symptoms very closely analagous to cholera, and from
this observation the conclusion has been drawn that the cholera miasm
is one produced by a specific decomposition of animal tissues forming
a combination of gases similar to those evolved in the decomposition
of sausages, and which is known as the \emph{sausage poison}. That this
view cannot be correct has become evident to nearly all.


Others have supposed that cholera is caused by what they have been
pleased to style the \emph{cholera-mite} a supposed microscopic animalcule
diffused in vast quantities through the air, the food and the drink, and
that these animalcules are the \emph{potentia nocens} of the disease. The
advocates of this hypothesis attack all other opinions and defend their
own with great violence, and are very strenuous in the advocacy of
what they assert to be the cause of cholera. Among those of this
class are many able microscopists, and yet they have neglected to bring
forward the only strong and indisputable evidence necessary to establish
the accuracy of their deductions. They are unable to bring forward
any person who has been so fortunate as ever to have seen this
wonderful cholera-mite.


Not a few have directed their attention mainly to the stomach and
the intestines and think they find in the vomiting and purging the true
explanation of the cause of epidemic cholera; which cause to them is
the irritative and congested condition of the alimentary track. On
this hypothesis have they based their high estimate of the value of
opium and have viewed it as a specific against the disease.


This view of the matter is so superficial and so illy sustained by the
symptoms of the disease and the results of treatment that most have
abandoned it. All who have had any experience in cholera and its
treatment will have observed that the danger of the attack is by no
means proportionate to the activity of the vomiting and purging, but
that it frequently appears extremely severe and fatal where but little
vomiting or purging have occurred. But this class of persons have
their attention so closely drawn to their fancied seat of the difficulty
that they never perceive these facts and never have the faintest glimpse
of the true cause of the disease.
\endinput

\oldpage{387}Some also, have considered cholera to be caused by some derangement
of the chylopoietic vicera, but the anatomical and autopsical examinations
have given no countenance to this conclusion.

The peculiar symptoms manifested in the asphyctic condition of
cholera has induced some to suppose that the whole difficulty arose
from a weakness or loss of functional power of the nerves, but more
especially of the spinal cord; but the revelations of the dead-house
have not confirmed these conclusions and did much to disprove the
accuracy of the opinion entertained.

More recently in their search for the seat and origin of cholera
physicians have been guided by what they have learned in regard to
the blood, its changes and decompositions, and they have observed that
in cholera there is a partial decomposition of the blood, with contemporaneous 
alteration in the walls of the capillaries by means of which
the \emph{serum sanguinis} in large quantities passes through their walls, or
is poured directly into the stomach and intestines, leading to the profuse
vomiting and purging by which this serum is removed entirely beyond
the organism. This decomposition of the blood and the outflow of
its watery portions is often produced with wonderful rapidity---while
the more solid portions, as the red-corpuscles, is retained in the vessels,
and thus the fluid is rendered thick, dark and very liable to stagnation,
to clog up and produce congestion of the vessels. The thick blood
also stagnates in the vessels of the skin, and causes the blue appearance
nearly always observed in that tissue. In the larger veins, and in the
brain, in the liver, and the spleen, and the lungs this stasis also occurs,
and hence the difficulty of breathing, the deafness, the aphonia, the
thirst, the scanty urine, the coldness and numbness, that accompanies
this disease.

Although this explanation is in accordance with the observations of
the profession, yet it may not satisfy all, for many deny that any explanation
can be given which shall prove satisfactory, and they desire
to know how it is, if the blood is \typo{really}{realy} separated, that a part of the
serum is not thrown into the cellular tissue, and not the whole of it
poured into the alimentary canal, or through the skin in profuse perspiration.
Why are not the pleural and abdominal cavities filled with this
fluid? why is the patient not attacked with hydro-thorax and anasarca?
are questions urged against this opinion.

There can be no doubt but the nerves which supply the vital force to
the walls of the blood-vessels are impaired during an attack of cholera
but those who entertained the opinion that the nervous power is diminished,
are divided as to which is the \emph{prime} cause of the difficulty, one\endinput
\oldpage{388}class supposing the atony of the vessels and the consequent out-flux
of the fluid impairs the nerves, while the other supposes that a poison
has been introduced into the system which acting directly on the brain
thus lessens the nerve power, and the loss of that power leads to the
atony of the blood-vessels and consequent exhalation of the serum.

There seems to be \emph{three} principle classes of opinions as to the primal
nature of cholera. 1st. A primary poisoning of the blood. 2nd. A
primary affection of the nerves. 3d. A primary local affection of the
alimentary canal.

From what has been said, we may perhaps draw the conclusion, that
cholera, like other epidemics, as scarlatina, measles, intermittent fever,
typhus fever, influenza, etc., owes its origin to a cause having a uniform
origin, or at least a uniform character, while, as in the other instances,
as to its peculiar nature, we may be entirely ignorant. Many physicians
have striven hard to learn the exact nature and character of this morbific
agent, and yet they have acknowledged a want of success.

With others, I too, have tried to solve this mystery and having had
an opportunity of observing personally more than two thousand cases
of cholera in different epidemics, I am led to present the following observations.

Cholera does not appear every where and at all times to possess precisely
the same characteristics. At one time it will appear mild and
very easily cured. But this slight form only appears in those persons
whose systems appear to have but a slight \emph{disposition} for the disease,
and hence it cannot exert as powerful an influence upon such as it does
upon those who are more predisposed to its attacks, and the disease will
take \emph{the form of cholera periculosa exquisita}.

Most of those who are disposed to inflammatory disease seem also
disposed to receive cholera, and hence the two diseases are often met
with in company. In ``\emph{cholera febrilis}'' there are several congestions
of the head, the lungs, or heart, in conjunction with the more ordinary
symptoms of cholera. In persons who have a predominating disposition
to vomit, the cholera will commence with vomiting, while with
those who are disposed to looseness of the bowels, it will commence
with a diarrhœa, while with those who are predisposed to the cholera,
and at the same time their nervous and arterial systems are equally
susceptible, the disease will take the form of \emph{cholera fulminitisima
asphyctica}.

Climate exercises upon any prevailing disease a powerful influence.
This is manifested in epidemic cholera. In some countries and climates,
it appears as \emph{cholera febrilis}, with intense congestions; in others\endinput
as \emph{cholera spasmodica}; while in other climates, vomiting and purging
are the most prominent features of the disease.

It is' also important, in the investigation of the cause of cholera, to
learn to distinguish between the genuine cholera and other forms of
disease, as well as to decide what results are produced from cholera
and what from other causes, and it is only those who have had considerable
experience in this disease who can always make this distinction.

The pathologico-anatomical, and the pathologico-chemical results of
the disease \typo{contribute}{coutribute} largely to the knowledge requisite to determine
the cause and nature of it, and they therefore must never be neglected
or overlooked. The post-mortem examinations and the examinations
of the secretions and excretions must be made, particularly the secretions
and excretions of the liver, the spleen, the kidneys, the stomach,
and the intestines, but more particularly the excretions of the kidneys,
and on the results of these examinations may we base our own view of
the nature and cause, as well as of the treatment of the diseases.

During the last epidemic the post-mortem examinations have furnished
in general and in particular, the same results as those obtained
during former epidemics. The skin of those who died of the disease has
been cyanotic, or blue colored, particularly the skin over the extremities.
The vessels in the sinuses and meninges of the brain have contained
much thick dark blood. The inner meninges have been congested, with
ecchymoses. The brain itself has been firm, and on intersection has
disclosed similar ecchymoses in its substance. The pleura and pericardium,
and all the serous membranes have a slippery feeling, showing a
separation of the delicate lining from the subjacent parts, and covered
with a glutenous albuminoid fluid. The lungs are dry and of a clear
red, and bloodless. The heart, particularly the left ventricle has been
drawn up, and in it and in the large vessels we have found a thick
black, tarry blood possessing little or no power of coagulation. The
liver is pale, and the gall-bladder filled with much dark bile. The
spleen is usually enlarged, dark red-brown, and its enveloping membrane
thrown into wrinkles. The stomach and intestines are filled up
with a rice-watery, or a bloody colored fluid, with the mucus membrane
of the stomach swollen and injected. The epithelium of the intestines
in nearly the whole extent is dead, and rubbed off; the mucus membrane
red, and the follicles swollen. The kidneys in most instances
presented distinctly the changes which are found in the disease known
as \emph{morbus Brightii}. This changed condition was particularly found in
those who had died of that form of cholera known as the \emph{cholera-typhus}.
The bladder was found contracted and empty.


In addition to the changes here specified others were noted but they
were supposed to be caused by the presence of some other modifying
disease, and hence not attributable to the cholera and not to be
accounted as a pathological result of the epidemic.

Pathologico-chemically, it was found that the blood was relatively
and absolutely poorer, or more deficient in water, having an appearance
resembling mud.

It was also quite deficient in alkalinity, particularly in the \typo{triple}{tripple}
phosphates, and the carbonate of soda. There was also often a deficiency
of the carbonate of ammonia which it is well known has equal
power to influence the coagulability of the blood and the integrity of
the red corpuscles.

In all instances it was found that the cholera blood chemically was
closely allied to putrescent blood, and readily made to undergo the
putrefactive ferment, far more easily than healthy blood.

The evacuations were all found to be rich in water, and in the alkalinity
of which the blood was deficient, particularly the tripple phosphates
and the carbonate of soda, while they contained but a trace of
albumen. Occasionally in the bladder would there be found a little of
the blue coloring matter mixed with chlorides and the earthy phosphates,
while under the microscope could be discerned in the sediment the tuff
cylinders and the epithelium which had been discharged from the lining
of Bellini's small urin-ducts.

The secretions from other parts of the body have not been as carefully
examined as they should be, but thus far have furnished only
negative results.

If now we consider the changes produced in cholera as here described
are not always uniform, or of an equally marked character, but that they
depend upon the force of different influences—that epidemic cholera not
unfrequently occurs with entire absence of vomiting or purging, but
with an extraordinary amount of \typo{perspiratory}{prespiratoy} exudation, or with spams
that speedily cause death—that in spasmodic cholera the anti-spasmodics
are generally found useful—that not a few cholera patients die from
want of what is called reaction, even where there was no appearance of
decomposition of the blood or deprivation of serum in the blood vessels,
we must come to the conclusion that the first impression of the cause of
cholera is sometimes made upon the blood and at other times upon the
nervous system, while in more rare instances it may impress both the
blood and the nerves at the same time.

The question as to why the serum or watery portion of the blood
\oldpage{391}should escape into the stomach and bowels to produce the rice-water
discharges may be answered by referring to the inevitable result of severe
congestion of the lymphatics, as is also shown in the pouring out the
serum upon the surface of the skin in the excessive perspiration which
is sometimes present.

The serum of the blood dissolves the epithelial cells of the alimentary
canal and these dissolved and partially dissolved cells are what gives to
the fluid its peculiar or ricy appearance.

\vspace{\baselineskip}

Is the cholera miasm, or \foreign{sui generis}, independent of the miasms which
produce other epidemic diseases?

Many physicians and natural philosophers have held that the cholera
miasm is but the product of the receding of some other form of disease
or rather a modification of a miasm which had produced some other
form of disease, and they have endeavored to sustain this position by
referring to the fact that an epidemic of cholera is usually preceded by
an epidemic of a different character. Others have considered that it
possesses an individual and independent character, unaltered by changes
and unaffected by climates, everywhere acting upon the alimentary canal
and on which, therefore, it must make its first impression.

Those who entertain this latter view consider the cholera miasm a
peculiar miasm, and call the cholera epidemic \emph{the epidemic of epidemics}
or the producer of epidemics, and the cholera miasm the miasm of
miasms, or the producer of miasms.

As has before been remarked, all miasms which produce epidemic
diseases have somewhat in common, but each also has something peculiar
or specific, and hence while the cholera has many characteristics
manifested in other epidemics, that it has an individuality of character
and an individuality of cause cannot well be denied.

The common characteristics which we observe in epidemics arise
from the fact that all miasms are of telluric and atmospheric origin,
and that all miasms in course of time have their power and influence
modified and changed. Yet they all nevertheless manifest essential
peculiarities of character and produce by a specific process each its own
individual disease. For instance, one miasm will produce scarlet fever,
another measles, and another cholera. If there is none of the specific
miasm there will be no measles, or no cholera, as the case may be.
Neither can one miasm produce another disease, for measles never produced
cholera, or cholera measles, or anything else but cholera.

This is the necessary result of the peculiar and specific character of
each individual miasm which possesses its own specific power and disposition.\endinput
\oldpage{392}``\emph{Quod libet miasma proprium generationes suae typum
in agendo sequitur.}''

In this as in every branch of the natural sciences, the conclusions
adopted may prove so clearly the hypothesis to be correct that it ceases
to be simply a hypothesis but may claim to be classed as an established
scientific truth. As in the natural sciences, so in medicine, the inquirer
after truth must at times adopt a hypothesis for the explanation of the
phenomena which he observes; and in this instance the explanation
which the hypothesis of a cholera miasm gives to the phenomena of
the disease comes near proving that to be the true origin of the
epidemic.

So also the later advances made in the science of chemistry have
nearly proved the cholera miasm to be a reality and not merely a
hypothesis. Dr. Horn of Munich, obtained from the atmosphere
\emph{Ozone}, or a negative electric body, and another body, \emph{Todsomone},
which has been found to combine in the body with carbon and by the
combination to produce effects upon the structures very similar to the
effects produced under similar circumstances by the cholera miasm. I
would not assert that these discoveries prove beyond cavil that the
cholera miasm is Todsomone, but this much is certain that we may feel
sure that observation will establish many practical truths by accepting
this hypothesis, and will also stamp upon it the seal of truth.

\vspace{\baselineskip}

Does the cholera miasm, as many suppose, make its direct impression
upon the stomach and intestines?

The circumstance that the first symptoms of cholera are vomiting and
purging, and other indications of derangement of the alimentary canal
goes to favor the idea that the mucus membrane of the prima vie is the
point at which the reception of the miasm first occurs and from which
it progresses farther into the organism. In opposition to this idea is
the fact that \emph{spasmodic} cholera, as was observed in thousands of cases
in the epidemic of 1831, frequently destroys the patient before vomiting
or purging presents itself; and also the processes of vomiting and
purging removes from the system a large amount of fluid which chemical
researches have proved to be changed blood serum, thus proving
conclusively that the vomiting and the purging are \emph{secondary}, and
sequela to the primary changes which had occurred in the fluids. So
also is shown that the cholera miasm must have impressed several parts
of the system and not alone the alimentary canal. The nervous system,
the blood, the lungs, and the ganglionic system all bear evidence of the\endinput
\oldpage{393}
presence of the cholera miasm, and that it must come in contact with
all these structures.

Neither may we loose sight of the fact that cholera has often been
produced by what is styled the \foreign{contagium psychicum}. It is a well-established
fact that many die through a fear of the disease, and particularly
through the influence of the sight of a cholera case upon an
impressible person. Many doubtless are thus led to suffer from the
epidemic who otherwise would have entirely escaped it.

With regard to which is first acted upon, the blood or the nervous
system, I think the true answer is that in this regard there is a great
diversity in the different cases, but that in many cases both the blood
and the nerves are simultaneously impressed.

\vspace{\baselineskip}

Has the cholera miasm and the Asiatic cholera undergone any alterations
in its original nature and character in its transit? Does it always
present forerunners of epidemics? Has it always also been followed by
other forms of epidemic disease as it has passed away?

The cholera miasm has certainly \emph{not} undergone any change but remains
ever the same in nature and quality as when it started from the
Punjaub as is shown by the unaltered and specific character of the epidemic
in all climes and seasons, without any regard to the state of the
weather, uninfluenced by heat or cold, or dryness or moisture. But no
one will deny that the cholera miasm and consequently the disease
which it produces does loose from time to time apart of its potency and
assume a more mild and manageable form, for the history of its various
epidemics has fully established these facts. But the succeeding epidemic
is found to equal in intensity any former, and the one of the year
1855 in the month of May, was more intense and destructive than any
which had preceded it.

In most instances preceding an epidemic of cholera, other epidemics
have been observed as preceding this, as intermittents, diarrhœa, dysentery.
So also after the epidemic of cholera has passed by, have
epidemic forms of disease appeared, as typhus, influenza, etc. These
observations have led to the opinion that the preceding diseases might
be considered as the forebodings of cholera, and the succeeding as the
sequelae of the disease.

I have already pointed out the common sources from which all
miasms arise and hence the connections of these various forms of epidemic
diseases can be explained without our concluding there is anything
more in common with these miasms than simply a relationship
of origin.

[to be continued.]\endinput
\section*{Observations on the Uses of Sanguinaria
Canadensis.}

by \textsc{Abr'm.\ Livezey, a.\ m., m.\ d.}

\oldpage{394}
In several medical journals I have taken the liberty to call the attention
of the profession to some of the uses of our indigenous medicinal
plants, and in the present communication I beg leave to offer some
remarks upon the medicinal value of the Sanguinaria---a plant incident
to all localities and the root of which is easily gathered.

Without prejudice to the use of any other article I feel warranted in
saying, from no little experience, that this plant with the aid of podophyllin
will exert a more happy influence in all hepatic derangements---both
as a cholagogue purgative and as an alterative---than any combination
of calomel.

Possessing undoubted nauseant, sedative and alterative properties,
blood-root will in cases of slight inflammation of the biliary organs, or
congestive states of the same, or where a species of spasmodic action
pervades those structures, give prompt relief; and where torpidity
exists and the physician thinks that the stimulant action of some mercurial
is indicated he need only combine a minute portion of the
podophyllin to obtain all the advantages that are supposed to be
derived from calomel.

Sanguinaria gives a decided aid to the action of podophyllin or any
other cathartic to which it is added.   It is, in the form of tincture, an
alterative expectorant in chronic bronchitis.   It is valuable in chronic
hepatitis combined with ext.\ taraxicum and ext.\ podophyllum, jalap or
rhei, if obstinate constipation exists.    Tinct.\ Sanguinaria can with
advantage be substituted for wine of antimony in the \emph{brown mixture}
and wherever the wine of antimony is used.   As a substitute for the
compound cathartic pill the following combination---already published
will generally prove more satisfactory:

\begin{center}
\begin{tabbing}
  \textsf{℞}. \= Podophyllin, \= gr. \= i., \\
    \> Leptandrin, \> ''\> iv., \\
    \> Sanguinaria, \> ''\> ii., \\
    \> Ext. Taraxicum, q.\ s.\ Misce.\ ft.\ pil.\ No.\ iv. \\
\end{tabbing}
\end{center}
Two or three for a cathartic; ½ to a whole one night and morning as a
hepatic alterative.

A graduate student of mine, Dr.\ Rice, late resident physician in the
W.\ C.\ Infirmary of Philadelphia, had a case of obstinate constipation
which had persisted four weeks---so said the patient, an Irish woman,\endinput
\oldpage{395}
when she presented herself at the clinic---and in twelve hours time
she had a free alvine evacuation from the use of Sanguinaria, well triturated
with white sugar and given in small doses every two hours.
Dr.\ R.\ is fully persuaded that blood-root is an admirable adjuvant in
all prescriptions for the restoration of healthful function in the liver,
and especially when constipation is coincident.

\textsc{Note}.---Perhaps no indigenous plant has attracted more attention
from those physicians who are accustomed to notice the living specimens
of materia medica as they spring up in the woods and fields
than the one under consideration. The early appearance of its beautiful
and pure blossom, the dark blood-color of its fleshy root, its marked
taste and its prompt action on the system all lead to its obtaining the
attention which has been bestowed upon it.

A trial of its therapeutic virtues has led those who have made use of
it to speak of it in the highest terms of praise and to earnestly recommend
it to the favorable notice of the profession, and yet, strangely, it
has never obtained that prominent position in the list of medicines all
its advocates think it deserves.

Nearly every writer on Botany and Materia Medica in our country
has delighted to give a full description of this plant and to speak highly
in praise of its beauty and usefulness. Among the earlier writers who
have made mention of it Dr.\ Shoepf says that fifteen or twenty grains
of the pulverized root will produce powerful emesis, but that it must not be
given in the form of a powder as thus it is apt to produce great irritation
of the fauces. He prefers a decoction or the pill form. Merat says it
is useful in gonorrhœa. Shoepf also mentioned the value of a weak
decoction of the root in gonorrhœa and refers to the fact that Golden
had found it useful in jaundice. In doses sufficient to produce emesis
it was found to dislodge worms from the stomach. Thatcher, in his
Dispensatory, speaks of the use made of it by Dr.\ Dexter in doses of
one grain of the powder or ten drops of the saturated tincture, as a
stimulant and diaphoretic. Dr.\ Downy was of the opinion that the
dose as recommended by Drs.\ Shoepf and Colden was larger than could
be administered with safety. In speaking of the value of the root in
jaundice Dr.\ Thatcher says it was believed to be the chief ingredient
of the quack medicine known as \emph{Rawson's Bitters}.

The younger Barton thinks that the only form in which the blood-root
should be used is that of a spirituous tincture. In this form he
used it in connection with the tincture of bitter-plants as a tonic with
great satisfaction. He also found it useful as a wash for old indolent
ulcers and sores with hardened edges and an ichorous discharge. He\endinput
\oldpage{396}
had also used the powdered root as an application to fungoid growths
and nasal polypi. Bigelow, and Dr.\ Smith also used it for the same
purpose. So also Dr.\ Shanks and Dr.\ Israel Sterling, according to
Thatcher, used it in place of digitalis in coughs and pneumonic complaints.
Dr.\ Darwin has used it in peripneumonia trachealis in the
form of a decoction and from the benefit thence derived Dr.\ Barton
thought it must be a useful medicine, particularly in cynanche maligna,
in cynanche trachealis and other similar affections.

Drs.\ Barton and Downy said that the \emph{leaves} of the puccoon as well as
the seeds are possessed of a \emph{narcotic} power similar to that of the seeds
of the stramonium and that they had produced dangerous symptoms.

In 1831 Daniel B.\ Smith published in the \booktitle{Journal of the Philadelphia
College of Pharmacy} a dissertation on this plant, in which,
he gives its natural and botanical history and speaks of the experiments
made by Dr.\ Dana on the root in 1824, when the \emph{Sanguinarina} was
probably first obtained.

Dr.\ Tully has carefully examined the medicinal powers of blood-root
and thinks it is therapeutically allied to squills, seneca, digitalis, guaiacum
and ammoniacum.

More recently Dr.\ Williams, formerly of Massachusetts but now of
Illinois, has written several valuable essays on the Sanguinaria, but
unfortunately I have lost the reference to them and I only remember
that he considered it one of the most valuable if not the most valuable
of all the North American plants.

Dr.\ Thom of Ohio, in a communication to the \booktitle{Western Journal},
says that for two years he had been closely engaged in observing the
effects of this remedy in various diseases and he concludes that it is a
\emph{sedative} of no ordinary powers. For reducing the force and frequency
of the pulse without prostrating the system he considered it one of the
most efficient remedies. He also styled it an \emph{alterative} with a marked
influence on the liver and the glandular system generally. He employed
it in hemorrhage from the lungs, particularly in those cases where
the hemorrhage appeared to be caused by vicarious menstruation, and
considered it of more value than any other agent he had used.

Dr.\ M'Bride in the \booktitle{South.\ Jour.\ of Med.} said he considered this
plant eminently serviceable in those disorders of the liver where the
secretion of the bile is either suppressed, deficient or vitiated. In imperfect
convalescence after bilious fever he says, ``the puccoon is the
best remedy.'' As an emmenagogue he thought highly of it. He
recommended it as a substitute for mercury.

Dr.\ J.\ L.\ Mothershead used it in dyspepsia in the form of pills,\endinput

\oldpage{397}giving from one to three grains at a dose three times a day. In
troublesome cough he found it valuable. He also used it satisfactorily
in tinea capitis, tetter and other forms of skin disease in the form of
powder or strong tincture on the affected part. He said: ``Of all
the articles in the Materia Medica, next to mercury and its preparations,
none in my opinion can compare with it in its powers to excite the
action of the liver, and it has the advantage of the former in its capability
of being used at all times and continued without producing any
of its unpleasant results.''

Dr.\ Bard in his Inaugural Dissertation confirmed the statement of
Dr.\ Downey in regard to the narcotic effect of the seeds and speaks
of using the root in croup, pneumonia, whooping-cough, phthisis and
jaundice. Dr.\ J.\ Allen of New York, says it powerfully promotes diaphoresis
in inflammatory rheumatism. Dr.\ Downy says that the leaves
are used in veterinary practice in Maryland for the purpose of facilitating
the shedding of the hair of animals. Dr.\ Griffeths has also given it to
horses for the cure of bots, one or two roots serving to produce a cure.

Dr.\ Branch, of South Carolina, thinks a decoction of the root of
more value than any other single remedy in croup. He denies that it
is possessed of any poisonous properties.

I have not been able to obtain the Inaugural Dissertation of Dr.\ Henry
West, of Belmont Co., Ohio, upon the use and value of this
agent, but evidently he must have placed a high value upon it to make
it the subject of his remarks.

Recently Dr.\ J.\ W.\ Fell has been permitted to make a trial of his
mode of treating cancer on the patients of the Middlesex Hospital of London
and as he had not previously made known the agents he had used
the \booktitle{London Lancet} condemned the secrecy which had governed him, and
finally Dr.\ Fell was led to publish a work on Cancer and its Treatment
in which he said he had used the ``bruised bloody pulp of the white-flowering
puccoon.''

The formula used by Dr.\ Fell differs from the chloride of zinc
paste of Dr.\ Papengurth and Prof. Hancke of Breslau and Dr.\ Canquoine
of Paris, from the addition of the blood-root to the ingredients used by
these surgeons in the treatment of cancer.   The formula is as follows:

\begin{center}
\begin{tabbing}
  \prescription. \= Sanguinaria Canad., \ounce ss, vel ounce j., \\
    \> Zinci Chlorid., \ounce ss, vel \ounce ij., \\
    \> Aqua, f \ounce ij., \\
    \> Tritic. Hybern. Sem. pulv., q. s.
\end{tabbing}
\end{center}
M.\ f.\ paste as thick as treacle and apply to the cancer,

For years this has been a popular remedy for the purpose of destroying\endinput

\oldpage{398}granulations and other morbid growths, and it is more than probable
the blood-root which has been added to various ointments and
applications which have been used upon cancerous affections has done
much toward effecting the cure.

A reference to the use of blood-root in the cure of cancer is now
causing considerable discussion in various sections of the country, particularly
in New England, and where much is being said in regard to
who was the physician who first used it for the cure of cancer. My
own opinion is that it was in use by the people and the \emph{country} physicians
long before we have any record of its being thus applied.

From the very imperfect abstract here given of a few of the articles
that have been published in our periodicals on the use of this root, we are
warranted in drawing the conclusion that it is a very valuable medicine
and should be introduced into more general use. But doubtless one
reason for its neglect is the fact that the root rapidly looses its value by
age and if kept more than one year may become nearly worthless.

The tincture and other preparations should be made from the root as
soon as possible after it is gathered and not from the old and nearly
worthless specimens usually sold by druggists.

In regard to the preparations sold under the names of \emph{Sanguinarin}
and \emph{Sanguinarina}, although I have had frequent letters of inquiry
addressed to me, I cannot give any satisfactory answer. I have no
means of knowing what these preparations are or how manufactured,
and of those who have used them I have never been able to obtain any
evidence of their character or value as therapeutic agents, but a friend
of mine who has manufactured these articles and sold them in considerable
quantities has told me, that as the result of his own observations
and the observations of those of the profession who, had bought and
used them, he was fully convinced they were of even less value than
the pulverized root. I consider it a duty I owe to the readers of the
\textsc{Journal} to present these facts.

If those who manufacture these agents would let us know enough
about them to warrant us in making a trial of them, and if those who
have used them would carefully observe their action and notify us of
the result, soon the readers of the \textsc{Journal} would be in the possession
of the required information. In the present state of the case the only
answer I can give is that I have never used them and know nothing
positive about them. \hfill{}C.\quad\endinput

\section*{Extractum Nicotianiæ Rademacheri.}

by \textsc{Theodore C.\ Miller,\ m.\ d.}

\oldpage{399}\textsc{I herewith} present a notice of an agent which to me is possessed of
extreme value. It is the \emph{extract of the Nicotiania rustica}, as prepared
by the late Dr.\ T.\ G.\ Rademacher.

It is not prepared from the dry but from the fresh and green tobacco
plant. In preparing the extract it is necessary that \emph{immediately} and
without delay, after the leaves have been pulled they must be pressed
so as to force out the juice and that juice evaporated to the consistency
of an extract. When prepared in this manner the extract has none of
the taste of the dried tobacco leaves; but if the leaves are pulled only
a few hours before the juice is expressed, then the extract will have a
taste more or less like that of smoking tobacco, in which case it is not
fit for therapeutical purposes. I have ahvays found it best to have the
leaves pressed at once on being pulled; and I have always prepared it
according to the directions of Rademacher, from the Nicotiania rustica,
and not from the Nicotiania tabacum. Rademacher's extract is one of
the best remedies in genuine cough of the lungs, and for that I can wdth
a clear conscience recommend it to the readers of the \textsc{College Journal}.
It is a remedy for which probably I could not find a substitute.

Rademacher gave it in doses of from one half to two grains, and repeated
it several times a day. It may be made into a pill with the
powdered marsh mallow root. It is a quick and safe remedy in a particular
diseased condition of the lungs for which I am not able to give
a name, but the want of the \emph{name} is no loss to the practical physician,
who must be governed by the nature of the disease and not by its
nomenclature.

That we can, with the extract of the fresh leaves of tobacco, cure an
inveterate genuine lung cough, and thus prevent pulmonary tuberculosis,
in my mind does not admit of a doubt, provided the cough is kept under
the control of the remedy. But there are forms of lung cough which
this extract will not control, and in those cases I would recommend a
trial of the \emph{Stibium Sulphuretum Auranticum}, as mentioned in the
\textsc{College Journal} for June, page 351. Rademacher truly says: ``In
general we must be guided in our minds in the practice of our art, by
the following fact: The diseases are not governed or changed in character
by the ideas and opinions of the physician, but the opinion of
the physician must be governed by the nature of the disease.''

That the extract of tobacco has a powerful controlling influence over\endinput

\oldpage{400}the genuine lung cough, serves as a diagnostic as to the real nature of
the disease. If the cough originates from the lungs it will be benefitted
by the extract, while if it owes its origin to a disease of some
other part of the system, the extract may fail of benefitting the
patient. But there may be coughs which in reality are caused by some
diseases of the lungs, and yet the extract may not prove beneficial.
For instance, a cough may be caused by a node, or from a closed or
an open abscess in the lungs, or from the pressure of a fractured rib
upon the pulmonary tissue and yet the extract would not produce a
cure. The extract has a favorable influence upon idiopathic but not
on secondary coughs. With opium we can often relieve secondary or
sympathetic coughs. We do not with that agent obtain a cure, but
we do obtain relief from the cough, and moderate it or pacify it. The
extract of tobacco is not as active as opium to \emph{allay} a cough, but far
more powerful to cure it when of the genuine lung origin.

\textsc{Idiopathic bleeding of the lungs.} When I speak of bleeding of the
lungs I mean to be understood that form of the disease which is commonly
called \emph{spitting of blood}, where a greater or less quantity of clear
blood, or blood mixed with phlegm, or phlegm streaked with blood,
will be raised from the lungs. The extract is valuable in these cases,
but may not be depended upon in \emph{Pneumorrhagia, or Apoplexia pulmonalis},
in which latter form of disease we must resort to the use of
allum and ice internally and cold wet cloths to the surface of the
chest, and to other appropriate remedial measures.

I would here remark that this preparation will not produce the vomiting
and purging which follows the administration of the dry tobacco,
and I have never used the dry tobacco as an emetic or an injection, as I
find the Lobelia inflata an equally efficient remedy.

I was called a few days since to see a patient where many other remedies
had been tried by three eminent physicians who had attended on
the case, without avail. The patient had been sick quite a length of
time but owing to my recent illness and the distance from me I could
not treat it. The case presented the characteristics of consumption, a
harrassing cough, with bloody sputa, etc. As I was unable to visit the
patient I was consulted by letter, and had ordered inhalations, and directed
the Wild Cherry, Lycopus Virginicus, and Lobelia combined
with Ipecacuanha, without benefit. I used the Lobelia, from having
found it of great value in cramps and affections of the chest, and
particularly in phthisis pulmonalis. For these purposes, and to relieve
the dry harrassing cough and tickling of the throat, it is in use
by many German physicians.\endinput

<span class=font1>1857.]&nbsp;<i>Extraclum Niootianiœ Rademacheri.&nbsp;</i>* 401</span>

<span class=font1>** As I was at the time out of the extract of tobacco I made a trial of
the Lobelia, but I obtained some from my brother and about two
weeks since I commenced its use. In six days the cough and the expec-
toration entirely ceased. I have since visited the patient and although
the symptoms are so much relieved, auscultation does not promise much
for the final recovery &lt;?f the patient. Too many persons had prescribed,
and the lungs are too much diseased to allow much hopes of a permanent
cure; but this case illustrates the power of the agent.</span>

<span class=font1>I am the more urgent to induce the profession to make a trial of
this extract, as I think it is nearly or quite unknown to the physicians
in this country.</span>

<span class=font0><b>aqua  nicotians tabacum  sperituos^] radamacheei.</b></span>

<span class=font1>This preparation is recommended highly in affections of the brain
accompanying fever, in <i>rheumatismus acutus fixus at vagus, </i>in other
affections of the brain and spinal marrow, in cholera morbus, and in
cholera Asiatica.</span>

<span class=font1>To prepare it: Take of choice fresh green leaves of Nicotianse ta-
bacum eight pounds, and cut them finely. Add of the best alcohol,
by weight one and a half pounds, of distilled water as much as is
necessary to distill over eight pounds (by weight) of the water.</span>

<span class=font1>The leaves are to be cut and the distillation effected immediately
after they are pulled, with great care that there shall be no over-heating
of the liquid, as, if the liquor be over heated it will have a very dis-
agreeable odor of tobacco, which it does not have when the water is
properly prepared.</span>

<span class=font1>Eademacher uses this water in every stage of the Asiatic cholera.!
In the earlier stages he gave the following:</span>

<span class=font1>Aqua Purse, f 3 vij.,</span>

<span class=font1>Soda Acet., 3 jss.,</span>

<span class=font1>Aqua Nicotian., f 3j.,</span>

<span class=font1>Gumi Arab., 5ss.
M.   Dose, one table-spoonful every hour.</span>

<span class=font1>The great majority of cases treated with this mixture recovered
immediately from the attack. In those cases where the attack was fol-
lowed with a typhoid condition, he gave:</span>

<span class=font1>Tinct. Ferri Acetici, f 3 j., ''.[</span>

<span class=font1>Aqua Nicotian., f 3j.,</span>

<span class=font1>Aqua Puree, f 3vj.,</span>

<span class=font1>Gumi Arabici, 3.
<b>M.   </b>Dose, one tea-spoonful every hour.</span>

<span class=font1><b>vol</b>. ii. <b>no. </b>9.-26.&nbsp;v&nbsp;- &quot;<sup>J</sup></span>\endinput

\oldpage{402}

With this treatment the patients all recovered after a longer or shorter
period.

\textsc{Note.---}The formula for preparing the acetic tincture of iron is to be
found on page 351 of the \textsc{College Journal}.

\chapter[Pleuritis, Latant][Pleuritis, Latant]{Pleuritis, Latant.}

by \textsc{C.\ E.\ Witham,\ m.\ d.}

\textsc{Pleuritis} is a disease which often presents obscure, important and interesting
complications, taxing the utmost skill of the experienced physician
in tracing the precise bearing and extent of the morbid action
established.

The heart, lungs, bronchia and liver are often implicated in this disease.
Asthenic pneumonia, and chronic and latent pleuritis have many common
symptoms. It is stated that pleuritis is more prone to produce
tubercular disease than pneumonia is, and it is thought by some authors
that the absorption of pus into the blood may explain this rather singular
fact. In the treatment of disease our object should be to remove
morbid action by the most simple and effectual treatment the case will
admit of. If the following report should be the means of stimulating
the young practitioner to a more thorough study of thoracic diseases I
shall be amply rewarded.

On the 11th of June, 1856, F.\ W., a lad 14 years of age was presented
for my advice. He was of a sanguine temperament, and a twin brother.
I had never seen him before, but from his father gained the following
history of his case. Five months previous to calling upon me he suddenly
lost the power of speech; did not know that he had previously
suffered from cold or exposure. The loss of speech was the first symptom
of disease he could recollect and this was not preceded by any
very marked indications of hoarseness. A low, hoarse and painful
whisper was the result of all his efforts at conversation. This condition
continued for one month when to his surprise and great joy he found
himself complete master of his vocal organs and congratulated himself
on so strange and unexpected a recovery. He said that while making
some slight exertion he felt something give away in his chest and immediately
he could talk as well as ever. At the end of one week he
was again deprived of speech in the same unexpected and sudden manner.
His physician after inspecting his throat, but making no other
examination, prescribed a gargle of ``pepper tea'' saying it would soon\endinput

\oldpage{403}effect a cure; but after a trial of several weeks this prescription was
discarded, as no change had resulted. Lancinating pain would occasionally
be felt in the chest, slight cough, expectoration streaked slightly
with blood. He continued to perform light work and had not been
confined to his bed. I found him presenting the following symptoms
five months after the first appearance of the disease.

The mucous membrane of the pharynx presented a pale and debilitated
appearance; the chest inclined forward, the body assuming a
stooping position; great tenderness of the spine from the first cervical
vertebra to the last dorsal; pressure over the lungs, liver, stomach
and spleen gave pain. In short no part of the chest nor abdomen
could be percussed without revealing deep-seated tenderness. The
skin was dry, pulse quick; there was much dyspnœa with abdominal
respiration. Percussion of the lungs gave rather a dull sound. Bowels
torpid. I diagnosed the disease to be Latent Pleuritis complicated
with chronic inflammation of the larynx which gave rise to the Aphonia.
As the patient was of a strumous diathesis and the disease of long standing
I doubted the efficacy of treatment but advised it and took charge
of the case on the 12th of June.

I first ordered morning bathing to be practiced daily, the water used
to be impregnated with chloride of sodium and bicarbonate of potassa.
Internal treatment:

\begin{center}
\begin{tabbing}
  \prescription. \= Podophyllin, \dram{}ss., \\
    \> Capsicum, gr. X., \\
    \> Ext. Taraxicum, q s.
\end{tabbing}
\end{center}
M.\ f.\ Pill, No. X.\quad{}Take one of these pills morning, noon and night
until the bowels are freely moved, then take but two a day.
\begin{center}
\begin{tabbing}
  \prescription. \= Comp.\ Syr.\ Stillingia, f\ounce{} iij., \\
    \> Capsicum, gr. X., \\
    \> Iodide of Potassium, \ounce{} j.
\end{tabbing}
\end{center}
M.\quad{}Take one teaspoonful four times a day. To test the progress of
the case I saw the patient daily. I discovered no change until the third
day; the bowels were then active, less tenderness about the cervical
vertebra, could whisper with less pain and more distinctly. On the
fourth day still more improved. I ordered the same treatment continued
and on the next day the patient recovered full power of speech and could
talk freely and without pain. He continued to improve and on the tenth
day of treatment I could discover no abnormal symptoms. Percussion
over the abdominal and thoracic viscera was no longer painful; no tenderness
of the spine could be detected. I now discontinued the former
treatment excepting one pill to be taken each day, and prescribed the
following:\endinput

\oldpage{404}

\begin{center}
\begin{tabbing}
  \prescription. \= Prussiate of Iron,  \\
    \> Hydrastin, āā \dram{} ss.
\end{tabbing}
\end{center}
Mix.\quad{}Make 15 powders, two to be taken a day. The patient felt
well and returned home, and now nearly one year has remained well as
usual.


\section*{What Influence has the Moon Upon Disease?}

by \textsc{Comely Jessup,\ m.\ d.}

\textsc{I wish} to ask your opinion and procure, if possible, the result of the
observations of your readers, relating to the influence (if such influence
exist), exerted by the moon upon disease. I have been of the number
who look upon the lunar influences except such as may be attributed to
the known laws of gravitation as entirely fabulous, but several instances
occurring within the sphere of my observation, which have indicated
the existence of some hidden agency, a few of which have been distinctly
marked, have awakened a desire to see the matter thoroughly
investigated and the truth or falsity of lunar influence fairly demonstrated.
The following are a few of the more marked instances of apparent lunar
periodicity which have fallen under my observation.

\textsc{Case 1.} Mr.\ C., aged perhaps 45, has been subject to epilepsy for
the last three years. About the time of the change and full of the
moon he will have from three or four to eight or ten convulsions. At
other times he is free from them, except occasionally about the time of
the first and last quarter.

\textsc{Case 2.} T.\ I., aged 72, was attacked some four years since with malignant
erysipelas, accompanied at first with paralytic symptoms,
which, together with a severe attack of ``Doctors,''---though he survived
them all---left him in a condition from which he has never recovered
and never will. The most prominent features in his case now
are pain in the back and head, which is remittent in its character,
being most severe in the early part of the day; nervousness, constant
trembling of the hands, or rather the peculiar shaking characteristic of
paralysis, to the extent that he can with difficulty feed himself; and
occasional attacks of general weakness and disposition to syncope.
These \typo{symptoms}{symptems} are all much aggravated at the time of the moon's
changes.

\textsc{Case 3}. A.\ E.\ S., aged 5; troubled with ascaris vermicularis at
the time of the new and full moon, which were during the intervening\endinput

\oldpage{405}space of time quiescent. This case would not have excited suspicion,
inasmuch as there seems to be frequently a periodicity in their
actions, but taken in connection with other cases it is a straw which
indicates the quarter from which the wind blows.

Now the question is, does the moon during its various phases exert
various influences which though unperceived by the robust constitution
of perfect health, make themselves felt to the sensitive system
of the invalid, or are these merely striking coincidences? These are
questions of interest to the Physiologist and medical Philosopher,
merely as significant facts, but doubly so to the practitioner to whom
a knowledge of every influence brought to bear upon those under his
charge is essential.

With a hope that others may be induced to make known the result
of their observations, I report these cases.

\chapter[Human Blood a Styptic][Human Blood a Styptic]{Human Blood a Styptic(?)}

by \textsc{O.\ Van Buskirk,\ m.\ d.}

\textsc{I wish} to communicate a few thoughts upon a case which came under
my observation a short time ago, in which I employed human blood as
a styptic with the most gratifying result. To you this may be no new
thing, but to me it is, and it may be to many other junior members of
the profession. From this consideration I thought I would write you
a brief account of the case and the manner in which I employed it.

The case was a lady from whom I extracted a tooth (the first molar),
and it was rather difficult to draw, but it came out whole and without
doing any perceptible damage to the jaw. The hemorrhage was not
very profuse at the time; not more than usual. When she left my office
she seemed as well as usual and continued so for two days, at which
time a profuse hemorrhage took place from the cavity in her jaw. By
means of a decoction of black-oak bark she checked it for about twenty-four
hours when it began again worse than before. I was then sent for
and found her quite weak and sick at her stomach. I applied geranin,
tannic acid, etc., all to no effect. I then took about two ounces of blood,
placed it over the fire and as soon as it came to the boiling point the
solid constituents of the blood coagulated, and left the aqueous portion
clear and limpid. I then poured off the water and left the other over
a slow fire until it assumed a thick, jelly-like form. I took a small lump
of this and filled the cavity and placed over it a small wad of cotton
wadding and directed her to close her jaws so as to keep the remedy in\endinput

<span class=font3><b>40G</b></span>

<span class=font3><i>Notes and Observations, by </i><b>D</b><span class=font2><b>r. </b></span><b>M</b><span class=font2><b>iller</b></span>. [September,</span>

<span class=font3>its place, and to my great delight it stopped the hemorrhage almost
instantly. The remainder of the blood which was in the vessel I took
through the same process and reduced it to a fine powder for future use.</span>

<span class=font3><b>R</b><span class=font2><b>emarks.</b></span><span class=font0>—</span>The use of blood as a remedial agent is not new to the
profession as a reference to the <b>J</b><span class=font2><b>ournal</b></span>, page 299 will show, where
Dr. Miller refers to the use made of it by Mauthner in ansemia, as
referred to by Dr. Davis, of Illinois, in the Transactions of the Illinois
State Medical Society for 1852, who says it may be given when inspis-
sated in doses of from ten to sixty grains at a dose, or dissolved in
water. The therapeutic use of blood is also referred to in the <i>Am.
Med. Journal </i>for 1853 and perhaps in other periodicals. But none of
these refer to the <i>styptic </i>properties it is supposed to possess by Dr.
Yan Buskirk, and which one single experiment neither proves or dis-
proves. The Pencil Savans are accustomed to deny nothing until it is
thoroughly disproved, and to admit nothing until it is thoroughly estab-
lished, but rather to <i>receive </i>the opinions of others and await farther
and full proof before their final disposition, and in this regard we may
do well to follow their example. <b>C.</b></span>

<span class=font3>NOTES AND OBSERVATIONS.</span>

<span class=font2><b>by t. c. miller, m. d.</b></span>

<span class=font3><b>A</b><span class=font2><b>mmonia </b></span><b>Y</b><span class=font2><b>alerianicum [ </b></span><i>Valerianate of Ammonia']. </i>This is formed
by the saturation of the Valerianic acid with the carbonate of ammonia.
It is usually a fluid, although an imperfect crystallization has been ob-
tained of the salt. In warm weather even the crystals are apt to de-
liquesce into a syrupy fluid, having a strong valerianic odor, and a
slight odor of ammonia.</span>

<span class=font3><b>O</b><span class=font2><b>ttinger </b></span>of Munich, Germany, has highly recommended this prepar-
ation in Asiatic Cholera in the following form.</span>

<span class=font1><i>V&lt;.   </i><span class=font3>Ammonia Valerianici, </span><i>B </i><span class=font3>j.,
Aqua Destillat., f'Siij.,
Syrup. Sacch., f&quot;3 ss.</span></span>

<span class=font3>M.   Dose, one tablespoonful once in from 15 to 30 minutes.</span>

<span class=font3><b>O</b><span class=font2><b>ttinger </b></span>used this mixture to the exclusion of every other internal
remedy, and after the severity of the attack had passed and reaction
was established, he gave but from four to six doses daily. He also
ordered ice to be rubbed over the abdomen externally, occasionally
changing the cold w<sup>r</sup>ater for hot or placing the patient in a hot bath in</span>

\textbf{[TRANSCRIPTION IS INCOMPLETE]}

\end{document}