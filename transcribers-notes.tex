\chapter*[Transcriber's Note]{TRANSCRIBER'S NOTE}
\lettrine[lines=3]{T}{his document} is an early-stage experiment in retypesetting early documents. Not for its own sake, but because I hope that by working through this---and a few other works from the 19\textsuperscript{th} Century---by hand, I can develop a suite of technical tools to support the \textsc{re-release} of digitized books and journals.

By \textsc{re-release} I don't mean ``POD re-publishing,'' and I don't mean piling them onto the massive searchable archives of poorly-OCRed page scans that Google has built, and I don't mean scraping them into falsely ``canonical'' ASCII transcriptions that Project Gutenberg curates. Those are helpful tools in their way, but you have to admit they don't \emph{give you the book to use}: a POD gives  you a new physical copy of a book; Google helps you find pictures of the pages, stripped by error-prone OCR of their words; Project Gutenberg gives you the words torn out of their physical and cultural context.

\textsc{``Re-release''} needs to mean something closer \emph{setting a work free again}. \emph{Unlocking} it. Lifting the words \emph{gently}  off the page, so you can set them back down again as you see fit.

\fancybreak{\ding{167}}

In this earliest experiment, I've scanned and OCRed my copy of the journal's pages, and written a few scripts to help convert the OCRed text from HTML to \LaTeX, then hand-edited each page (a bit) to correct some of the most obvious OCR errors. There are some simple styles in the backbone file \textsf{work.tex}, and I've stored the entire experiment from its beginning a nice little \href{http://github.com/Vaguery/collegeJournalOfMedicalScienceSeptember1857}{github repository} so that you can fork it and change it and\ldots \emph{have it}.

Whether it's ``in progress,'' or I think I'm done. Whether you want to print it, or want to fix some stupid decision I've made. \href{http://git-scm.com/}{Distributed version control} has recorded all my decisions for you, and lets you see them and unwind them. It lets you start from any step along my history and hare off in whatever direction you prefer. Or, if you're content, it lets you watch my progress, and comment, and improve, and give back.

Thanks to \href{http://abbyy.com}{ABBYY's FineReader} software, I've been able to avoid re-typing most of this. Thanks to my experience with \href{http://pgdp.net}{Distributed Proofreaders}, I've some understanding of how to improve digitized works. Thanks to the copy of \href{http://scripts.sil.org/cms/scripts/page.php?site_id=nrsi&id=xetex}{\XeLaTeX} included in \href{http://www.tug.org/texlive/}{TeX Live}, and the amazing \href{http://www.tex.ac.uk/CTAN/macros/latex2e/contrib/memoir/}{\textsf{memoir} package}, I was able to re-typeset something like this 50-page journal almost trivially. Thanks to \href{http://shinntype.com/}{Nick Shinn's amazing effort} re-creating the Scotch Roman family, I've even had the pleasure of seeing it arise \emph{as if reborn} from the original's foxed, rumpled, disbound  pages.

Have a look at \href{http://github.com/Vaguery/collegeJournalOfMedicalScienceSeptember1857}{the github project} for more information, and to fork and work on the project yourself.

This book isn't ``done''. Make it more like what you think it should be.

\hfill{}---\textsc{Bill Tozier}, \today 
