\oldpage{393}
presence of the cholera miasm, and that it must come in contact with
all these structures.

Neither may we loose sight of the fact that cholera has often been
produced by what is styled the \emph{contagium psychicum}. It is a well-established
fact that many die through a fear of the disease, and particularly
through the influence of the sight of a cholera case upon an
impressible person. Many doubtless are thus led to suffer from the
epidemic who otherwise would have entirely escaped it.

With regard to which is first acted upon, the blood or the nervous
system, I think the true answer is that in this regard there is a great
diversity in the different cases, but that in many cases both the blood
and the nerves are simultaneously impressed.

\vspace{\baselineskip}

Has the cholera miasm and the Asiatic cholera undergone any alterations
in its original nature and character in its transit? Does it always
present forerunners of epidemics? Has it always also been followed by
other forms of epidemic disease as it has passed away?

The cholera miasm has certainly \emph{not} undergone any change but remains
ever the same in nature and quality as when it started from the
Punjaub as is shown by the unaltered and specific character of the epidemic
in all climes and seasons, without any regard to the state of the
weather, uninfluenced by heat or cold, or dryness or moisture. But no
one will deny that the cholera miasm and consequently the disease
which it produces does loose from time to time apart of its potency and
assume a more mild and manageable form, for the history of its various
epidemics has fully established these facts. But the succeeding epidemic
is found to equal in intensity any former, and the one of the year
1855 in the month of May, was more intense and destructive than any
which had preceded it.

In most instances preceding an epidemic of cholera, other epidemics
have been observed as preceding this, as intermittents, diarrhœa, dysentery.
So also after the epidemic of cholera has passed by, have
epidemic forms of disease appeared, as typhus, influenza, etc. These
observations have led to the opinion that the preceding diseases might
be considered as the forebodings of cholera, and the succeeding as the
sequelae of the disease.

I have already pointed out the common sources from which all
miasms arise and hence the connections of these various forms of epidemic
diseases can be explained without our concluding there is anything
more in common with these miasms than simply a relationship
of origin.

[to be continued.]\endinput