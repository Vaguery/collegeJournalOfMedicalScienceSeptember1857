<FONT?><b>1857.]</b>

<FONT?><i>The Eclectic College of Medicine.</i>

<FONT?><b>421</b>

<FONT?>As we have already said, we have no desire to dictate to our medi-
cal brethren, but having been asked for our views on this subject we
have, as in duty bound, endeavored to give a succinct statement of them.
We will close this article by saying that we sincerely hope that Eclectic
physicians everywhere, whether members of associations or not, will
take and maintain high ground in opposition to every form of impos-
ture in medicine, and demonstrate to the world that the Ethics as well
as the medication of our branch of the profession is an improvement
upon that of the old school party.

<FONT?><b>the eclectic college of medicine,
It </b>is with pleasure that we announce to the friends and patrons of this
College that the Trustees have added to its former advantages the halls,
fixtures and furniture of the American Medical College, which recently
occupied part of the same edifice. They have also secured the services
of a full Faculty who will reside in the city. Whether Prof. Buchanan,
Emer. Prof, of Cerebral Physiology and Institutes of Medicine, who
resides in Louisville, Ky., can spend any time with us during the session
or not, we cannot say, though we hope he will be able to do so. The
Trustees and Faculty have, however, deemed it due to him who has so
long and efficiently labored for our cause and whose sympathies are still
with us, to retain his name in an honorable position in the Announce-
ment of the College.

<FONT?><b>dr. weedon's apparatus for fractured clavicle.</b>

<FONT?><b>D</b><FONT?><b>r. </b><b>F</b><FONT?><b>rank </b>H. <b>H</b><FONT?><b>amilton</b>, and others, are inclined to doubt if Dr.
Wheedon, of Albany, is the inventor of the apparatus described by him,
and which was mentioned in the <b>J</b><FONT?><b>ournal </b>for August, p. <b>369.</b>

<FONT?>One writes that the same apparatus has been in use some years, and
is sold in New York under the name of " Bush's clavicular apparatus,"
but that Dr. Wheedon has made an improvement by constructing the
rods of two parts---extensible---so as to fit persons of different heights.

<FONT?>Dr. Hamilton says : " The use of a <b>T </b>splint, as a dressing for a broken
clavicle, is certainly as old as the days of Heister, who, in his great w<sup>r</sup>ork
entitled <i>Institutiones Chirurgicœ, </i>published at Amsterdam in <b>1839, </b>has
given a description and an engraving of this apparatus as it was then
used by himself."

<FONT?>In the <i>Transactions of the American Med. Association </i>for <b>1855,
</b>p. <b>407, </b>Dr. Hamilton reported a case treated by himself with an appa-\endinput
