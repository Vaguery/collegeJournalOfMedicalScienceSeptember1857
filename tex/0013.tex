<span class=font1>1857.]&nbsp;<i>Observations on Sanguinaria Canadensis. </i>397</span>

<span class=font1>giving from one to three grains at a dose three times a day. In
troublesome cough he found it valuable. He also used it satisfactorily
in tinea capitis, tetter and other forms of skin disease in the form of
■ powder or strong tincture on the affected part. He said: &quot;Of all
the articles in the Materia Medica, next to mercury and its preparations,
none in my opinion can compare with it in its powers to excite the
action of the liver, and it has the advantage of the former in its capa-
bility of being used at all times and continued without producing any
of its unpleasant results.&quot;</span>

<span class=font1>Dr. Bard in his Inaugural Dissertation confirmed the statement of
Dr. Downey in regard to the narcotic effect of the seeds and speaks
of using the root in croup, pneumonia, whooping-cough, phthisis and
jaundice. Dr. J. Allen of New York, says it powerfully promotes dia-
phoresis in inflammatory rheumatism. Dr. Downy says that the leaves
are used in veterinary practice in Maryland for the purpose of facilitating
the shedding of the hair of animals. Dr. Griffeths has also given it to
horses for the cure of bots, one or two roots serving to produce a cure.</span>

<span class=font1>Dr. Branch, of South Carolina, thinks a decoction of the root of
more value than any other single remedy in croup. He denies that it
is possessed of any poisonous properties.</span>

<span class=font1>I have not been able to obtain the Inaugural Dissertation of Dr.
Henry West, of Belmont Co., Ohio, upon the use and value of this
agent, but evidently he must have placed a high value upon it to make
it the subject of his remarks.</span>

<span class=font1>Recently Dr. J. W. Fell has been permitted to make a trial of his
mode of treating cancer on the patients of the Middlesex Hospital of Lon-
don and as he had not previously made known the agents he had used
the <i>London Lancet </i>condemned the secrecy which had governed him, and
finally Dr. Fell was led to publish a work on Cancer and its Treatment
<i>I   </i>in which he said he had used the &quot; bruised bloody pulp of the white-
flowering puccoon.&quot;
I      The formula used by Dr. Fell differs from the chloride of zinc
paste of Dr. Papengurth and Prof. Hancke of Breslau and Dr. Canquoine
of Paris, from the addition of the blood-root to the ingredients used by
these surgeons in the treatment of cancer.   The formula is as follows :
Sanguinaria Canad., Sss, velgj.,
Zinci Chlorid., 3ss, velgij.,
Aqua, f3 ij.,</span>

<span class=font1>Tritic. Hybern. Sem. pulv., q. s.
<b>M. </b>f. paste as thick as treacle and apply to the cancer,
<span class=font0><b>j-      </b></span>For years this has been a popular remedy for the purpose of destroy-</span>\endinput
