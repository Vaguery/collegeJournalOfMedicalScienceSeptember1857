<FONT?><b>1857.]</b>

<FONT?><i>Notes and Observations, by </i><b>D</b><FONT?><b>r. </b><b>M</b><FONT?><b>iller.</b>

<FONT?>409

<FONT?>œdematoTis, acute, almost sudden swelling of the leg, and the one half
of the external genitals and the glands in the groins. It does not
affect both legs at the same time.

<FONT?>Many consider the swelling to be the result of a metastasis of the
milk, and hence the popular name; others think it a disease of the
lymphatics originally, and others think the veins to be the original
seat of the disease.

<FONT?>But the majority now hold it has its origin through and in conse-
quence of the inflammation of the crural vein, and resulting in the
obliteration of the same. I agree with <b>L</b><FONT?><b>ebert </b>that it is caused by a
checking of the venous circulation from an obstruction in the veins.
We find the disease mostly in those women who have suffered great
loss of blood by venesection or hemorrhage and in those who take cold
during the confinement.

<FONT?>The prognosis has always been very favorable under the treatment
which I adopt.

<FONT?>I always enjoin quiet, and wrap up the affected leg in roasted meal
and afterwards in oil-cloth. I give only the mildest salts, as Bochelle
salts for the purpose of evacuating the bowels, or use injections for
the same purpose, and give effervescing powders. As soon as the
febrile excitement has passed or is diminished I allow an easily digested
and nourishing diet, with Tonics, and the sub-carbonate of Iron. As
soon as the swelling becomes œdematous I consider the disease on the
decline.

<FONT?>Dr. ScnRiMER, from his experience, as well as many others, was led
to the opinion that a careful tonic and sustaining treatment is the best
in Phlegmasia dolens alba and soon results in a cure. He states that
in an obstinate and severe case which followed a severe, tedious labor,
with extensive loss of blood, that, under what is styled the <i>antiphlo-
gistic, </i>or depleting treatment, the disease grew continuously worse, but
so soon as mild fomentation was applied, and a solution of Iodide of
Potassium and iron, given internally, the cure was speedily effected.

<FONT?><b>[N</b><FONT?><b>ote</b>. A somewhat different opinion, and the reasons therefore as
regards the nature and origin of this disease, may be found in Prof.
King's <i>American Obstetrics. </i>C]

<FONT?><b>S</b><FONT?><b>ummer </b><b>C</b><FONT?><b>omplaint</b>. In <b>27 </b>cases I have speedily arrested the dis-
ease in from 4 to <b>12 </b>hours, using in some cases the <i>Compound powder
of Rhubarb, </i>and in other cases I have used only the <i>Nitrate of Bis-
muth </i>combined with a little Bhubarb.\endinput
