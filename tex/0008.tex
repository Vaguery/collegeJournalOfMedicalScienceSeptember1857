\oldpage{392}``\foreign{Quod libet miasma proprium generationes suae typum
in agendo sequitur.}''

In this as in every branch of the natural sciences, the conclusions
adopted may prove so clearly the hypothesis to be correct that it ceases
to be simply a hypothesis but may claim to be classed as an established
scientific truth. As in the natural sciences, so in medicine, the inquirer
after truth must at times adopt a hypothesis for the explanation of the
phenomena which he observes; and in this instance the explanation
which the hypothesis of a cholera miasm gives to the phenomena of
the disease comes near proving that to be the true origin of the
epidemic.

So also the later advances made in the science of chemistry have
nearly proved the cholera miasm to be a reality and not merely a
hypothesis. Dr.~Horn of Munich, obtained from the atmosphere
\emph{Ozone}, or a negative electric body, and another body, \emph{Todsomone},
which has been found to combine in the body with carbon and by the
combination to produce effects upon the structures very similar to the
effects produced under similar circumstances by the cholera miasm. I
would not assert that these discoveries prove beyond cavil that the
cholera miasm is Todsomone, but this much is certain that we may feel
sure that observation will establish many practical truths by accepting
this hypothesis, and will also stamp upon it the seal of truth.

\vspace{\baselineskip}

Does the cholera miasm, as many suppose, make its direct impression
upon the stomach and intestines?

The circumstance that the first symptoms of cholera are vomiting and
purging, and other indications of derangement of the alimentary canal
goes to favor the idea that the mucus membrane of the prima vie is the
point at which the reception of the miasm first occurs and from which
it progresses farther into the organism. In opposition to this idea is
the fact that \emph{spasmodic} cholera, as was observed in thousands of cases
in the epidemic of 1831, frequently destroys the patient before vomiting
or purging presents itself; and also the processes of vomiting and
purging removes from the system a large amount of fluid which chemical
researches have proved to be changed blood serum, thus proving
conclusively that the vomiting and the purging are \emph{secondary}, and
sequela to the primary changes which had occurred in the fluids. So
also is shown that the cholera miasm must have impressed several parts
of the system and not alone the alimentary canal. The nervous system,
the blood, the lungs, and the ganglionic system all bear evidence of the\endinput