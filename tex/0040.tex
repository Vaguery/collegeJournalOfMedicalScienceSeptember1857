<span class=font1><i>^|&nbsp;Booh Notices. </i>[September,</span>

<span class=font1>maxim of the Homœopathists, <i>&quot;similia similibus curantur&quot; </i>and the
Allopathic one of <i>&quot;contraria contrariis curantur&quot; </i>of Hippocrates
and his followers, and adopts the Allopathic doctrine as having its
foundation in reason, embodying the plain, practical, logical view <i>of
</i>the subject, and being sustained by the experience of a vast majority of
the most scientific men in every country.</span>

<span class=font1>Dr. <b>W</b><span class=font0><b>arren </b></span>says: &quot; The causes of phthisis may be properly divided
into two classes :   1. General causes.   2. Special causes.&quot;</span>

<span class=font1>Among the most prominent of the general causes he names <i>heredi-
tary predisposition, </i>and considers the fact that it is an hereditary
affection <i>as prima facie </i>evidence of its <i>nervous </i>origin. He next con-
siders the influence of improper aliments, the influence of impressions
made on the skin, and lastly, those impressions on the nerves connected
with the <i>emotions. </i>Of these latter he enumerates <sup>tl</sup> the gratification
of lust, indulgence in onanism, depression of spirits, violent grief, and
indeed all passions whereby immediate depression or subsequent reac-
tion is induced ;&quot; quotes from Lombard, Moreton, Laennec, Hippocrates,
Dupay, Amestoy, Wood and Williams, in support of this proposition.</span>

<span class=font1>Among the <i>special causes </i>of the disease he names various callings,
improper clothing, suppression of habitual discharges, and various dis-
eases which tend to direct an unusual amount of blood upon the pul-
monary tissues.</span>

<span class=font1>In his second chapter Dr. <b>W</b><span class=font0><b>arren </b></span>endeavors to prove that there is
an <i>antagonism </i>between the development of tubercle and the state of
pregnancy, and to do this he <i>assumes </i>that in pregnancy there is a dis-
position to the establishment of <i>inflammatory action, </i>which is so immi-
nent as to demand &quot; the production of certain methods of relief to the
economy, whereby its normal condition may be secured and retained,&quot;
and enumerates loss of blood, nausea, vomiting, disgust for food, etc., as
the means required for &quot;the perfection of nature's most important work.&quot;</span>

<span class=font1>The final conclusion of the author is so clearly presented in the clos-
ing paragraph of his Essay that we quote it entire:</span>

<span class=font1><sup>u</sup> I have thus attempted, by arguments, facts and authorities, to
prove that pregnancy prevents the progress of phthisis, even when that
disease is perfectly developed. Whether this effort has been successful
or not, must be left to the judgement of my readers ; and to them I
- confide my cause, with the full assurance not only that their decision
will be equitable in regard to all that has been urged in support of my
position, but that they will agree with me in the conclusion that if
pregnancy can arrest the progress of consumption when fully established,
then for a still stronger reason it must &quot;retard the development of
tubercles in those predisposed to phthisis.&quot; <sub>s</sub></span>
