\oldpage{428}

\fancybreak{* * *}
\footnotesize
\booktitle{The Five Gateways to Knowledge}; by \textsc{George Wilson}, M.~D, F.~R.~S.~E., etc., etc.,
Philadelphia, \textsc{Parry} and \textsc{McMillan}. 1857.
\plainbreak{1}
\normalsize

\lettrine[lines=1]{}{This} little volume receives a hearty welcome; it is one of those rare
works we occasionally meet with in which the useful is combined
with the ornamental, and we find something in each page which will
amuse and instruct.

The style is simple and clear, and at the same time beautiful and
comprehensive. The dull monotony that too often characterizes such
works is not observable here, but each page is enriched with original
thought, clothed in language which the child may understand, and the
sage may read with pleasure.

The Author imagines himself standing sentinel to the city of the Soul
described by \textsc{John Bunyan}, and as he watched the five gates known
as the ``Eye-gate, Ear-gate, Nose-gate, Mouth-gate, and Feel-gate,'' he
records all who bear tidings to the dwellers within.

Seldom have we seen so many of the truths of philosophy enriched
by the beauties of poetry. In his description of the Ear the following
occurs which we give as characteristic of the work.

``But music is never more delightful than when listened to in utter
darkness without obtrusion of music paper, or instrument, or performers,
and whilst we forget that we have ears, and are content to be living
souls floating in a sea of melodious sound.''

We cannot recommend the work more highly than to advise our readers
to obtain and read it for themselves.\hfill{}J.\quad{}

\fancybreak{* * *}

\booktitle{The American Family Physician}, or Domestic Guide to Health. For the use of Physicians,
Families, Plantations, Ships, Travelers, etc. By \textsc{John King}, M.~D., Professor of
Obstetrics and Diseases of Women and Children, in the Eclectic College of Medicine;
formerly Professor of Materia Medica, Therapeutics, and Medical Jurisprudence in the
Memphis Institute; author of the American Eclectic Dispensatory, the American
Eclectic Obstetrics, etc., etc. Cincinnati, \textsc{Longley Brothers}, Publishers, 1857.
\plainbreak{1}
\normalsize

\lettrine[lines=1]{}{The} above title-page, and several sheets of the above work have
been presented us by the Publishers, with the intimation that it would
be ready for sale in about a month.

When the work is published we shall express our opinion upon its
contents and character; but for the present would simply refer to it
knowing that most of our readers are so well acquainted with the former
works of the Author as not to require any commendation of the one
forthcoming.

\fancybreak{* * *}

\booktitle{The American Journal of Dental Science} for July.
\plainbreak{1}
\normalsize

\lettrine[lines=1]{}{This} able advocate of the Dental profession is filled with valuable
matter, which it would be well for physicians, as well as dentists, to
become acquainted with.\endinput
