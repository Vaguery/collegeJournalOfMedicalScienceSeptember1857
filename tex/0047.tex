<span class=font2>1857.]</span>

<span class=font2><i>Abstracts.</i></span>

<span class=font2><b>431</b></span>

<span class=font0><b>ABSTRACTS.</b></span>

<span class=font2><b>E</b><span class=font0><b>pilepsy.</b></span><span class=font1><b>—</b></span>Dr. Eobert Hunt maintains that an abnormal excess of
alkali in the blood directly predisposes the nervous system to disease
and not only this but that it also causes various chemical changes in'
the blood, which result in the generation or retention of noxious matters
in the system, which excite a tendency to spasms and convulsions. In
epilepsy, where there is an uniform deficiency of organic principles, as
urea, and an excess of mineral matters, especially chloride of sodium,
in the urine, they have some connection w<sup>7</sup>ith the state of the blood and
the disease in question, and in such cases he has effected cures by keep-
ing the bowels regular, administering twenty drops of diluted nitro-
muriatic acid two or three times a day, before meals, with the use of a
dilute nitro-muriatic acid bath every night, remaining in it 15 or 20
minutes each time; subacid fruits are to be used freely.</span>

<span class=font2><b>H</b><span class=font0><b>emoptysis.</b></span><span class=font1><b>—</b></span>The only remedies to be trusted in severe cases are oil
of turpentine, gallic acid, chloride of sodium, or nitre with digitalis;
alum is not sufficient to meet the danger. Blood-letting temporarily
arrests the bleeding, but is dangerous, owing to the debility and increased
susceptibility to the intercurrent affections it gives rise to. In using the
gallic acid or turpentine in severe cases we should first procure a transient
check of the hemorrhage by ligatures to the limbs and ice to the chest,
allowing the internal means to consolidate this temporary cure.</span>

<span class=font2><b>H</b><span class=font0><b>emorrhagic </b></span><b>D</b><span class=font0><b>iathesis.</b></span><span class=font1><b>—</b></span>Give full diet: mutton chop, eggs, milk,
etc.; as a medicine, gallic acid five grains, dilute sulphuric acid twenty
minims, decoction of cinchona an ounce and a half, mix for a dose; to
be repeated three or four times a day. If the hemorrhage be from the
mouth, apply turpentine with a sponge three times a day. If the surface
of the skin be extensively ecchymosed, give dilute sulphuric acid 20
minims, sulphate of iron four grains, sulphate of magnesia half a drachm,
infusion of columbo an ounce and a half; mix for a dose, and repeat
three times a day. Should there be restlessness, give at night muriate
of morphia one-third of a grain, lactucarium four grains.</span>

<span class=font2><b>B</b><span class=font0><b>right's </b></span><b>D</b><span class=font0><b>isease.</b></span><span class=font1><b>—</b></span>Liquor Ammoniae Acetatis, by acting on thelskin
and relieving the congested condition cf the kidney, has been found the
best of all remedies for this disease ; diuretics at the same time to be
strictly avoided, but solvents may be used.</span>

<span class=font2><b>P</b><span class=font0><b>iles.</b></span><span class=font1><b>—</b></span>When associated with protrusion of the rectum, may be
treated with nitric acid ; it is much safer and quite as effectual <b>a
</b>remedy as the ligature or knife, and there is much less danger from
phlebitis.</span>\endinput
