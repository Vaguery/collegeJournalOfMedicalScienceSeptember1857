<FONT?><b>1857.]</b>

<FONT?><i>Notes and Observations, by </i><b>D</b><FONT?><b>r. </b><b>M</b><FONT?><b>iller.</b>

<FONT?><b>407</b>

<FONT?>which had been dissolved from one ounce to one ounce and a half of
caustic potash. Many other physicians have tried these remedial
measures for the treatment of Asiatic cholera and have spoken highly
of it.

<FONT?><b>M</b><FONT?><b>ichael </b><b>Y</b><FONT?><b>on </b><b>Y</b><FONT?><b>isanik </b>speaks in regard to this mode of treatment
in the following manner : " A very favorable result has been procured
by the Yalerianate of Ammonia,by adding a scruple to three ounces of
distilled w<sup>r</sup>ater. We tried it in <i>sixteen </i>selected dangerous cases, of
wdiich one third showed already symptoms of asphyxia. We gave in
the beginning one tablespoonful every quarter of an hour, but aftei
the system became affected with the remedy w<sup>T</sup>e gave the medicine
each half hour or every hour. After twelve or fifteen hours, and the
use of a few doses of the medicine, the pulse which before could not
be felt would appear again, the discharges and the cramps would cease,
and the skin would acquire its natural color, elasticity and feeling,
become moderately moist, and the patient present a tendency to sleep.

<FONT?>As soon as the use of the remedy had produced a turgescence of the
face, and symptoms of congestion of the brain were presented, we
ceased to longer use the valerianate, and by lifting the head up and
applying cold water these symptoms were checked. In this w<sup>T</sup>ay, of
the sixteen patients we saved <i>ten, </i>and <i>five </i>died. One case could not
be made to take the medicine. Others have complained that their
patients could not be made to retain the medicine, but our experience
convinces us that the remedy is one deserving every attention, and
should be recommended for further trials.

<FONT?><b>B</b><FONT?><b>leeding in </b><b>P</b><FONT?><b>regnancy.</b><FONT?>---The celebrated author and practitioner,
Dr. K. G. Neumann, expresses himself in regard to bleeding in preg-
nancy in the following manner:

<FONT?>" While pregnant women do not continue to menstruate, some old
women of either sex in and out of the profession have imagined that
impurities must accumulate in the system unless <i>bleeding </i>is resorted to
to furnish the desired outlet. These imaginings are certainly foolish,
and while we can excuse women for entertaining them, we certainly
cannot excuse physicians for entertaining and perpetuating this folly.

<FONT?>Such physicians should, as often as they bleed pregnant women,
have several pounds of their own blood drawn off, so that they should
soon die for the benefit of humanity. The blood which is supplied by
the vital forces is required for the formation and perfection of the fœtus,
and it is a <i>crime </i>to waste it, and thus rob the unborn innocent of its
most precious patrimony."\endinput
