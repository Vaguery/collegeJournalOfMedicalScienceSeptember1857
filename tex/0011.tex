<span class=font2>1857.]</span>

<span class=font2><i>Observations on Sanguinaria Canadensis.</i></span>

<span class=font2>395</span>

<span class=font2>when she presented herself at the clinic—and in twelve hours time
she had a free alvine evacuation from the use of Sanguinaria, well tritu-
rated with white sugar and given in small doses every two hours.
Dr. R. is fully persuaded that blood-root is an admirable adjuvant in
all prescriptions for the restoration of healthful function in the liver,
and especially when constipation is coincident.</span>

<span class=font2><b>N</b><span class=font1><b>ote</b></span><span class=font0>.—</span>Perhaps no indigenous plant has attracted more attention
from those physicians who are accustomed to notice the living speci-
mens of materia medica as they spring up in the woods and fields
than the one under consideration. The early appearance of its beauti-
ful and pure blossom, the dark blood-color of its fleshy root, its marked
taste and its prompt action on the system all lead to it3 obtaining the
attention which has been bestowed upon it.</span>

<span class=font2>A trial of its therapeutic virtues has led those who have made use of
it to speak of it in the highest terms of praise and to earnestly recom-
mend it to the favorable notice of the profession, and yet, strangely, it
has never obtained that prominent position in the list of medicines all
its advocates think it deserves.</span>

<span class=font2>Nearly every writer on Botany and Materia Medica in our country
has delighted to give a full description of this plant and to speak highly
in praise of its beauty and usefulness. Among the earlier writers who
have made mention of it Dr. Shoepf says that fifteen or twenty grains
of the pulverized root will produce powerful emesis, but that it must not be
given in the form of a powder as thus it is apt to produce great irritation
of the fauces. He prefers a decoction or the pill form. Merat says it
is useful in gonorrhoea. Shoepf also mentioned the value of a weak
decoction of the root in gonorrhoea and refers to the fact that Golden
had found it useful in jaundice. In doses sufficient to produce emesis
it was found to dislodge worms from the stomach. Thatcher, in his
Dispensatory, speaks of the use made of it by Dr. Dexter in doses of
one grain of the powder or ten drops of the saturated tincture, as a
stimulant and diaphoretic. Dr. Downy was of the opinion that the
dose as recommended by Drs. Shoepf and Colden was larger than could
be administered with safety. In speaking of the value of the root in
jaundice Dr. Thatcher says it was believed to be the chief ingredient
of the quack medicine known as <i>Rawsorfs Bitters.</i></span>

<span class=font2>The younger Barton thinks that the only form in which the blood-
root should be used is that of a spirituous tincture. In this form he
used it in connection with the tincture of bitter-plants as a tonic with
great satisfaction. He also found it useful as a wash for old indolent
ulcers and sores with hardened edges and an ichorous discharge. He</span>
\endinput