
\oldpage{390}In addition to the changes here specified others were noted but they
were supposed to be caused by the presence of some other modifying
disease, and hence not attributable to the cholera and not to be
accounted as a pathological result of the epidemic.

Pathologico-chemically, it was found that the blood was relatively
and absolutely poorer, or more deficient in water, having an appearance
resembling mud.

It was also quite deficient in alkalinity, particularly in the \typo{triple}{tripple}
phosphates, and the carbonate of soda. There was also often a deficiency
of the carbonate of ammonia which it is well known has equal
power to influence the coagulability of the blood and the integrity of
the red corpuscles.

In all instances it was found that the cholera blood chemically was
closely allied to putrescent blood, and readily made to undergo the
putrefactive ferment, far more easily than healthy blood.

The evacuations were all found to be rich in water, and in the alkalinity
of which the blood was deficient, particularly the tripple phosphates
and the carbonate of soda, while they contained but a trace of
albumen. Occasionally in the bladder would there be found a little of
the blue coloring matter mixed with chlorides and the earthy phosphates,
while under the microscope could be discerned in the sediment the tuff
cylinders and the epithelium which had been discharged from the lining
of Bellini's small urin-ducts.

The secretions from other parts of the body have not been as carefully
examined as they should be, but thus far have furnished only
negative results.

If now we consider the changes produced in cholera as here described
are not always uniform, or of an equally marked character, but that they
depend upon the force of different influences—that epidemic cholera not
unfrequently occurs with entire absence of vomiting or purging, but
with an extraordinary amount of \typo{perspiratory}{prespiratoy} exudation, or with spams
that speedily cause death—that in spasmodic cholera the anti-spasmodics
are generally found useful—that not a few cholera patients die from
want of what is called reaction, even where there was no appearance of
decomposition of the blood or deprivation of serum in the blood vessels,
we must come to the conclusion that the first impression of the cause of
cholera is sometimes made upon the blood and at other times upon the
nervous system, while in more rare instances it may impress both the
blood and the nerves at the same time.

The question as to why the serum or watery portion of the blood\endinput