<span class=font0>1857.]</span>

<span class=font0><i>Application of Croton Oil to the Eye.</i></span>

<span class=font0>417</span>

<span class=font0>mation of the eyes, which kept hirn from his business about two months.
He did not entirely recover from the disease this time, and in 1855
there was another acute attack w<sup>T</sup>hich lasted some four weeks, leaving
his eyes still affected. About a year ago his eyes became again sud-
denly inflamed, and the swelling and pain made him completely blind.
He was purged, bled repeatedly, cupped, blistered, and had a variety of
washes applied to his eyes, and treated on this plan of no plan until the
beginning of winter, when the lids were thickly studded with hard, irri-
table granulations, and similar hard and large granulations had sprung
up over the sclerotic conjunctiva ,and the pannus threatened to cover over
the entire cornea. There was much pain and intolerance of light, pro-
fuse lachrymation, and a free discharge of muco-purulent matter.</span>

<span class=font0>He was leeched, the lids were scarified, purgatives were administered,
and the nitrate of silver applied regularly to the eyes for some weeks.
After this the nitrate of silver was alternated with the sulphate of cop-
per, and warm cataplasms were applied. After a time he improved
but did not get well, and soon there was a relapse.</span>

<span class=font0>In April the attendant surgeon determined on inoculating the eyes
with the <i>virus </i>of <i>Gonorrhœa, </i>a Germanic transcendental mode of
treatment, apparently an offspring of the Hahnemannic school. The pa-
tient was not informed in regard to the nature of the virus of inoculation
but was told &quot;that it was a new preparation called <i>glandola.&quot;</i></span>

<span class=font0>The introduction of the Gonorrhoea virus produced very violent in-
flammation. On the third day &quot;the lids were enormously swollen and
purple, and the whole side of the face and neck erysipelatous—the dis-
charge was excessive, and he w<sup>r</sup>as racked with intense neuralgic pain in
the eyebrow. There was, at this time, no possibility of seeing the globe of
the eye, in consequence of the extreme tumefaction and acute pain
where the lids were touched.&quot;</span>

<span class=font0>The after treatment consisted in washing the eye with lead water, and
a tw<sup>?</sup>enty grain solution of the nitrate of silver, an occasional purgative,
and morphia. After many weeks the eyes improved, the pupils were
free, the cornea had a soreness, and the sight was improving daily.</span>

<span class=font0>We are half promised that the Surgeon who treated the above case,
will publish a paper <i>on the indications for inoculation, </i>and for one,
I should be pleased to have him show wherein his &quot;new preparation
called glandola&quot; is superior to the Croton oil mentioned by Dr. Rose.</span>

<span class=font0>The European papers recently make frequent mention of this novel
mode of treating eye diseases, but it will take more than a European
reputation to commend the method to the favor of American Surgeons.</span>

<span class=font0><b>vol</b>. ii. <b>no</b>. 9.—27.</span>

<img src="p - 0033-1.png" alt="" style=" width:19.30pt; height:12.67pt;">
