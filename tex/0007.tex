\oldpage{391}should escape into the stomach and bowels to produce the rice-water
discharges may be answered by referring to the inevitable result of severe
congestion of the lymphatics, as is also shown in the pouring out the
serum upon the surface of the skin in the excessive perspiration which
is sometimes present.

Tbe serum of the blood dissolves the epithelial cells of the alimentary
canal and these dissolved and partially dissolved cells are what gives to
the fluid its peculiar or ricy appearance.

\vspace{\baselineskip}

Is the cholera miasm, or \emph{sui generis}, independent of the miasms which
produce other epidemic diseases?

Many physicians and natural philosophers have held that the cholera
miasm is but the product of the receding of some other form of disease
or rather a modification of a miasm which had produced some other
form of disease, and they have endeavored to sustain this position by
referring to the fact that an epidemic of cholera is usually preceded by
an epidemic of a different character. Others have considered that it
possesses an individual and independent character, unaltered by changes
and unaffected by climates, everywhere acting upon the alimentary canal
and on which, therefore, it must make its first impression.

Those who entertain this latter view consider the cholera miasm a
peculiar miasm, and call the cholera epidemic \emph{the epidemic of epidemics}
or the producer of epidemics, and the cholera miasm the miasm of
miasms, or the producer of miasms.

As has before been remarked, all miasms which produce epidemic
diseases have somewhat in common, but each also has something peculiar
or specific, and hence while the cholera has many characteristics
manifested in other epidemics, that it has an individuality of character
and an individuality of cause cannot well be denied.

The common characteristics which we observe in epidemics arise
from the fact that all miasms are of telluric and atmospheric origin,
and that all miasms in course of time have their power and influence
modified and changed. Yet they all nevertheless manifest essential
peculiarities of character and produce by a specific process each its own
individual disease. For instance, one miasm will produce scarlet fever,
another measles, and another cholera. If there is none of the specific
miasm there will be no measles, or no cholera, as the case may be.
Neither can one miasm produce another disease, for measles never produced
cholera, or cholera measles, or anything else but cholera.

This is the necessary result of the peculiar and specific character of
each individual miasm which possesses its own specific power and disposition.\endinput