<span class=font1>1857.]&nbsp;<i>Extraclum Niootianiœ Rademacheri.&nbsp;</i>* 401</span>

<span class=font1>** As I was at the time out of the extract of tobacco I made a trial of
the Lobelia, but I obtained some from my brother and about two
weeks since I commenced its use. In six days the cough and the expec-
toration entirely ceased. I have since visited the patient and although
the symptoms are so much relieved, auscultation does not promise much
for the final recovery &lt;?f the patient. Too many persons had prescribed,
and the lungs are too much diseased to allow much hopes of a permanent
cure; but this case illustrates the power of the agent.</span>

<span class=font1>I am the more urgent to induce the profession to make a trial of
this extract, as I think it is nearly or quite unknown to the physicians
in this country.</span>

<span class=font0><b>aqua  nicotians tabacum  sperituos^] radamacheei.</b></span>

<span class=font1>This preparation is recommended highly in affections of the brain
accompanying fever, in <i>rheumatismus acutus fixus at vagus, </i>in other
affections of the brain and spinal marrow, in cholera morbus, and in
cholera Asiatica.</span>

<span class=font1>To prepare it: Take of choice fresh green leaves of Nicotianse ta-
bacum eight pounds, and cut them finely. Add of the best alcohol,
by weight one and a half pounds, of distilled water as much as is
necessary to distill over eight pounds (by weight) of the water.</span>

<span class=font1>The leaves are to be cut and the distillation effected immediately
after they are pulled, with great care that there shall be no over-heating
of the liquid, as, if the liquor be over heated it will have a very dis-
agreeable odor of tobacco, which it does not have when the water is
properly prepared.</span>

<span class=font1>Eademacher uses this water in every stage of the Asiatic cholera.!
In the earlier stages he gave the following:</span>

<span class=font1>Aqua Purse, f 3 vij.,</span>

<span class=font1>Soda Acet., 3 jss.,</span>

<span class=font1>Aqua Nicotian., f 3j.,</span>

<span class=font1>Gumi Arab., 5ss.
M.   Dose, one table-spoonful every hour.</span>

<span class=font1>The great majority of cases treated with this mixture recovered
immediately from the attack. In those cases where the attack was fol-
lowed with a typhoid condition, he gave:</span>

<span class=font1>Tinct. Ferri Acetici, f 3 j., ''.[</span>

<span class=font1>Aqua Nicotian., f 3j.,</span>

<span class=font1>Aqua Puree, f 3vj.,</span>

<span class=font1>Gumi Arabici, 3.
<b>M.   </b>Dose, one tea-spoonful every hour.</span>

<span class=font1><b>vol</b>. ii. <b>no. </b>9.-26.&nbsp;v&nbsp;- &quot;<sup>J</sup></span>\endinput
