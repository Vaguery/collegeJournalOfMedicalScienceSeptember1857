<span class=font1>1857.J&nbsp;.^^J <i>J;  Queries Answered.&nbsp;</i><span class=font0><b>'       ' 4£{J.</b></span></span>

<span class=font1>rilled with rarified air serve them a good purpose in night, and in
mammals in supporting their ponderous heads.</span>

<span class=font1>The horse is subject to attacks of acute inflammation of the membrane
lining these cavities, and among farmers the disease is called &quot; horse
distemper.&quot; The inner structure of the horns of kine is liable to take
on the same disease, and then it has the appellation of &quot; horn-ail.&quot;</span>

<span class=font1>Considering that a large portion of the human face is taken up with
antrums and sinuses, all lined with a membrane extending from the nasal
cavities, it is not surprising that &quot; coryza&quot; and &quot;catarrh&quot; so often pre-
vail. Existing in the chronic form, the symptoms of catarrh become
somewhat varied and complicated. From sympathy of continuity the
disease extends itself through the nasal duct and the lachrymal canals,
and affects the conjunctiva, which, together with a congestion of the
vessels about the origin and along the course of the optic nerves, inter-
feres with vision. The patient is unable to read or use the eyes upon
minute objects, for much time, without dimness or a blending of objects
being the result. Hearing is impaired in a manner before hinted at,
and the patient is often treated by pretending &quot;aurists&quot; with appli-
cations to the external ear, leaving the real cause entirely overlooked,,
while any laryngeal trouble is nursed as &quot; bronchitis.&quot; Frequently the
congested state of the lining membrane of the frontal sinuses will pro-
duce headache, which is mostly confined to the region over the eyes and
about the temples—nervous headache, the patient calls it—and it is apt to
recur periodically, once or twice a day. A feeling of heat and pressure
at a point half way between the crow<sup>T</sup>n and forehead, directly ov'er the
sphenoidal sinuses, is not uncommon. There is a dry, unpleasant sensa-
tion in the anterior nares, and by dilating these openings the septum nasi
will be observed redder than natural. The patient feels a disposition to
&quot; hem&quot; in]order to relieve the fauces and a quantity of mucus mixed with
globules of the same in a more condensed form will be brought into the
mouth by the effort. Every morning the throat has to be cleared of
&quot; phlegm&quot; by coughing and other efforts. In several cases mucus finds
its way down the oesophagus, exciting nausea and favoring accumula-
tions of gases in the stomach ; and from the proximity of the heart to the
stomach its functions are interrupted, causing the patient at night some-
times to spring from bed as though suffocation were about to result from
a heart disease.</span>

<span class=font1>The treatment of chronic catarrh, as given by authors, is meagre and
unsatisfactory in its results. By many physicians the disease is pro-
nounced incurable, yet they recommend their patients to snuff up the
nose cold water and use astringent gargles.   Fumigation by directing</span>
