<span class=font1><b>1857.]</b></span>

<span class=font1><i>Book Notices.</i></span>

<span class=font1><b>425</b></span>

<span class=font0><b><i>Domestic Medicine. </i>A Treatise on the Practice of Medicine, adapted to the Reformed
System, comprising a Materia Medica, with numerous Illustrations, By J. Kost, M. D.,
etc., etc., etc.   Cincinnati, R. Wilson, 1857.</b></span>

<span class=font1><b>F</b><span class=font0><b>rom </b></span>the date upon the title page one would naturallv suppose this
book another recent addition to the literature of the Reformers in Med-
icine, but a cursory examination of the volume leads to the conviction
that it is a simple unchanged reprint of the work copyrighted by the
author some six years since.</span>

<span class=font1>Prof. Kost has written tw<span class=font0><b><sup>7</sup>0 </b></span>or three other works, which have had
a fair circulation, and has been engaged as a teacher in several
medical colleges, and hence the author and his productions are very
well known to the profession. In this work was embodied much of what
was known at the time it was written of the principles and practice of
that branch of medical reforms to which the author has been more in-
timately connected, and hence it is of interest as indicating the very
great progress <b>R</b><span class=font0><b>ational </b></span><b>M</b><span class=font0><b>edicine </b></span>has made during the past few years.</span>

<span class=font0><b><i>An Epitome of the American Eclectic Practice of Medicine; </i>embracing Pathology, Symptom-
atology, Diagnosis and Treatment. Containing also a complete list of the remedies used
by Allopathists, Homœopathists, and an Eclectic Pharmacy and Glossary. Designed
for Physicians, the Student of Medicine, and as a Domestic Practice for Families. By
William Paine, M. D., Professor of the Principles and Practice of Mediciue and Pa-
thology in the Eclectic Medical College of Pennsylvania. Philadelphia, H. Cow-
ferthwait and Co. 1857.</b></span>

<span class=font1><b>F</b><span class=font0><b>rom </b></span>the many subjects embraced in this title-page we might natu-
rally expect a library of many volumes ; but instead the author has been
able to present what he had to&quot; offer on these important matters in a
single twelve-mo. volume of about four hundred pages. Whether our
Allopathic, Homoeopathic, Hydropathic, and Chrono-thermal neighbors
will think full justice has been done them in this little manual or not,
we cannot determine. We are sure, however, that physicians, students
and families will agree with us in regretting that the majority of the
subjects have been so briefly considered.</span>

<span class=font1>We have repeatedly expressed our objections to the use of this class
of works for students and physicians, as tending to produce a careless,
indifferent or superficial consideration of matters of the deepest scientific
and humanitary interest, and a consequent empirical and unsuccessful
practice. But children usually creep before they walk, and walk
before they are able to run, and hence we may reasonably anticipate
these primal efforts to supply the profession with the demanded pro-
gressive literature will partake somewhat of a juvenility of character.
Even failures may, however, be considered successful in one point of
view, as the masts of sunken vessels serve as indicators of the pres-
ence and location of the rocks upon which they were stranded.</span>\endinput
