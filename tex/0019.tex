<span class=font2>1857.]</span>

<span class=font2><i>Pleuritis, Latant, by </i><b>D</b><span class=font1><b>r. </b></span><b>W</b><span class=font1><b>itham.</b></span></span>

<span class=font2>403</span>

<span class=font2>effect a cure; but after a trial of several weeks this prescription was
discarded, as no change had resulted. Lancinating pain would occasion-
ally be felt in the chest, slight cough, expectoration streaked slightly
with blood. He continued to perform light work and had not been
confined to his bed. I found him presenting the following symptoms
five months after the first appearance of the disease.</span>

<span class=font2>The mucous membrane of the pharynx presented a pale and debili-
tated appearance ; the chest inclined forward, the body assuming a
stooping position; great tenderness of the spine from the first cer-
vical vertebra to the last dorsal; pressure over the lungs, liver, stomach
and spleen gave pain. In short no part of the chest nor abdomen
could be percussed without revealing deep-seated tenderness. The
skin was dry, pulse quick ; there w<sup>T</sup>as much dyspnoea with abdominal
respiration. Percussion of the lungs gave rather a dull sound. Bowels
torpid. I diagnosed the disease to be Latent Pleuritis complicated
with chronic inflammation of the larynx which gave rise to the Aphonia.
As the patient was of a strumous diathesis and the disease of long stand-<sup>5
</sup>ing I doubted the efficacy of treatment but advised it and took charge
of the case on the 12th of June.</span>

<span class=font2>I first ordered morning bathing to be practiced daily, the water used
to be impregnated with chloride of sodium and bicarbonate of potassa.
Internal treatment:</span>

<span class=font2>#   Podophyllin, 3ss., <span class=font0>x</span></span>

<span class=font2>Capsicum, gr. X.,</span>

<span class=font2>Ext. Taraxicum, q s.
M. f. Pill, No. X.   Take one of these pills morning, noon and night
until the bowels are freely moved, then take but two a day.</span>

<span class=font2>Comp. Syr. Stillingia, fSiij.,</span>

<span class=font2>Iodide of Potassium, 3j.
<b>M. </b>Take one teaspoonful four times a day. To test the progress of
the case I saw the patient daily. I discovered no change until the third
day; the bowels were then active, less tenderness about the cervical
vertebra, could whisper with less pain and more distinctly. On the
fourth day still more improved. I ordered the same treatment continued
and on the next day the patient recovered full power of speech and could
talk freely and without pain. He continued to improve and on the tenth
day of treatment I could discover no abnormal symptoms. Percussion
over the abdominal and thoracic viscera was no longer painful; no ten-
derness of the spine could be detected. I now discontinued the former
treatment excepting one pill to be taken each day, and prescribed the
following:</span>
