<FONT?><b>1857.]</b>

<FONT?><i>Book Notices.</i>

<FONT?>427

<FONT?>a certain unity with the Divine Nature;" * * * " that the salva-
tion of man---his transformation from evil to good, or his passage from
a state of nature to a state of grace---was symbolized under the figure
of the transmutation of metals, * * * that the Alchemists were
<i>Reformers </i>in their time, * * that they were religious men when
the spirit of religion was buried in forms and ceremonies, and when
the priesthood had armed itself with the civil power to put down all
opposition, and suppress all freedom, intellectual, civil, moral and re-
ligious." Hence these <i>freemen </i>were accustomed to w<sup>7</sup>rite and speak
to each other in symbolic language, using the terms salt, sulphur, gold,
mercury, etc., while refering to principles and truth.

<FONT?>According to the writer of this volume, the whole subject of Alchemy
is Man, but each writer for the most part designated man by a word of
his ow<sup>T</sup>n choosing. Hence one called man Antimony, another Lead,
another Zinc, etc., and under these and similar names we can if we
choose to investigate the matter, learn what those writers thought of
God, Nature, and Man, or Man, Nature and God---one and three, three
and one. That the Alchemists were Protestants, when Protestantism
could not find open free speech for fear of the <i>auto da fe, </i>the dungeon
or the stake, seems apparent from their entire writings, and that their
opinions were expressed to each other in quaint language, distinctly
understood by each other, but unintelligible to the uninitiated : that
Alchemy, or Hermetic Philosophy, was a religious philosophy or faith,
and the writers on this subject teachers of the new---or protestant form
of religion, rather than of new ideas in chemistry.

<FONT?><b><i>The Physician's Visiting List, Diary, and Book of Engagements for </i>1858. Philadelphia,
Lindsat and Blakiston.</b>

<FONT?><b>T</b><FONT?><b>he </b>united voice of all who have enjoyed the facilities and advant-
ages of this Pocket Account Book and Diary has been only in praise
of this convenient Annual, which has become a <i>necessity </i>to those who
have used it.

<FONT?>It contains an Almanac, a Table of Poisons and their Antidotes
(rather too brief), a Table for calculating the period of Utero Gestation,
a Visiting list for the year, for 25 or 50 patients daily, and a vast
amount of other convenient memoranda, by means of which physicians
avoid errors, or forgetfulness---and by the aid of which many times
the cost of the work may be saved daily.

<FONT?>This Visiting List is bound in a convenient form, and should be found
in the pocket of each practicing phys ician. Copies can be sent by mail
for $0,75, and $1,00.\endinput
