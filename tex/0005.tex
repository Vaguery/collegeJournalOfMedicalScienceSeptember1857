as \emph{cholera spasmodica}; while in other climates, vomiting and purging
are the most prominent features of the disease.

It is' also important, in the investigation of the cause of cholera, to
learn to distinguish between the genuine cholera and other forms of
disease, as well as to decide what results are produced from cholera
and what from other causes, and it is only those who have had considerable
experience in this disease who can always make this distinction.

The pathologico-anatomical, and the pathologico-chemical results of
the disease \typo{contribute}{coutribute} largely to the knowledge requisite to determine
the cause and nature of it, and they therefore must never be neglected
or overlooked. The post-mortem examinations and the examinations
of the secretions and excretions must be made, particularly the secretions
and excretions of the liver, the spleen, the kidneys, the stomach,
and the intestines, but more particularly the excretions of the kidneys,
and on the results of these examinations may we base our own view of
the nature and cause, as well as of the treatment of the diseases.

During the last epidemic the post-mortem examinations have furnished
in general and in particular, the same results as those obtained
during former epidemics. The skin of those who died of the disease has
been cyanotic, or blue colored, particularly the skin over the extremities.
The vessels in the sinuses and meninges of the brain have contained
much thick dark blood. The inner meninges have been congested, with
ecchymoses. The brain itself has been firm, and on intersection has
disclosed similar ecchymoses in its substance. The pleura and pericardium,
and all the serous membranes have a slippery feeling, showing a
separation of the delicate lining from the subjacent parts, and covered
with a glutenous albuminoid fluid. The lungs are dry and of a clear
red, and bloodless. The heart, particularly the left ventricle has been
drawn up, and in it and in the large vessels we have found a thick
black, tarry blood possessing little or no power of coagulation. The
liver is pale, and the gall-bladder filled with much dark bile. The
spleen is usually enlarged, dark red-brown, and its enveloping membrane
thrown into wrinkles. The stomach and intestines are filled up
with a rice-watery, or a bloody colored fluid, with the mucus membrane
of the stomach swollen and injected. The epithelium of the intestines
in nearly the whole extent is dead, and rubbed off; the mucus membrane
red, and the follicles swollen. The kidneys in most instances
presented distinctly the changes which are found in the disease known
as \emph{morbus Brightii}. This changed condition was particularly found in
those who had died of that form of cholera known as the \emph{cholera-typhus}.
The bladder was found contracted and empty.
\endinput