<span class=font1>400&nbsp;<i>Extractum Nicotiania Rademacheri. </i>[September,</span>

<span class=font1>the genuine lung cough, serves as a diagnostic as to the real nature of
the disease. If the cough originates from the lungs it will be bene-
fitted by the extract, while if it owes its origin to a disease of some
other part of the system, the extract may fail of benefitting the
patient. But there may be coughs which in reality are caused by some
diseases of the lungs, and yet the extract may not prove beneficial.
For instance, a cough may be caused by a node, or from a closed or
an open abscess in the lungs, or from the pressure of a fractured rib
upon the pulmonary tissue and yet the extract would not produce a
cure. The extract has a favorable influence upon idiopathic but not
on secondary coughs. With opium we can often relieve secondary or
sympathetic coughs. We do not with that agent obtain a cure, but
we do obtain relief from the cough, and moderate it or pacify it. The
extract of tobacco is not as active as opium to <i>allay </i>a cough, but far
more powerful to cure it when of the genuine lung origin.</span>

<span class=font1><b>I</b><span class=font0><b>diopathic bleeding of the lungs</b></span>. When I speak of bleeding of the
lungs I mean to be understood that form of the disease which is com-
monly called <i>spitting of blood, </i>where a greater or less quantity of clear
blood, or blood mixed with phlegm, or phlegm streaked with blood,
will be raised from the lungs. The extract is valuable in these cases,
but may not be depended upon in <i>Pneumorrhagia, or Apoplexiapul-
monalis, </i>in which latter form of disease we must resort to the use of
allum and ice internally and cold wet cloths to the surface of the
chest, and to other appropriate remedial measures.</span>

<span class=font1>I would here remark that this preparation will not produce the vom-
iting and purging which follows the administration of the dry tobacco,
and I have never used the dry tobacco as an emetic or an injection, as I
find the Lobelia inflata an equally efficient remedy.</span>

<span class=font1>I was called a few days since to see a patient where many other reme-
dies had been tried by three eminent physicians who had attended on
the case, without avail. The patient had been sick quite a length of
time but owing to my recent illness and the distance from me I could
not treat it. The case presented the characteristics of consumption, a
harrassing cough, with bloody sputa, etc. As I was unable to visit the
patient I was consulted by letter, and had ordered inhalations, and di-
rected the Wild Cherry, Lycopus Virginicus, and Lobelia combined
with Ipecacuanha, without benefit. I used the Lobelia, from having
found it of great value in cramps and affections of the chest, and
particularly in phthisis pulmonalis. For these purposes, and to re-
lieve the dry harrassing cough and tickling of the throat, it is in use
by many German physicians.    ^ ** ,</span>
