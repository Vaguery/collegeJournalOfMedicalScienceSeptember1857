
Some also, have considered cholera to be caused by some derangement
of the chylopoietic vicera, but the anatomical and autopsical examinations
have given no countenance to this conclusion.

The peculiar symptoms manifested in the asphyctic condition of
cholera has induced some to suppose that the whole difficulty arose
from a weakness or loss of functional power of the nerves, but more
especially of the spinal cord; but the revelations of the dead-house
have not confirmed these conclusions and did much to disprove the
accuracy of the opinion entertained.

More recently in their search for the seat and origin of cholera
physicians have been guided by what they have learned in regard to
the blood, its changes and decompositions, and they have observed that
in cholera there is a partial decomposition of the blood, with contemporaneous 
alteration in the walls of the capillaries by means of which
the \emph{serum sanguinis} in large quantities passes through their walls, or
is poured directly into the stomach and intestines, leading to the profuse
vomiting and purging by which this serum is removed entirely beyond
the organism. This decomposition of the blood and the outflow of
its watery portions is often produced with wonderful rapidity---while
the more solid portions, as the red-corpuscles, is retained in the vessels,
and thus the fluid is rendered thick, dark and very liable to stagnation,
to clog up and produce congestion of the vessels. The thick blood
also stagnates in the vessels of the skin, and causes the blue appearance
nearly always observed in that tissue. In the larger veins, and in the
brain, in the liver, and the spleen, and the lungs this stasis also occurs,
and hence the difficulty of breathing, the deafness, the aphonia, the
thirst, the scanty urine, the coldness and numbness, that accompanies
this disease.

Although this explanation is in accordance with the observations of
the profession, yet it may not satisfy all, for many deny that any explanation
can be given which shall prove satisfactory, and they desire
to know how it is, if the blood is \typo{really}{realy} separated, that a part of the
serum is not thrown into the cellular tissue, and not the whole of it
poured into the alimentary canal, or through the skin in profuse perspiration.
Why are not the pleural and abdominal cavities filled with this
fluid? why is the patient not attacked with hydro-thorax and anasarca?
are questions urged against this opinion.

There can be no doubt but the nerves which supply the vital force to
the walls of the blood-vessels are impaired during an attack of cholera
but those who entertained the opinion that the nervous power is diminished,
are divided as to which is the \emph{prime} cause of the difficulty, one