<span class=font2>1857.&nbsp;<i>Tincture of Gdseminum in Dysentery.</i></span>

<span class=font2><b>R</b><span class=font1><b>emarks.</b></span><span class=font0>—</span>I am pleased to know that my efforts to extend a knowledge
of this agent are already successful to a considerable extent, and I
gladly respond to the request for farther information.</span>

<span class=font2>In regard to the value of the Tincture in Gonorrhoea I am not at
present prepared to advance an opinion, as neither my own experience
or that of my friends, has presented a sufficient amount of results on
which to base any absolute opinion.</span>

<span class=font2>In regard to its use in other diseases, perhaps it will be found to be
possessed of other properties in addition to its power as a sedative to
the heart, which will prove it to be of great value in many instances.
The splanchnic system of nerves doubtless govern the secreting organs,
as well as the processes of chemical change and nutrition, and when
these functions, as well as that of circulation are performed too ac-
tively, great harm may result and the agent which is capable of
moderating, checking or controlling these changes, may be found to
possess more valuable remedial properties than have been heretofore
suspected.</span>

<span class=font2>In the article quoted by Dr. Mayes, I said : &quot; I am satisfied that as
a sedative to the nerves branching from the spinal cord and going to
the organs of locomotion, or the nerves of voluntary motion ; and in a
lesser degree to the vagus and sympathetic nerves that are distributed
to the heart and lungs, inducing a less powerful and less frequent pulse,
and a more sluggish and feeble respiration, the Gelseminum will prove
highly satisfactory to any who may give it a trial.&quot;</span>

<span class=font2>I also accord wdth the remarks made by Dr. Mayes, as quoted in the
<b>J</b><span class=font1><b>ournal</b></span>, p. 187, except that I think as the agent impresses, as has
been stated, the <i>Exito-Secretory </i>nerves, it is capable of diminishing their
undue activity, as in Gonorrhoea and Dysentery and other forms of un-
due activity and excitability of those nerves, and hence it will prove
not only a valuable adjuvant to other treatment, but also a direct rem-
edial agent of no inconsiderable value in a very large number of dan-
gerous and painful diseases, including inflammations of the brain, the
lungs, the pleura, the viscera and in rheumatism, and various disorders
of the fluids of the body.</span>

<span class=font2>But before we can determine the actual value of this potent agent
we need the results of many carefully made trials of it, cautiously noted^
and frequently repeated, and we hope to be favored with these from
all who have made such observations and have noted the results ob-
tained. C.</span>\endinput
