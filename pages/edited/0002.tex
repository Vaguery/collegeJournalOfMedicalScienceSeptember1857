
\oldpage{386}It would prove fatiguing to the reader for me to enumerate all the
various hypotheses which have from time to time obtained in regard to
the etiology of cholera, and hence I shall only glance at a few of the
more prominent of them as prefatory to the opinions and conclusions to
which I have arrived.


It has been observed that fat, and particularly the fats that are to be
found in diet where the food has become sour and rancid, will if eaten
often produce symptoms very closely analagous to cholera, and from
this observation the conclusion has been drawn that the cholera miasm
is one produced by a specific decomposition of animal tissues forming
a combination of gases similar to those evolved in the decomposition
of sausages, and which is known as the \emph{sausage poison}. That this
view cannot be correct has become evident to nearly all.


Others have supposed that cholera is caused by what they have been
pleased to style the \emph{cholera-mite} a supposed microscopic animalcule
diffused in vast quantities through the air, the food and the drink, and
that these animalcules are the \foreign{potentia nocens} of the disease. The
advocates of this hypothesis attack all other opinions and defend their
own with great violence, and are very strenuous in the advocacy of
what they assert to be the cause of cholera. Among those of this
class are many able microscopists, and yet they have neglected to bring
forward the only strong and indisputable evidence necessary to establish
the accuracy of their deductions. They are unable to bring forward
any person who has been so fortunate as ever to have seen this
wonderful cholera-mite.


Not a few have directed their attention mainly to the stomach and
the intestines and think they find in the vomiting and purging the true
explanation of the cause of epidemic cholera; which cause to them is
the irritative and congested condition of the alimentary track. On
this hypothesis have they based their high estimate of the value of
opium and have viewed it as a specific against the disease.


This view of the matter is so superficial and so illy sustained by the
symptoms of the disease and the results of treatment that most have
abandoned it. All who have had any experience in cholera and its
treatment will have observed that the danger of the attack is by no
means proportionate to the activity of the vomiting and purging, but
that it frequently appears extremely severe and fatal where but little
vomiting or purging have occurred. But this class of persons have
their attention so closely drawn to their fancied seat of the difficulty
that they never perceive these facts and never have the faintest glimpse
of the true cause of the disease.\endinput