\oldpage{388}class supposing the atony of the vessels and the consequent out-flux
of the fluid impairs the nerves, while the other supposes that a poison
has been introduced into the system which acting directly on the brain
thus lessens the nerve power, and the loss of that power leads to the
atony of the blood-vessels and consequent exhalation of the serum.

There seems to be \emph{three} principle classes of opinions as to the primal
nature of cholera. 1st.\ A primary poisoning of the blood. 2nd.\ A
primary affection of the nerves. 3d.\ A primary local affection of the
alimentary canal.

From what has been said, we may perhaps draw the conclusion, that
cholera, like other epidemics, as scarlatina, measles, intermittent fever,
typhus fever, influenza, etc., owes its origin to a cause having a uniform
origin, or at least a uniform character, while, as in the other instances,
as to its peculiar nature, we may be entirely ignorant. Many physicians
have striven hard to learn the exact nature and character of this morbific
agent, and yet they have acknowledged a want of success.

With others, I too, have tried to solve this mystery and having had
an opportunity of observing personally more than two thousand cases
of cholera in different epidemics, I am led to present the following observations.

Cholera does not appear every where and at all times to possess precisely
the same characteristics. At one time it will appear mild and
very easily cured. But this slight form only appears in those persons
whose systems appear to have but a slight \emph{disposition} for the disease,
and hence it cannot exert as powerful an influence upon such as it does
upon those who are more predisposed to its attacks, and the disease will
take \emph{the form of cholera periculosa exquisita}.

Most of those who are disposed to inflammatory disease seem also
disposed to receive cholera, and hence the two diseases are often met
with in company. In ``\foreign{cholera febrilis}'' there are several congestions
of the head, the lungs, or heart, in conjunction with the more ordinary
symptoms of cholera. In persons who have a predominating disposition
to vomit, the cholera will commence with vomiting, while with
those who are disposed to looseness of the bowels, it will commence
with a diarrhœa, while with those who are predisposed to the cholera,
and at the same time their nervous and arterial systems are equally
susceptible, the disease will take the form of \foreign{cholera fulminitisima
asphyctica}.

Climate exercises upon any prevailing disease a powerful influence.
This is manifested in epidemic cholera. In some countries and climates,
it appears as \foreign{cholera febrilis}, with intense congestions; in others\endinput