\oldpage{426}

We anticipate that Prof.\ Paine will be called upon to issue a second
edition of his work, and in anticipation of that event, we would suggest
that in regard to his nomenclature for remedies he may be able to
make some corrections of errors and introduce some improvements.

In accordance with the usage of modern Chemists, all names of
Alkalies and Alkaloids should terminate with an \emph{a}, as in Soda, Potassa,
Quinia, Morphia. The names of all Acids terminate in \emph{ic}, as Acetic,
Tannic, Gallic, Sulphuric, Nitric. The names of all Resins and Resinoids
terminate in \emph{in}, as Resin, Rosin, Podophyllin, Jallapin. The
names of such agents as may belong to one of these four classes, but
whose character is as yet undetermined by scientific writers, are made
to terminate in \emph{ine}, as Cornine, Santonine, Salicine, Piperine.

In looking over the list of medicines named in this work we are sorry
to perceive that these simple and truly scientific rules have been disregarded.
Hence we have ``Cinchon\textit{ine}'' as the name of an alkaloid
principle, instead of Cinchon\textit{ia}; ``Emet\textit{ine}'' defined as the alkaloid
principle of Ipecacuanha; ``Myric\textit{in}'' as an alkaloid; ``Quin\textit{ine},''
and other similar errors, which can readily be corrected in future editions.

In conclusion we would cordially welcome, and heartily exert the influence
of our \textsc{Journal} to assist all publications which are calculated to
enlighten and elevate the profession and advance the cause of humanity.

\fancybreak{* * *}

\footnotesize
\booktitle{Remarks upon \typo{Alchemy}{Aclhemy} and the Alchemist}, indicating a method of discovering the true
nature of Hermetic Philosophy; and showing that the search after the Philosopher's
Stone had not for its object the Transmutation of Metals. Being also an attempt to
rescue from undeserved opprobium the reputation of a class of Extraordinary Thinkers
in past ages. ``Man may not live by bread alone.'' Boston; \textsc{Crosby}, \textsc{Nichols}, and
\textsc{Company}. 1857.
\plainbreak{1}
\normalsize


\lettrine[lines=1]{}{Some} two years since, the writer of the present volume published a
small pamphlet on the subject of Alchemy, in which he presented the
idea that the \emph{Philosopher's Stone} of the old Hermetic Philosophers
was not a supposed reality, but a symbol of \emph{wisdom} or \emph{truth}.

That pamphlet being designed solely for the friends of the author, was
never offered for sale. It attracted the attention of a writer in the
\emph{Westminster Review}, who did not aim at the same conclusion with
the pamphleteer, and hence the author was called upon to defend his
position. He therefore pursued his investigation by reading still
other works on Alchemy, and this farther investigation only confirmed
him in the opinion that although the Alchemists were the forerunners
of modern Chemists, yet their investigations were not mainly directed
to material matter, but that ``the \emph{subject} of Alchemy was \emph{Man}; while
the \emph{object} was the perfection of Man, which was supposed to center in\endinput
