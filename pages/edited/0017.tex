\oldpage{401}

As I was at the time out of the extract of tobacco I made a trial of
the Lobelia, but I obtained some from my brother and about two
weeks since I commenced its use. In six days the cough and the expectoration
entirely ceased. I have since visited the patient and although
the symptoms are so much relieved, auscultation does not promise much
for the final recovery of the patient. Too many persons had prescribed,
and the lungs are too much diseased to allow much hopes of a permanent
cure; but this case illustrates the power of the agent.

I am the more urgent to induce the profession to make a trial of
this extract, as I think it is nearly or quite unknown to the physicians
in this country.

\begin{center}\textsc{aqua nicotiaiæ tabacum sperituosæ radamacheri.}\end{center}

This preparation is recommended highly in affections of the brain
accompanying fever, in \emph{rheumatismus acutus fixus at vagus}, in other
affections of the brain and spinal marrow, in cholera morbus, and in
cholera Asiatica.

To prepare it: Take of choice fresh green leaves of Nicotianiæ tabacum
eight pounds, and cut them finely. Add of the best alcohol,
by weight one and a half pounds, of distilled water as much as is
necessary to distill over eight pounds (by weight) of the water.

The leaves are to be cut and the distillation effected immediately
after they are pulled, with great care that there shall be no over-heating
of the liquid, as, if the liquor be over heated it will have a very disagreeable
odor of tobacco, which it does not have when the water is
properly prepared.

Rademacher uses this water in every stage of the Asiatic cholera.
In the earlier stages he gave the following:
  \begin{tabbing}
    \prescription. \= Aqua Puræ, f \ounce vij., \\
      \> Soda Acet., \ounce jss., \\
      \> Aqua Nicotian., f \ounce j., \\
      \> Gumi Arab., \ounce ss. \\
  \end{tabbing}
M.\quad{}Dose, one table-spoonful every hour.


The great majority of cases treated with this mixture recovered
immediately from the attack. In those cases where the attack was followed
with a typhoid condition, he gave:
  \begin{tabbing}
    \prescription. \= Tinct. Ferri Acetici, f \ounce j., \\
      \> Aqua Nicotian., f \ounce j., \\
      \> Aqua Puræ, f \ounce vj., \\
      \> Gumi Arab., \ounce. \\
  \end{tabbing}
M.\quad{}Dose, one tea-spoonful every hour.
\endinput
