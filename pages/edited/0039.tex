\oldpage{423}Society. This offer of a prize has led to an animal competition among
the best writers of the country, which competition has brought forth
some very valuable Essays upon Medical topics of great practical importance.

The two Essays now issued in this volume were first printed in the
\booktitle{American Journal of Medical Science}, and from its pages have they
been reprinted for the purpose of presenting them to the profession in
a more permanent form, and giving them a more extensive circulation.

After presenting the opinions of the most eminent pathologists in
regard to \emph{the nature of pulmonary tuberculization}, Dr.~Lee comes to
the conclusion that ``tuberculization is a disease depending upon an
alteration of the blood from its normal condition.'' * * * ``Principally
caused by suppression or diminished action of the functions of the
skin and a deficiency of the red corpuscles, and that consequently it
should not be considered as merely a local disease but requires to be
treated with reference chiefly to the disordered condition of the blood
and to the causes which have been instrumental in producing it, before
it has arrived at so advanced a stage as to preclude all rational hopes
of recovery.'' * * * ``It is therefore against the diathesis, or the
cachectic state of the system, and not against its local manifestations
that our remedies should be directed.''

In regard to the \emph{effects of climate}, in the treatment of tubercular
diseases, the author presents many valuable facts which he has embodied
in twenty separate cases. He says:

``6.\quad{}The chief indications in the treatment of pulmonary tuberculization
by means of climate, are first to remedy as far as possible the
morbid condition of the blood which constitutes the cachectic state, and
by this means to prevent or arrest the formation of the morbid product;
and secondly, to allay the general and local excitation caused by
the organic lesion. These indications are not unfrequently opposed to
each other and in many cases the practitioner is obliged to restrict himself
to endeavoring to fulfill the second, and to palliate the symptoms
by pharmaceutical remedies.''

Although much has been said of late in favor of a high northern
latitude, and Dr.~\textsc{Kane} has expressed the opinion that no one living
among the Esquimeaux will be likely to die of pulmonary tubercular
disease, Dr.~\textsc{Lee} does not seem to have had his attention drawn to this
matter.

In regard to the \emph{influence} of \emph{pregnancy}, which is the subject of Dr.~\textsc{Warren}'s
Essay, much has been said and yet but little of a reliable character
has been recorded, except in scattered fragments and isolated remarks.

Dr.~\textsc{Warren} commences his essay with a quotation of the opposing\endinput
