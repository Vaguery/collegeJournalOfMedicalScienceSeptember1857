\oldpage{416}into the nostrils the fumes of burning sugar, ginger and cinnabar,
has comprised a part of the treatment for catarrh, but it is attended
with too little success to favor repetition.

Dr.~Ira Warren, of Boston, for many years has douched the nostrils
and pharynx with a solution of nitrate of silver, employing a syringe
with a long curved nozzle.

Instead of any of the above treatment, I would recommend the patient
to take in some convenient vehicle the muriate or chloride of gold
in one-twentieth grain doses three times a day, and make frequent use
of an errhine composed of pulverized kalmia angustifolia and sassafras,
equal parts. Tincture of Bryonia inhaled, often proves serviceable in
this disease, and when there is much pain in the head chloroform may
be added to the Bryonia. Deafness arising from obstructions of the
Eustachian tubes may be relieved by douching the passages with a dilute
tincture of Arnica flowers.\hfill{}H.\quad{}

\fancybreak{* * *}

\section*{Accidental Application of Croton Oil to the Eye.}

by \textsc{Wellington Rose,\ m.~d.}

\SectionStartWords{A very} respectable lady, of a plethoric habit, and sanguine temperament,
aged about 60 years, had for two or three years been afflicted
with an infirmity of her eyes. The sight was becoming dim and black
specks were apparently flying before them.

She poured some Croton Oil from one phial into another, an some
of it adhered to one of her fingers, with which she indiscreetly rubbed
the eye which was the most affected. That eye and eyelid immediately
began to smart and burn very severely. Sweet cream was first applied
to it but gave no relief. Olive Oil was next used, and the pain soon began
to subside, and ere long all disappeared. The black specks also immediately
disappeared, and her eyesight has been more clear and strong
since the accident than it had been for a long time previous. This I narrate
as it may serve as a useful hint to physicians in regard to the treatment
of some diseases of the eye.

\textsc{Remarks}---I have delayed publishing the above for the purpose of first
obtaining the report of a case treated in this city. The patient's father
died of consumption, and his mother now suffers from cough and other
pulmonary difficulties. He and others of the family had sore eyes in 1846
from which he recovered after a few weeks. In 1852 he again had inflammation\endinput
