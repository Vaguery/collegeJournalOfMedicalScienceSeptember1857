THE
COLLEGE JOURNAL
OF MEDICAL SCIENCE.

Vol. II.
SEPTEMBER, 1857.
No. 9.

\chapter*{Original Contributions.}

\section*{Nature and Causes of Asiatic Cholera.}

by Prof.\ \ProperName{Michael von Visanik}.

Translated from the German by \ProperName{T.~C.\ Miller}, \md

\SectionStartWords{Ever} since the first appearance of the epidemic cholera physicians
have endeavored to investigate and discover the inner nature and true
cause of the disease, that upon the knowledge thus obtained they might
base a rational mode of treatment.

But in accordance with the prevailing usages of the profession, the
nature and causes of this fell disease have remained unknown and the
attempted explanations have been based upon hypotheses and empirical
observation and not upon any rational and accurate course of
observation and reasoning. Some have explained the etiology as
a ``\textit{virus}'' others as ``\textit{humores alienati},'' and have contented themselves
with these unintelligible abstractions as sufficient to account for
the origin and duration of the contagion and disease. But we have
recently become dissatisfied with these phrases in place of ideas, and
since the researches of the pathologist and the chemist have given us
more definite information, the reformatory truths obtained from them
have enabled us to penetrate far more profoundly into the true nature
of this epidemic.

% \textsc{vol. ii. no. 9.---25.}
\endinput