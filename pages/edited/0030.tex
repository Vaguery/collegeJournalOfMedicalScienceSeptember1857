\oldpage{414}\chapter*{Queries Answered}

\section*{``Catarrh.''}

\lettrine[lines=1]{}{I will} make the following query to which I would be pleased to get
an answer in the next No.\ of the \textsc{College Journal}. I am annoyed considerably
with noises in my ears, it being however almost entirely confined
to the left one. Sometimes the noise is of a ringing character; at others
there is a roaring and rushing. It is almost constant, yet frequently
more marked in the evening and aggravated by every slight cold I take.
I have no pain in the ears but sometimes a slight itching; I have been
very subject to irritation of the larynx from the slightest check of perspiration.
My tonsils also were formerly irritated and somewhat enlarged,
but by means of astringents and stimulants locally applied they
were reduced and have not given me any trouble for some months. My
general health is good, but I have been somewhat troubled with papular
eruptions on the face. I have at times felt more or less dizziness when
raising my head suddenly after having bowed down; and not long since,
after having taken a slight cold, the noise was greatly augmented in my
ear and I became quite dizzy and for a short time (after a somewhat full
meal), was unable to walk straight. My hearing has been slightly affected.
Now what is most probably the difficulty and what course of treatment
would you recommend? I might have mentioned that I never had acute
inflammation of the ear either external or internal, nor have I been subject
to headache.

\textsc{Answer.}---Your annoyances probably arise from chronic inflammation
of the mucous membrane of the nares and pharynx; the ``ringing'' in the
ears is produced by a partial obstruction of the Eustachian tube, preventing
a free passage of air from the throat to the ear. The laryngeal difficulty
depends upon the same disease which is continued into the vocal
organ; and the dizziness is consequent upon a congested state of the vessels
of the head, the blood being determined thither in undue quantities
by the irritation existing in and about the pharynx. The disease, in the
acute form called ``coryza'' and ``influenza'' is a common one, especially
in countries subject to great atmospheric vicissitudes.

The lower animals are subject to a similar disease. It is well known
that within the heads cf mammalia there are extensive pneumatic cavities
communicating with the mouth and nose. These cavities in man
are called, ``antrums,'' as that of Highmore; and ``sinuses,'' as the frontal,
ethmoidal and sphenoidal sinuses. The elephant and the owl derive
considerable reputation for intellectual profundity from the prominence
given by these airy cells. Pneumatic cavities in the bones of birds\endinput
