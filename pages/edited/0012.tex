\oldpage{396}
had also used the powdered root as an application to fungoid growths
and nasal polypi. Bigelow, and Dr.~Smith also used it for the same
purpose. So also Dr.~Shanks and Dr.~Israel Sterling, according to
Thatcher, used it in place of digitalis in coughs and pneumonic complaints.
Dr.~Darwin has used it in peripneumonia trachealis in the
form of a decoction and from the benefit thence derived Dr.~Barton
thought it must be a useful medicine, particularly in cynanche maligna,
in cynanche trachealis and other similar affections.

Drs.~Barton and Downy said that the \emph{leaves} of the puccoon as well as
the seeds are possessed of a \emph{narcotic} power similar to that of the seeds
of the stramonium and that they had produced dangerous symptoms.

In 1831 Daniel~B.~Smith published in the \booktitle{Journal of the Philadelphia
College of Pharmacy} a dissertation on this plant, in which,
he gives its natural and botanical history and speaks of the experiments
made by Dr.~Dana on the root in 1824, when the \emph{Sanguinarina} was
probably first obtained.

Dr.~Tully has carefully examined the medicinal powers of blood-root
and thinks it is therapeutically allied to squills, seneca, digitalis, guaiacum
and ammoniacum.

More recently Dr.~Williams, formerly of Massachusetts but now of
Illinois, has written several valuable essays on the Sanguinaria, but
unfortunately I have lost the reference to them and I only remember
that he considered it one of the most valuable if not the most valuable
of all the North American plants.

Dr.~Thom of Ohio, in a communication to the \booktitle{Western Journal},
says that for two years he had been closely engaged in observing the
effects of this remedy in various diseases and he concludes that it is a
\emph{sedative} of no ordinary powers. For reducing the force and frequency
of the pulse without prostrating the system he considered it one of the
most efficient remedies. He also styled it an \emph{alterative} with a marked
influence on the liver and the glandular system generally. He employed
it in hemorrhage from the lungs, particularly in those cases where
the hemorrhage appeared to be caused by vicarious menstruation, and
considered it of more value than any other agent he had used.

Dr.~M'Bride in the \booktitle{South.\ Jour.\ of Med.} said he considered this
plant eminently serviceable in those disorders of the liver where the
secretion of the bile is either suppressed, deficient or vitiated. In imperfect
convalescence after bilious fever he says, ``the puccoon is the
best remedy.'' As an emmenagogue he thought highly of it. He
recommended it as a substitute for mercury.

Dr.~J.~L.\ Mothershead used it in dyspepsia in the form of pills,\endinput
