\oldpage{417} of the eyes, which kept him from his business about two months.
He did not entirely recover from the disease this time, and in 1855
there was another acute attack which lasted some four weeks, leaving
his eyes still affected. About a year ago his eyes became again suddenly
inflamed, and the swelling and pain made him completely blind.
He was purged, bled repeatedly, cupped, blistered, and had a variety of
washes applied to his eyes, and treated on this plan of no plan until the
beginning of winter, when the lids were thickly studded with hard, irritable
granulations, and similar hard and large granulations had sprung
up over the sclerotic conjunctiva, and the pannus threatened to cover over
the entire cornea. There was much pain and intolerance of light, profuse
lachrymation, and a free discharge of muco-purulent matter.

He was leeched, the lids were scarified, purgatives were administered,
and the nitrate of silver applied regularly to the eyes for some weeks.
After this the nitrate of silver was alternated with the sulphate of copper,
and warm cataplasms were applied. After a time he improved
but did not get well, and soon there was a relapse.

In April the attendant surgeon determined on inoculating the eyes
with the \emph{virus} of \emph{Gonorrhœa}, a Germanic transcendental mode of
treatment, apparently an offspring of the Hahnemannic school. The patient
was not informed in regard to the nature of the virus of inoculation
but was told ``that it was a new preparation called \emph{glandola}.''

The introduction of the Gonorrhœa virus produced very violent inflammation.
On the third day ``the lids were enormously swollen and
purple, and the whole side of the face and neck erysipelatous---the discharge
was excessive, and he was racked with intense neuralgic pain in
the eyebrow. There was, at this time, no possibility of seeing the globe of
the eye, in consequence of the extreme tumefaction and acute pain
where the lids were touched.''

The after treatment consisted in washing the eye with lead water, and
a twenty grain solution of the nitrate of silver, an occasional purgative,
and morphia. After many weeks the eyes improved, the pupils were
free, the cornea had a soreness, and the sight was improving daily.

We are half promised that the Surgeon who treated the above case,
will publish a paper \emph{on the indications for inoculation}, and for one,
I should be pleased to have him show wherein his ``new preparation
called glandola'' is superior to the Croton oil mentioned by Dr.\ Rose.

The European papers recently make frequent mention of this novel
mode of treating eye diseases, but it will take more than a European
reputation to commend the method to the favor of American Surgeons. \hfill{}C.

% vol</b>. ii. <b>no</b>. 9.---27.
\endinput
