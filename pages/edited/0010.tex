\oldpage{394}\section{Observations on the Uses of Sanguinaria
Canadensis.}

\byline*{\ProperName{Abr'm.\ Livezey}, a.~m., \md}

\SectionStartWords{In} several medical journals I have taken the liberty to call the attention
of the profession to some of the uses of our indigenous medicinal
plants, and in the present communication I beg leave to offer some
remarks upon the medicinal value of the Sanguinaria---a plant incident
to all localities and the root of which is easily gathered.

Without prejudice to the use of any other article I feel warranted in
saying, from no little experience, that this plant with the aid of podophyllin
will exert a more happy influence in all hepatic derangements---both
as a cholagogue purgative and as an alterative---than any combination
of calomel.

Possessing undoubted nauseant, sedative and alterative properties,
blood-root will in cases of slight inflammation of the biliary organs, or
congestive states of the same, or where a species of spasmodic action
pervades those structures, give prompt relief; and where torpidity
exists and the physician thinks that the stimulant action of some mercurial
is indicated he need only combine a minute portion of the
podophyllin to obtain all the advantages that are supposed to be
derived from calomel.

Sanguinaria gives a decided aid to the action of podophyllin or any
other cathartic to which it is added.   It is, in the form of tincture, an
alterative expectorant in chronic bronchitis.   It is valuable in chronic
hepatitis combined with ext.\ taraxicum and ext.\ podophyllum, jalap or
rhei, if obstinate constipation exists.    Tinct.\ Sanguinaria can with
advantage be substituted for wine of antimony in the \emph{brown mixture}
and wherever the wine of antimony is used.   As a substitute for the
compound cathartic pill the following combination---already published
will generally prove more satisfactory:

\begin{center}
\begin{tabbing}
  \prescription. \= Podophyllin, \= gr. \= i., \\
    \> Leptandrin, \> ''\> iv., \\
    \> Sanguinaria, \> ''\> ii., \\
    \> Ext. Taraxicum, q.\ s.\ Misce.\ ft.\ pil.\ No.\ iv. \\
\end{tabbing}
\end{center}
Two or three for a cathartic; ½ to a whole one night and morning as a
hepatic alterative.

A graduate student of mine, Dr.\ Rice, late resident physician in the
W.~C.\ Infirmary of Philadelphia, had a case of obstinate constipation
which had persisted four weeks---so said the patient, an Irish woman,\endinput