\oldpage{398}granulations and other morbid growths, and it is more than probable
the blood-root which has been added to various ointments and
applications which have been used upon cancerous affections has done
much toward effecting the cure.

A reference to the use of blood-root in the cure of cancer is now
causing considerable discussion in various sections of the country, particularly
in New England, and where much is being said in regard to
who was the physician who first used it for the cure of cancer. My
own opinion is that it was in use by the people and the \emph{country} physicians
long before we have any record of its being thus applied.

From the very imperfect abstract here given of a few of the articles
that have been published in our periodicals on the use of this root, we are
warranted in drawing the conclusion that it is a very valuable medicine
and should be introduced into more general use. But doubtless one
reason for its neglect is the fact that the root rapidly looses its value by
age and if kept more than one year may become nearly worthless.

The tincture and other preparations should be made from the root as
soon as possible after it is gathered and not from the old and nearly
worthless specimens usually sold by druggists.

In regard to the preparations sold under the names of \emph{Sanguinarin}
and \emph{Sanguinarina}, although I have had frequent letters of inquiry
addressed to me, I cannot give any satisfactory answer. I have no
means of knowing what these preparations are or how manufactured,
and of those who have used them I have never been able to obtain any
evidence of their character or value as therapeutic agents, but a friend
of mine who has manufactured these articles and sold them in considerable
quantities has told me, that as the result of his own observations
and the observations of those of the profession who, had bought and
used them, he was fully convinced they were of even less value than
the pulverized root. I consider it a duty I owe to the readers of the
\ThisJournal{Journal} to present these facts.

If those who manufacture these agents would let us know enough
about them to warrant us in making a trial of them, and if those who
have used them would carefully observe their action and notify us of
the result, soon the readers of the \ThisJournal{Journal} would be in the possession
of the required information. In the present state of the case the only
answer I can give is that I have never used them and know nothing
positive about them. \hfill{}C.\endinput
