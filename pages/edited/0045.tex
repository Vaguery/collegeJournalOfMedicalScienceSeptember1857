\oldpage{429}

\fancybreak{* * *}
\footnotesize
\booktitle{Demand of the Age on Colleges.}---Speech Delivered by the Hon.~\textsc{Horace Mann}, President of
Antioch College, before the Christian Convention, at its Quadrennial Session, held at
Cincinnati, Ohio, October 5, 1854. New York. \textsc{Fowler} and \textsc{Wells}, 1857.
\plainbreak{1}
\normalsize

\lettrine[lines=1]{}{This} is the product of a Master Mind, governed by a Master Will,
which leads to a bold, fearless enunciation of whatever is considered as
\emph{truth}, particularly those truths which have a bearing upon the spiritual
or physical welfare of the human race.

Those who give utterances to \emph{positive} and more particularly \emph{original}
ideas, are apt to find far more opponents than followers at first, and
hence, probably, but few will be ready to take the same views as our
author on the first perusal of this little work, and yet most will read it
the second and the third time, and whether they adopt the opinions
advanced or otherwise, they must be benefitted thereby.

We most earnestly wish that such Speeches as this and the Address
to the Students of Antioch College, which is added as an Appendix,
together with the \booktitle{Address to the Students of the Eclectic College of
Medicine}, by Prof.~I.~J. \textsc{Allen}, M.~D., L.L.~D., a part of which was
published in the \textsc{College Journal} in March last, could be read by every
Medical Student in the land.

That our readers may judge of the character of this Speech, we
make the following brief extracts:

``Let me say, then, in a single sentence, that our hope and aim is to
\emph{meet not merely the advanced, but the advancing Demands of the Age}.
What, then, does the age demand that our College should be? or rather,
in the first place, what does the age demand that it should not be?

``It should not be an Egyptian pyramid, for the preservation of old
mummies, literal or psychological. Whatever has vitality in it, whatever
has truth in it, these let us religiously preserve, for Truth is endued
with immortal youth and beauty, and can give forever and to all without
self-exhaustion or impoverishment. But as for the mummies, let
the Arab peasants continue to burn them, as travelers tell us they are
now accustomed to do, for cooking their dinners. Would to Heaven
that all the tyrants of the present day, political and mental, could be
put to as good a use.

``Dugald Stewart likens some of the literary institutions of his time
to old hulks sunk in the stream, which by their stationary position
show to the passers-by how far the living have advanced beyond the
dead on the River of Progress. We do not desire to enter into any
competition with those old hulks for the honor or repose of their conservatism.
Among the moral surveyors who are measuring the onward
march of mankind, we would aspire to be found among the foremost
chainbearers, pressing right forward in defiance of any obstacle, and
up any acclivity; and let those who come after keep the tally.\endinput
